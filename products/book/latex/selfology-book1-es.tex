% Selfology Book 1: Тематические программы
% Полная книга - 29 программ, 190 кластеров
% Сгенерировано: 2025-12-03 21:05
% Язык: es

\documentclass[11pt,a4paper,oneside]{book}
\usepackage{selfology-book}
\setdefaultlanguage{spanish}

\begin{document}

\bookcover{29 Programas\\de Autoconocimiento}{Preguntas que cambian tu vida}{2025}
\booktoc

\thispagestyle{empty}
\vspace*{0.2\textheight}

{\centering\sffamily\bfseries\fontsize{24pt}{28pt}\selectfont Bienvenido\par}

\vspace{2em}

Este libro no es un test ni un cuestionario. Es un espacio para encontrarte contigo mismo.

Aqui no hay respuestas correctas o incorrectas. Solo las tuyas---honestas, reales, vivas.

\vspace{1.5em}
{\sffamily\bfseries Como funciona este libro}

Encontraras tres niveles de preguntas:

\textbf{Superficie} --- preguntas ligeras para calentar. Te ayudan a sintonizar y conectar contigo mismo.

\textbf{Exploracion} --- preguntas mas profundas. Aqui comienza el verdadero trabajo. Puede resultar incomodo---es normal.

\textbf{Profundidad} --- las preguntas mas importantes. Requieren fuerza, tiempo y un espacio seguro.

\vspace{1.5em}
{\sffamily\bfseries La metafora del acuario}

Imagina que tu mente es un acuario.

Arriba---agua cristalina. Alli puedes ver claramente tus pensamientos, deseos y decisiones. Pero esta parte representa solo el 10--20\%. Es tu mente consciente.

El 80--90\% restante es agua turbia en el fondo. Es dificil ver algo alli. Es tu subconsciente---el lugar donde viven las verdaderas razones de tus decisiones, miedos y deseos.

Este libro es como un colador. Sacaras del fondo pensamientos y sentimientos que normalmente permanecen invisibles.

\newpage
\thispagestyle{empty}
\vspace*{0.1\textheight}

{\sffamily\bfseries Como proceder}

\textbf{Crea espacio.} Elige un momento en que nadie te moleste.

\textbf{Sintoniza.} Sirvete tu bebida favorita. Ponte comodo. Date permiso para ser honesto.

\textbf{Escribe a mano.} Cuando escribes a mano, tu cerebro procesa la informacion mas profundamente.

\textbf{No te apresures.} Date todo el tiempo que necesites. Una pregunta puede tomar un minuto o media hora.

\vspace{1.5em}
{\sffamily\bfseries Pautas de seguridad}

Necesitaras energia para este viaje.

No intentes completar todo de una vez. Avanza a tu propio ritmo. Toma descansos.

Si una pregunta despierta emociones fuertes---es una senal de que has tocado algo importante. Acercate con delicadeza.

\vspace{2em}
{\centering\itshape Listo? Comencemos.\par}

\chapter{Pensar en~la~Vida}
\programbreak

\cluster{Aqu y~Ahora}{5-10 min}{dnde estoy ahora}

\questionF{1. Cmo describiras tu~estado actual con tres palabras?}

\questionF{2. Cmo es normalmente tu~maana?}

\questionF{3. Qu haces cuando tienes tiempo libre y~nadie espera nada de~ti?}

\questionF{4. Cundo fue la~ltima vez que~te sentiste completamente lleno de~energa y~vitalidad? Qu ocurra a~tu alrededor?}


\cluster{Personas a~Mi Alrededor}{10-20 min}{quin me~influye}

\questionE{1. Quin ha tenido una~influencia particularmente fuerte en~ti? Qu fue exactamente lo~que esta persona hizo o~te dio?}

\questionE{2. Con quin has estado pasando ms tiempo ltimamente?}

\questionE{3. Estas personas te~ayudan a~crecer y~desarrollarte, o~te mantienen en~el~mismo lugar?}

\questionE{4. De qu nunca hablas con tu~familia?}

\questionE{5. En qu sientes que~eres nico/a en~comparacin con tu~familia?}


\cluster{Sueos y~Ambiciones}{10-20 min}{a~dnde quiero llegar}

\questionE{1. Cmo imaginas tu~da libre ideal? Dnde pasas el~tiempo, qu haces y~con quin?}

\questionE{2. Con qu sueas cuando te~imaginas la~libertad financiera?}

\questionE{3. Qu es lo~que quieres aprender en~primer lugar?}

\questionE{4. A dnde quieres ir y~por qu precisamente ah?}

\questionE{5. Por qu razn estaras dispuesto a~madrugar tanto?}

\questionE{6. Imagina que~has logrado cambiar el~mundo para mejor. Qu ha cambiado especficamente?}


\cluster{Valores y~Creencias}{10-20 min}{lo~que es importante para m}

\questionE{1. Qu valores has definido para ti? Escribe al~menos tres.}

\questionE{2. En qu aspectos te~consideras nico en~comparacin con las personas cercanas a~ti?}

\questionE{3. Completa esta frase: «Si la~gente supiera que~en~realidad yo..., ellos...}

\questionE{4. Te quieres a~ti mismo?}

\questionE{5. Cmo te~relacionas contigo mismo? Con calidez o~de~manera crtica? Qu influye en~esa relacin?}


\cluster{Defensa y~Autoengao}{10-20 min}{de~qu me~escondo}

\questionE{1. De qu tratas de~protegerte al~engaarte a~ti mismo?}

\questionE{2. De qu necesitas liberarte antes de~alcanzar tus prximos logros?}

\questionE{3. Cules son los miedos profundos que~encuentras en~tu vida?}

\questionE{4. Qu es lo~que ms te~asusta de~la~intimidad con los dems?}


\cluster{El~Pasado}{10-20 min}{mi~camino hasta hoy}

\questionE{1. Quin eras hace un~ao? Quin eres ahora? Cmo has cambiado?}

\questionE{2. Cmo calificaras el~ao pasado en~las principales reas de~tu vida en~una~escala del~1 al~10? Qu eventos importantes en~cada rea influyeron en~tu calificacin?}

\questionE{3. Recuerda tres hbitos que~te ayudaron este ao. Cmo exactamente te~apoyaron?}

\questionE{4. Recuerda 2 o~3 ocasiones en~las que~las cosas no salieron como esperabas. A qu cosas buenas te~llevaron finalmente esas situaciones?}

\questionE{5. Qu no lograste este ao y~por qu? Qu piensas al~respecto ahora?}

\questionE{6. Hay algo por lo~que todava te~enojes contigo mismo o~no puedas perdonarte?}


\cluster{Orgullo y~Sentido}{15-30 min}{de~qu me~siento orgulloso/a}

\questionI{1. De qu te~sientes ms orgulloso u~orgullosa?}

\questionI{2. Si~solo te~quedara un~ao de~vida, cmo te~gustara vivirlo?}

\questionI{3. Si~maana todo perdiera su~significado, cul sera tu~mayor arrepentimiento?}


\cluster{Visin del~Futuro}{15-30 min}{en~quin quiero convertirme}

\questionI{1. Imagina tu~futuro siendo la~persona que~quieres llegar a~ser en~lo~que realmente te~importa. Qu consejos te~dara esa versin de~ti a~quien eres hoy?}

\questionI{2. Si~tus padres realmente te~vieran, qu entenderan?}

\questionI{3. Qu deseas ms en~este momento?}


\newpage

\chapter{Pensar en~la~Carrera o~el~Negocio}
\programbreak

\cluster{Dnde estoy ahora}{5-10 min}{mis asuntos laborales}

\questionF{1. En qu rea trabajas o~estudias actualmente?}

\questionF{2. Cuntos aos llevas en~este mbito?}

\questionF{3. Tienes formacin especfica en~tu rea profesional actual?}

\questionF{4. Trabajas en~la~oficina, de~forma remota o~en~modalidad hbrida?}

\questionF{5. Cul es tu~horario de~trabajo actual?}


\cluster{Satisfaccin}{10-20 min}{me gusta mi~trabajo?}

\questionE{1. En~una~escala del~1 al~10, qu tan satisfecho ests con tu~trabajo actualmente?}

\questionE{2. Qu sientes hacia tu~trabajo? Qu resuena en~l y~qu no?}

\questionE{3. En qu medida coinciden tus valores personales con la~cultura corporativa de~la~organizacin donde trabajas?}

\questionE{4. Te sucede que~te criticas a~ti mismo por descansar, incluso cuando trabajas mucho?}


\cluster{Metas y~Sueos}{10-20 min}{mis sueos sobre el~trabajo}

\questionE{1. Cmo sera el~trabajo de~tus sueos?}

\questionE{2. A qu te~dedicaras si~todos los trabajos tuvieran el~mismo salario?}

\questionE{3. Qu haras si~el~dinero no fuera una~limitacin?}

\questionE{4. Imagnate: dentro de~5 aos te~encuentras con un~viejo amigo. Qu le contaras con orgullo acerca de~tu trabajo?}

\questionE{5. Si~supieras que~definitivamente vas a~tener xito, qu empezaras a~hacer ahora mismo?}


\cluster{Barreras y~Recursos}{10-20 min}{qu obstaculiza y~qu ayuda}

\questionE{1. Qu es exactamente lo~que te~impide avanzar hacia tus metas profesionales en~tu trabajo actual?}

\questionE{2. Si~necesitas un~mentor en~tu trabajo, quin podra ser esa persona y~cmo podras encontrar a~alguien as?}

\questionE{3. Si~maana decidieras dejar de~trabajar, cunto tiempo podras vivir de~tus ahorros?}

\questionE{4. A dnde va especficamente tu~dinero ahora mismo: hacia tu~futuro (inversiones, educacin, desarrollo) o~hacia tu~presente (compras, entretenimiento, comodidad)?}


\cluster{Elecciones y~Acciones}{10-20 min}{mis prximos pasos}

\questionE{1. A dnde te~gustara ir a~trabajar si~decidieras hacer un~cambio en~tu carrera?}

\questionE{2. Si~imaginas una~balanza, por un~lado estn los rechazos en~entrevistas de~trabajo y, por el~otro, trabajar en~un~lugar donde no te~valoran. Qu sera ms importante para ti evitar?}

\questionE{3. En qu se~basa tu~confianza en~que tu~idea funcionar? Qu datos, experiencias u~observaciones respaldan tu~decisin?}

\questionE{4. Si~tuvieras todos los recursos necesarios (tiempo, dinero, contactos), qu te~gustara hacer y~cmo emplearas tu~tiempo?}


\cluster{Mtricas de~Negocio (para Emprendedores)}{15-30 min}{Un~clster psicolgico enfocado en~desarrollar el~pensamiento analtico y~las habilidades de~toma de~decisiones basadas en~datos, esenciales para el~emprendimiento exitoso. Ayuda a~los emprendedores a~comprender los indicadores clave de~rendimiento, mtricas financieras y~analtica empresarial para optimizar sus emprendimientos y~tomar decisiones estratgicas informadas.}

\questionI{1. Cunto te~cuesta adquirir un~cliente nuevo para tu~negocio?}

\questionI{2. Cul es el~indicador clave de~rendimiento que~sigues en~tu negocio?}


\newpage

\chapter{Reflexionar sobre la~salud}
\programbreak

\cluster{Sueo y~Recuperacin}{5-10 min}{mi~sueo y~descanso}

\questionF{1. A qu hora te~acuestas normalmente y~a~qu hora te~levantas?}

\questionF{2. Cuntas horas sueles dormir por la~noche?}

\questionF{3. Cmo te~sientes habitualmente por las maanas: lleno de~energa, decado o~somnoliento?}

\questionF{4. Qu sueles hacer durante la~hora antes de~acostarte? Te ayuda a~quedarte dormido o~dormida?}

\questionF{5. Eres ms de~madrugar o~de~trasnochar, y~qu te~aporta esto en~la~vida?}


\cluster{Fisiologa}{10-20 min}{qu como y~cmo me~muevo}

\questionE{1. Cmo entiendes qu necesita y~qu no necesita tu~cuerpo?}

\questionE{2. Cunta agua bebes al~da? Notas alguna conexin entre la~cantidad de~agua que~consumes y~cmo te~sientes?}

\questionE{3. Qu tipos de~actividades fsicas te~traen alegra?}

\questionE{4. Cmo te~relacionas con tu~salud fsica? Qu significa para ti estar sano?}


\cluster{Seales del~Cuerpo}{10-20 min}{lo~que siente mi~cuerpo}

\questionE{1. Qu le sucede a~tu cuerpo cuando ests agotado? Cules son las primeras seales que~te enva tu~organismo?}

\questionE{2. En qu parte de~tu cuerpo se~acumula ms a~menudo la~tensin? Qu intenta decirte tu~cuerpo?}

\questionE{3. Cundo fue la~ltima vez que~escuchaste a~tu cuerpo y~seguiste sus seales en~lugar de~resistir y~continuar adelante?}


\cluster{El~Sentido de~la~Salud}{15-30 min}{para qu necesito estar sano}

\questionI{1. Por qu es importante para ti tener buena salud? Qu beneficios te~aporta?}

\questionI{2. Si~tuvieras una~salud perfecta, qu cambiara en~tu vida?}


\cluster{Salud en~el~Futuro}{15-30 min}{a~dnde quiero llegar}

\questionI{1. Qu cambios te~gustara ver en~tu salud dentro de~un~ao?}

\questionI{2. Dnde te~ves en~trminos de~salud dentro de~un~ao? Cmo te~sentiras?}

\questionI{3. Qu tipo de~descanso necesitan tu~cuerpo y~tu alma? Dnde y~cmo te~gustara recuperar energa?}


\newpage

\chapter{Conocerse a~S Mismo}
\programbreak

\cluster{Arquetipos y~Modelos}{5-10 min}{quin me~influy}

\questionF{1. Quin ha ejercido la~influencia ms fuerte sobre ti? Qu cualidades o~enfoques de~esa persona encuentras en~ti?}

\questionF{2. A quin admiras ms? Por qu precisamente a~esa persona?}


\cluster{Energa y~Recursos}{10-20 min}{de~dnde saco mis fuerzas}

\questionE{1. Cmo describiras tu~estado de~nimo en~este momento con tres palabras?}

\questionE{2. Cundo fue la~ltima vez que~te sentiste completamente lleno de~vitalidad y~energa? Qu ocurra en~tu vida en~ese momento?}

\questionE{3. Cules son las tres acciones o~situaciones que~te devuelven la~energa ms rpidamente?}


\cluster{Valores a~travs del~orgullo}{10-20 min}{de~qu me~siento orgulloso/a}

\questionE{1. De qu te~sientes ms orgulloso/a y~por qu es tan importante para ti?}

\questionE{2. Qu cualidades o~caractersticas tuyas conoces mejor que~nadie, pero~de~las que~rara vez hablas?}


\cluster{Deseos y~Miedos}{10-20 min}{lo~que quiero y~temo}

\questionE{1. Qu es lo~que ms deseas en~este momento?}

\questionE{2. Qu miedo vive en~lo~ms profundo de~tu alma? Qu es lo~que intentas proteger?}


\cluster{Futuro Ideal}{15-30 min}{en~quin quiero convertirme}

\questionI{1. Describe tu~da de~descanso ideal: dnde estaras, con quin estaras y~qu estaras haciendo?}

\questionI{2. Imagina que~has cambiado el~mundo para mejor. Qu ha cambiado especficamente gracias a~ti?}

\questionI{3. Si~supieras que~no puedes fracasar, qu haras?}


\cluster{Desarrollo y~Camino}{15-30 min}{hacia dnde estoy creciendo}

\questionI{1. Qu es lo~que ms deseas aprender ahora mismo y~por qu es importante para ti en~este momento?}

\questionI{2. A qu lugar del~mundo te~gustara ir y~qu es lo~que te~atrae de~all?}

\questionI{3. Qu habilidad o~cualidad te~gustara desarrollar el~prximo ao?}


\newpage

\chapter{Mejorar el~Estado Emocional}
\programbreak

\cluster{Mapa de~Emociones}{5-10 min}{me gusta mi~trabajo?}

\questionF{1. Qu emociones notas con ms frecuencia a~lo~largo del~da?}

\questionF{2. Menciona una~emocin que~te sorprenda por su~frecuencia.}

\questionF{3. Hay alguna emocin que~viva en~tu cuerpo constantemente, como un~ruido de~fondo?}


\cluster{Corporalidad de~las Emociones}{10-20 min}{lo~que mi~cuerpo me~dice}

\questionE{1. Dnde sientes la~alegra en~tu cuerpo? Describe la~sensacin: es calor, ligereza, expansin?}

\questionE{2. Recuerda un~momento en~el~que te~embargaba la~gratitud. Qu sentas en~tu cuerpo?}

\questionE{3. Qu emociones te~brindan una~sensacin de~paz y~armona? Cmo se~manifiestan en~tu cuerpo?}

\questionE{4. En qu parte de~tu cuerpo reside esa emocin que~no puedes expresar con palabras?}


\cluster{Conciencia de~los Sentimientos}{10-20 min}{qu estoy sintiendo ahora mismo}

\questionE{1. Qu tan bien eres consciente de~tus sentimientos cuando aparecen? Califcate del~1 al~10.}

\questionE{2. Te permites simplemente sentir lo~que hay o~a~menudo piensas que~no deberas sentir esto?}

\questionE{3. Qu emociones te~permites sentir nicamente cuando ests a~solas contigo mismo?}

\questionE{4. Cuando alguien te~pregunta qu ests sintiendo, tu primer impulso es responder con honestidad o~decir lo~que esperan escuchar?}


\cluster{Defensas y~Mscaras}{10-20 min}{mis mscaras y~escudos}

\questionE{1. Usas el~humor o~el~sarcasmo para expresar u~ocultar tus sentimientos?}

\questionE{2. Hay emociones seguras que~muestras en~lugar de~tus verdaderos sentimientos?}


\cluster{Emociones como Informacin}{10-20 min}{lo~que mis sentimientos me~susurran}

\questionE{1. Qu te~dicen los sentimientos que~experimentas habitualmente?}

\questionE{2. Si~tu ansiedad fuera un~maestro sabio, qu te~estara enseando?}

\questionE{3. Qu emocin sueles considerar «mala» cuando en~realidad est tratando de~ayudarte?}


\cluster{El~Estado de~nimo como una~Eleccin}{15-30 min}{Cmo manejar tu~estado de~nimo}

\questionI{1. Qu estado de~nimo te~gustara cultivar por la~maana antes de~comenzar tu~da?}

\questionI{2. Qu te~ayuda a~recuperar el~equilibrio cuando las emociones te~abruman?}


\newpage

\chapter{Revisar Objetivos}
\programbreak

\cluster{Inventario de~Objetivos}{5-10 min}{mis objetivos para hoy}

\questionF{1. Enumera todas las metas que~persigues actualmente (profesionales, personales, financieras).}

\questionF{2. De dnde surgi cada una~de estas metas: de~tu propia eleccin o~de~las expectativas de~otras personas?}

\questionF{3. De qu objetivo es hora de~desprenderte, uno que~ya no te~sirve ni~te inspira?}

\questionF{4. Quizs estas metas que~te agobian ya no sean relevantes para ti?}


\cluster{Verificacin de~Objetivos}{10-20 min}{cules de~los objetivos estn vivos}

\questionE{1. Elige una~meta que~sea importante para ti. En qu se~basa tu~confianza de~que precisamente este camino te~llevar al~resultado esperado?}

\questionE{2. En qu datos especficos o~experiencias basas tu~confianza en~esta meta? Qu te~da la~seguridad de~que realmente es alcanzable?}

\questionE{3. Podra existir una~forma ms eficaz de~alcanzar esta misma meta? Qu te~ayudara a~descubrirla?}

\questionE{4. Tus metas te~inspiran o, ms bien, te~generan ansiedad?}


\cluster{Objetivos Profesionales}{10-20 min}{qu quiero del~trabajo}

\questionE{1. Cul es tu~trabajo soado? Imagnalo en~detalle: dnde tiene lugar, qu incluye, con qu tipo de~personas interactas.}

\questionE{2. Qu aspectos de~tu trabajo actual te~estn ayudando a~avanzar hacia tus objetivos profesionales, y~qu ms puedes hacer para seguir progresando?}

\questionE{3. Qu, especficamente, de~tu trabajo actual te~impide alcanzar tus objetivos profesionales?}

\questionE{4. En qu medida te~ayuda tu~trabajo actual a~avanzar hacia tus objetivos profesionales?}


\cluster{Logros y~Orgullo}{15-30 min}{de~lo~que me~siento orgulloso/a}

\questionI{1. De cul de~tus logros te~sientes ms orgulloso/a?}

\questionI{2. Qu logro te~sorprendi ms, algo que~no esperabas conseguir?}


\cluster{Priorizacin}{10-20 min}{qu es lo~ms importante ahora mismo}

\questionE{1. Si~pudieras cumplir solo tres objetivos de~todos los que~tienes, cules seran?}

\questionE{2. Cmo sabrs si~te ests moviendo hacia tus objetivos en~la~direccin correcta?}


\cluster{Accin}{15-30 min}{mi~plan de~accin}

\questionI{1. Qu paso concreto puedes dar esta semana para acercarte a~tu meta principal?}

\questionI{2. Quin o~qu podra ayudarte a~acercarte a~esta meta? Qu recursos y~apoyo necesitas?}


\newpage

\chapter{Soadores}
\programbreak

\cluster{Qu es un~sueo para m}{5-10 min}{qu significa soar}

\questionF{1. Recuerda tu~infancia: con qu soabas entonces? Qu ha cambiado desde aquel momento?}

\questionF{2. Cul es tu~principal sueo en~este momento?}

\questionF{3. Es realmente tu~sueo o~es algo que~una~vez escuchaste de~otros y~adoptaste como propio?}

\questionF{4. Qu tan cerca ests de~vivir la~vida de~tus sueos en~este momento?}


\cluster{Prueba de~Realidad}{10-20 min}{lo~que es realmente alcanzable}

\questionE{1. Conoces a~alguien que~ya est viviendo un~estilo de~vida similar o~trabajando en~un~rea que~te interese? Qu le ayud a~lograrlo?}

\questionE{2. Tal vez tu~sueo se~basa en~informacin limitada? Qu te~ayudara a~comprender su~realidad ms profundamente?}

\questionE{3. Si~tuvieras todas las oportunidades y~recursos disponibles, qu haras primero?}


\cluster{Casa de~Sueos}{10-20 min}{cmo se~ve mi~sueo}

\questionE{1. Si~todos tus problemas estuvieran resueltos y~todo en~tu vida estuviera en~perfecto orden, dnde viviras y~cmo sera la~casa de~tus sueos?}

\questionE{2. Imagina la~casa de~tus sueos. Qu tiene de~especial? Describe de~3 a~5 detalles que~la~hacen verdaderamente tuya.}

\questionE{3. Qu pequeo y~seguro paso puedes dar hoy para acercarte a~la~casa de~tus sueos?}


\cluster{Trabajo de~Ensueo}{10-20 min}{dnde trabajar con alegra}

\questionE{1. Cmo visualizas tu~trabajo ideal? Trata de~describirlo de~la~manera ms especfica posible.}

\questionE{2. Qu tendra que~cambiar en~tu vida para que~pudieras dedicarte al~trabajo de~tus sueos?}

\questionE{3. Quin puede ayudarte a~acercarte al~trabajo de~tus sueos? Qu contactos, conocimientos o~recursos necesitas?}


\cluster{Dinero e~Inversiones}{10-20 min}{adnde va mi~dinero}

\questionE{1. Hacia dnde se~dirige exactamente tu~dinero en~este momento: hacia tu~futuro (inversiones, educacin, activos) o~hacia tu~presente (comodidad, entretenimiento, compras)?}

\questionE{2. Qu porcentaje de~tus ingresos inviertes para hacer realidad tu~sueo?}

\questionE{3. Si~pudieras destinar el~10\% de~tus ingresos a~hacer realidad tu~sueo, en qu lo~invertiras?}


\cluster{Accin}{15-30 min}{mi~primer paso hoy}

\questionI{1. Qu paso concreto podras dar esta semana para acercarte a~tu sueo?}

\questionI{2. Si~no tomas accin, cmo ves tu~vida en~unos aos? Cul sera tu~arrepentimiento ms doloroso?}

\questionI{3. Cundo quieres empezar? Tienes algn da disponible esta semana?}


\newpage

\chapter{Reflexin}
\programbreak

\cluster{Lo~Que Aprend Sobre M Mismo/a}{5-10 min}{Lo~que entend sobre m mismo/a}

\questionF{1. Cul fue el~descubrimiento ms inesperado que~hiciste sobre ti mismo?}

\questionF{2. Qu descubriste sobre ti mismo que~no sabas antes de~esta reflexin?}

\questionF{3. Qu creencias sobre ti mismo has reconsiderado o~cambiado recientemente?}


\cluster{Libertad y~Cambios}{10-20 min}{qu ha cambiado en~m}

\questionE{1. Qu libertad trae una~nueva comprensin de~ti mismo?}

\questionE{2. Qu te~permites hacer ahora que~antes no te~permitas?}

\questionE{3. A qu ests dispuesto a~renunciar? Qu hay en~tu vida que~ya no te~sirve?}


\cluster{Accin}{10-20 min}{qu hago despus}

\questionE{1. Qu accin ests dispuesto a~tomar hoy basndote en~tu nueva comprensin?}

\questionE{2. Con quin puedes compartir tus nuevos descubrimientos sobre ti mismo? Quin est dispuesto a~acompaarte en~este camino?}


\cluster{Continuacin del~Camino}{15-30 min}{hacia dnde voy ahora}

\questionI{1. Qu pregunta te~hars regularmente a~ti mismo?}

\questionI{2. Cundo volvers a~hacer esta autorreflexin, en~un~mes, en~un~trimestre o~en~un~ao?}


\newpage

\chapter{3 Pilares de~Purificacin}
\programbreak

\cluster{Aceptacin}{5-10 min}{lo~que estoy dispuesto a~aceptar}

\questionF{1. Qu parte de~ti mismo rechazaste en~el~pasado, pero~con el~tiempo lograste aceptar?}

\questionF{2. Qu cambi en~tu vida cuando aceptaste aquello contra lo~que antes luchabas?}

\questionF{3. Qu parte de~ti an espera ser aceptada?}


\cluster{Dejar Ir}{10-20 min}{lo~que ests dispuesto a~dejar ir}

\questionE{1. De qu te~has logrado desprender este ao: creencias, hbitos o~relaciones?}

\questionE{2. Qu te~sucede cuando sueltas algo? Cmo cambia esto tu~estado interno?}

\questionE{3. Qu ests reteniendo en~este momento, sabiendo que~es momento de~soltar?}


\cluster{Transformacin}{15-30 min}{cmo cambio}

\questionI{1. Cundo fue la~ltima vez que~tu crecimiento ocurri no a~travs de~logros, sino al~aceptar y~soltar aquello que~no funcionaba?}

\questionI{2. Qu aspectos nuevos de~ti has descubierto gracias a~este proceso de~purificacin?}

\questionI{3. En qu te~has convertido despus de~este ciclo de~purificacin? Qu ha cambiado dentro de~ti?}


\cluster{Integracin}{15-30 min}{qu cambiar en~la~vida}

\questionI{1. Cmo puede influir este nuevo entendimiento en~tus decisiones, tanto en~el~futuro cercano como en~el~lejano?}

\questionI{2. Qu paso concreto ests dispuesto a~dar hoy mismo, basado en~esta purificacin?}

\questionI{3. Qu nuevos conocimientos o~habilidades podran enriquecer tu~experiencia de~purificacin y~hacerla ms integral?}


\newpage

\chapter{Recurso}
\programbreak

\cluster{Personas como Recurso}{5-10 min}{quin me~apoya}

\questionF{1. Quin ha tenido una~influencia especialmente fuerte en~ti? Qu hizo o~dijo exactamente esta persona?}

\questionF{2. Quin es tu~fuente principal de~apoyo en~este momento?}

\questionF{3. En quin puedes confiar en~los momentos difciles? Por qu precisamente en~estas personas?}

\questionF{4. Quin cree en~ti, incluso cuando t~mismo dudas de~ti mismo?}


\cluster{Recursos Internos}{10-20 min}{lo~que me~da fuerzas}

\questionE{1. De qu te~sientes ms orgulloso/a y~por qu es importante para ti?}

\questionE{2. Cul es tu~principal fortaleza y~cmo la~utilizas?}

\questionE{3. Qu experiencia de~tu pasado te~ayuda ahora y~qu te~ense?}

\questionE{4. Cules son las tres cualidades que~consideras tu~mayor ventaja?}


\cluster{Recursos Materiales}{10-20 min}{lo~que tengo}

\questionE{1. Qu recursos materiales tienes disponibles? (dinero, bienes, educacin, conexiones)}

\questionE{2. Cmo ests utilizando actualmente tus recursos: tiempo, energa y~finanzas? Qu est funcionando bien y~qu te~gustara cambiar?}

\questionE{3. Qu recursos (tiempo, dinero, energa) tienes en~abundancia y~cules te~faltan para lo~que realmente te~importa?}


\cluster{Tiempo y~Energa}{10-20 min}{dnde se~van mis fuerzas}

\questionE{1. En qu inviertes la~mayor parte de~tu tiempo? Te trae beneficios?}

\questionE{2. Qu actividades te~energizan? Con qu frecuencia las realizas?}

\questionE{3. Qu actividades te~consumen ms energa? Puedes cambiar esto?}


\cluster{Desarrollo de~Recursos}{15-30 min}{cmo volverse ms fuerte}

\questionI{1. Qu recurso te~gustara desarrollar primero?}

\questionI{2. Qu necesitas hacer para tener acceso a~nuevos recursos?}

\questionI{3. A quin o~a~qu puedes recurrir para ampliar tus oportunidades de~crecimiento?}


\newpage

\chapter{Lmites de~la~Personalidad}
\programbreak

\cluster{Qu es un~lmite para m}{5-10 min}{qu significa soar}

\questionF{1. En qu situaciones sueles permitir que~las personas traspasen tus lmites personales?}

\questionF{2. Por qu permites esto? Por miedo, por cortesa o~por costumbre?}

\questionF{3. Con quin te~resulta ms fcil establecer lmites y~con quin te~resulta ms difcil hacerlo? Por qu?}


\cluster{Lmites Personales}{10-20 min}{mis «s» y~«no»}

\questionE{1. Qu deseo ya no te~sirve?}

\questionE{2. Cul es el~objetivo del~que es momento de~desprenderte, uno que~ya no te~sirve?}

\questionE{3. Qu ritual o~hbito te~gustara introducir o~cambiar y~por qu?}

\questionE{4. Sientes culpa al~romper tus propias reglas?}


\cluster{Lmites en~las relaciones}{10-20 min}{donde me~abro a~las personas}

\questionE{1. Con cules de~tus seres queridos puedes permitirte ser vulnerable y~con cules sientes la~necesidad de~parecer fuerte?}

\questionE{2. A quin le permites ver tu~debilidad y~vulnerabilidad?}

\questionE{3. Ante quin te~pones una~mscara y~por qu?}

\questionE{4. Qu personas te~exigen estar siempre a~la~altura? Quieres cambiar esto?}


\cluster{Lmites en~el~Trabajo}{10-20 min}{dnde termina el~trabajo y~dnde empiezo yo}

\questionE{1. En qu medida tus responsabilidades laborales coinciden con tus intereses y~valores?}

\questionE{2. Si~pones en~la~balanza los rechazos en~entrevistas de~trabajo frente a~trabajar en~un~lugar donde no te~valoran, qu inclinara la~balanza para ti?}

\questionE{3. Qu te~hace, especficamente a~ti, la~persona indicada para resolver esta tarea? Quin ms podra encargarse de~ella?}

\questionE{4. Cundo fue la~ltima vez que~dijiste no a~un~trabajo adicional? Qu te~impidi hacerlo antes?}


\cluster{Lmites con la~Tecnologa y~el~Tiempo}{10-20 min}{dnde se~va mi~tiempo}

\questionE{1. Cundo fue la~ltima vez que~pasaste un~da entero sin tu~telfono ni~internet?}

\questionE{2. Has establecido lmites para el~uso de~internet en~tu vida? Cules son?}

\questionE{3. Qu de~lo~que hiciste hoy te~acerc a~tus metas principales y~qu, tal vez, no fue tan importante?}


\cluster{Lmite entre trabajo y~vida personal}{10-20 min}{La~frontera psicolgica que~separa las responsabilidades y~actividades profesionales de~la~vida personal, las relaciones y~el~tiempo de~ocio. Esta frontera ayuda a~mantener la~salud mental, previene el~agotamiento y~asegura una~adecuada distribucin de~tiempo y~energa entre las demandas laborales y~el~bienestar personal.}

\questionE{1. Tienes un~horario claro de~cundo termina tu~jornada laboral?}

\questionE{2. Si~maana todo en~tu vida perdiera su~significado, de qu te~arrepentiras ms?}


\cluster{Lmites Fsicos y~Corporales}{10-20 min}{lo~que me~dice mi~cuerpo}

\questionE{1. Qu lmites fsicos personales respetas de~manera constante y~cules sueles transgredir con frecuencia?}

\questionE{2. Has llegado a~amarte lo~suficiente como para renunciar a~lo~que te~destruye?}


\cluster{Accin}{15-30 min}{mis primeros pasos}

\questionI{1. Qu lmite deseas establecer durante el~prximo mes?}

\questionI{2. Cmo sabrs si~este lmite est funcionando? Qu seales te~lo~mostrarn?}

\questionI{3. A quin le pediras apoyo para establecer este lmite?}


\newpage

\chapter{Trabajo con los Miedos}
\programbreak

\cluster{Conociendo el~Miedo}{5-10 min}{Me gusta mi~trabajo?}

\questionF{1. Qu miedo ha vivido en~ti durante tanto tiempo que~se ha vuelto parte de~ti?}

\questionF{2. Con qu frecuencia se~manifiesta este miedo y~cundo se~activa?}

\questionF{3. En qu parte del~cuerpo sientes miedo?}

\questionF{4. Qu posible escenario futuro te~causa ms ansiedad o~preocupacin?}


\cluster{Encuentro con el~miedo}{10-20 min}{qu siento cuando tengo miedo}

\questionE{1. Recuerda un~momento en~que enfrentaste tu~miedo cara a~cara. Qu sentiste?}

\questionE{2. Qu sucede con tu~miedo cuando respiras a~travs de~l? Se vuelve ms fcil?}

\questionE{3. Qu palabras clidas y~de~apoyo podras decirle a~tu miedo?}


\cluster{Anlisis Profundo del~Miedo}{10-20 min}{qu se~esconde detrs del~miedo}

\questionE{1. Si~tu miedo pudiera hablar, qu dira sobre lo~que est protegiendo?}

\questionE{2. Si~tu ansiedad fuera un~maestro sabio, qu te~estara enseando?}

\questionE{3. Qu parte de~ti ya no teme aquello que~te asustaba antes?}


\cluster{Miedos Especficos}{10-20 min}{mis principales miedos}

\questionE{1. Qu miedos financieros no te~dejan dormir tranquilo?}

\questionE{2. Qu es lo~que quieres, pero~tienes miedo de~desear de~verdad?}

\questionE{3. Crees que~otras personas pueden sentir incomodidad o~miedo al~comunicarse contigo? Si~es as, a qu crees que~podra estar relacionado?}


\cluster{Miedo y~Motivacin}{10-20 min}{qu me~hace el~miedo}

\questionE{1. Tu miedo te~motiva a~avanzar o~te paraliza?}

\questionE{2. Cuando evitas ciertas situaciones, qu te~asusta exactamente de~ellas: un~peligro real o~la~posibilidad de~descubrir algo nuevo sobre ti mismo?}

\questionE{3. Qu evitas de~manera tan sutil que~ni siquiera te~das cuenta?}


\cluster{Miedo en~las Relaciones}{10-20 min}{la~intimidad me~asusta}

\questionE{1. Das tu~tiempo y~atencin a~las personas por generosidad o~por miedo al~rechazo?}

\questionE{2. Qu te~asusta ms: mostrar tu~vulnerabilidad o~el~riesgo de~que la~usen en~tu contra?}

\questionE{3. Qu te~asusta ms: quedarte solo o~enfrentarte a~ti mismo?}

\questionE{4. Si~no tuvieras miedo a~la~soledad, qu relaciones acabaras?}


\cluster{Miedos existenciales}{10-20 min}{de~qu tengo realmente miedo}

\questionE{1. Qu temes ms: la~intensidad de~tus emociones o~su total ausencia?}

\questionE{2. Quin llegars a~ser cuando dejes de~tener miedo?}

\questionE{3. En quin te~conviertes cuando eliges el~amor en~vez del~miedo?}


\cluster{Miedos inconscientes}{10-20 min}{de~qu tengo miedo en~secreto}

\questionE{1. Qu verdad sobre ti mismo ya sientes, pero~an no ests preparado para enfrentar?}

\questionE{2. Si~tus relaciones ms cercanas pudieran hablar, qu vivencias internas ocultas revelaran sobre ti?}


\cluster{Accin}{15-30 min}{qu estoy haciendo ahora}

\questionI{1. Qu pequeo paso podras dar hacia aquello que~te preocupa?}

\questionI{2. Qu persona o~recurso podra ayudarte en~este trabajo?}


\newpage

\chapter{Entender las Relaciones}
\programbreak

\cluster{Mapa de~Mis Relaciones}{5-10 min}{mis personas cercanas y~lejanas}

\questionF{1. Cuntas personas te~conocen realmente tal como eres? Quines son?}

\questionF{2. Con quin en~tu vida te~sientes seguro de~ser t~mismo?}

\questionF{3. Quin es para ti un~ejemplo de~una~relacin armoniosa y~saludable?}

\questionF{4. Qu tan satisfecho ests con tus amistades en~este momento? Califcalo en~una~escala del~1 al~10.}


\cluster{xitos y~Recursos en~las Relaciones}{10-20 min}{mis victorias en~el~amor}

\questionE{1. De qu xito en~tus relaciones te~sientes orgulloso?}

\questionE{2. Recuerda una~relacin en~la~que te~sentiste verdaderamente visto y~comprendido. Qu la~haca especial?}

\questionE{3. Piensa en~alguien que~te acepta completamente tal como eres. Qu sientes al~estar con esa persona?}

\questionE{4. Recuerda una~relacin que~te ayud a~ser una~mejor persona: qu tena de~sanador?}

\questionE{5. Cmo has ayudado a~otra persona? Qu te~ha aportado eso a~ti?}


\cluster{Patrones y~Ciclos}{10-20 min}{mis errores recurrentes}

\questionE{1. Qu patrones o~comportamientos destruyeron tus relaciones pasadas?}

\questionE{2. Qu patrones de~comportamiento se~repiten en~tus relaciones?}

\questionE{3. Atraes a~personas que~necesitan ayuda o~eliges a~aquellas a~las que~puedes ayudar?}

\questionE{4. Sueles perdonar genuinamente a~las personas o~finges que~las has perdonado para evitar la~confrontacin?}


\cluster{Yo~en~las relaciones}{10-20 min}{quin soy junto a~otros}

\questionE{1. Hay aspectos de~tu personalidad que~prefieres no mostrar en~las relaciones?}

\questionE{2. Con cul de~las personas cercanas a~ti puedes ser vulnerable y~autntico, y~con cul sientes la~necesidad de~parecer ms fuerte?}

\questionE{3. En qu parte del~cuerpo sientes que~una~relacin no es correcta para ti?}

\questionE{4. Eliges la~intimidad con las personas o~con la~imagen que~ellas han creado de~ti?}

\questionE{5. Qu parte de~ti mismo escondes o~no muestras en~las relaciones con los dems?}


\cluster{Amor y~Expresin}{10-20 min}{mi~forma de~amar}

\questionE{1. Cmo demuestras amor a~las personas cercanas a~ti? Y cmo te~gustara recibirlo?}

\questionE{2. Cmo expresas tu~amor a~las personas que~quieres?}

\questionE{3. Dedicas tu~tiempo y~atencin a~las personas por generosidad o~por miedo a~ser rechazado?}


\cluster{Influencia del~Origen}{10-20 min}{lo~que me~dieron mis padres}

\questionE{1. Cmo influyen tus relaciones con tus padres en~tus relaciones romnticas?}

\questionE{2. Qu caractersticas de~las relaciones de~tu infancia te~gustara conservar en~tu vida adulta y~cules te~gustara cambiar?}

\questionE{3. Con qu frecuencia te~comunicas de~manera significativa con tus familiares cercanos? Te enriquece esto?}


\cluster{Vulnerabilidad e~Intimidad}{10-20 min}{abrirse o~defenderse}

\questionE{1. Qu te~impide abrirte a~las personas cercanas a~ti?}

\questionE{2. Buscas comprensin en~las relaciones o~confirmacin de~lo~que ya piensas sobre ti mismo?}

\questionE{3. Si~no tuvieras miedo a~la~soledad, qu relaciones acabaras?}


\cluster{Transformacin}{10-20 min}{quin soy ahora}

\questionE{1. En quin te~conviertes cuando eliges el~amor en~vez del~miedo?}

\questionE{2. Si~tus relaciones ntimas pudieran hablar, qu revelaran sobre lo~que sientes la~necesidad de~ocultar o~proteger?}


\cluster{Accin}{15-30 min}{qu har a~continuacin}

\questionI{1. Cmo puedes mejorar tu~relacin con alguno de~tus seres queridos? Cul es el~primer paso que~ests dispuesto/a a~dar?}

\questionI{2. Con quin ests dispuesto/a a~compartir algo ms personal sobre ti esta semana?}

\questionI{3. Qu cualidades deseas desarrollar en~ti para tener relaciones ms cercanas y~de~mayor confianza?}


\newpage

\chapter{Agotamiento  Recurso}
\programbreak

\cluster{Seales de~Agotamiento}{5-10 min}{lo~que siento y~noto}

\questionF{1. Qu le pasa a~tu cuerpo cuando ests agotado? Qu seales te~enva tu~organismo?}

\questionF{2. Qu emocin se~esconde detrs de~tu cansancio crnico?}

\questionF{3. En qu momentos sientes ira pero~la~llamas cansancio o~irritacin?}

\questionF{4. Cundo tus objetivos te~nutren de~energa y~cundo empiezan a~agotarte?}


\cluster{Disparadores y~Patrones}{10-20 min}{lo~que me~agota}

\questionE{1. Qu es lo~que ms te~quita energa en~la~vida?}

\questionE{2. Cundo fue la~ltima vez que~sentiste un~descanso verdadero, completo, sin culpa ni~urgencia?}

\questionE{3. Qu responsabilidades asumes por sentido del~deber, aunque no te~inspiren?}


\cluster{Recuperacin --- Mtodos Rpidos}{10-20 min}{primeros auxilios para ti mismo}

\questionE{1. Qu te~ayuda a~recuperarte rpidamente del~estrs y~recargar energas?}

\questionE{2. Qu actividades te~ayudan mejor a~recuperar tu~energa: msica, socializacin, movimiento, naturaleza o~creatividad?}

\questionE{3. Qu te~ayuda a~relajarte rpidamente y~a~recuperar tu~equilibrio interior?}


\cluster{Recuperacin --- Mtodos Profundos}{10-20 min}{lo~que me~da fuerzas}

\questionE{1. Qu tipo de~descanso te~restaura de~manera ms completa: fsico, emocional, creativo o~social?}

\questionE{2. Cmo es tu~fin de~semana ideal? Qu te~da energa y~te ofrece un~verdadero descanso?}

\questionE{3. Qu personas te~ayudan a~recuperar tu~energa? Cmo te~das cuenta de~que, a~su lado, te~sientes ms vivo y~lleno de~vida?}


\cluster{Autosoporte}{10-20 min}{cmo me~abrazo a~m mismo/a}

\questionE{1. Cmo te~apoyas en~los momentos de~cansancio o~cuando sientes que~tus fuerzas decaen?}

\questionE{2. Qu te~dices a~ti mismo cuando sientes agotamiento? Te apoyas o~te criticas?}

\questionE{3. Qu ritual de~renovacin te~gustara incorporar en~tu vida?}


\cluster{Equilibrio y~Lmites}{10-20 min}{qu tomo y~qu doy}

\questionE{1. A qu ests dispuesto a~renunciar para liberar espacio para tu~recuperacin?}

\questionE{2. Qu lmites necesitas establecer para no llegar al~agotamiento?}

\questionE{3. Cmo cambiara tu~vida si~pudieras descansar tanto como trabajas?}


\cluster{Accin}{15-30 min}{mis primeros pasos}

\questionI{1. Qu ritual de~recuperacin puedes incorporar esta semana?}

\questionI{2. A quin le contars sobre tu~agotamiento y~pedirs apoyo?}

\questionI{3. Cmo pasaras un~da completo de~descanso sin ningn tipo de~limitaciones?}


\newpage

\chapter{Sanacin del~Pasado}
\programbreak

\cluster{Reconocimiento de~lo~que Fue}{5-10 min}{Mi~verdad sobre el~pasado}

\questionF{1. Qu evento de~tu pasado an resuena en~ti?}

\questionF{2. Qu leccin aprendiste del~perodo ms difcil de~tu vida?}

\questionF{3. Qu has aprendido de~importante ltimamente?}

\questionF{4. En qu aspectos te~has fortalecido ltimamente? Qu te~ha aportado esto?}


\cluster{Gratitud}{10-20 min}{lo~que me~ha dado el~pasado}

\questionE{1. Recuerda un~momento en~el~que te~invadi la~gratitud. Qu sentiste en~tu cuerpo?}

\questionE{2. Por qu aspectos de~tu experiencia pasada puedes sentirte agradecido?}

\questionE{3. Cmo has ayudado a~otra persona y~qu te~ha aportado eso?}


\cluster{Personas que~ayudaron a~sanar}{10-20 min}{quin estuvo ah para ti}

\questionE{1. Recuerda las relaciones que~te ayudaron a~ser una~mejor persona. Qu las haca sanadoras?}

\questionE{2. Recuerda las relaciones en~las que~te sentiste realmente comprendido y~aceptado tal como eres. Qu haca que~esas conexiones fueran tan especiales para tu~proceso de~sanacin?}

\questionE{3. Recuerda a~alguien que~te acepta completamente. Qu sientes al~estar a~su lado?}

\questionE{4. Con quin puedes estar en~silencio y~sentirte comprendido?}

\questionE{5. Con quin en~tu vida te~sientes seguro de~ser t~mismo?}


\cluster{Perdn}{10-20 min}{cmo liberarse del~dolor}

\questionE{1. A quin no has perdonado an, a~otros o~a~ti mismo?}

\questionE{2. Perdonas ms a~menudo a~las personas o~finges que~las has perdonado para evitar la~confrontacin?}

\questionE{3. Si~tu tristeza pudiera hablar, qu dira acerca de~tu vida?}

\questionE{4. Qu ests dispuesto a~soltar de~tu pasado para liberar espacio para lo~que realmente es importante para ti ahora?}


\cluster{Emociones del~Pasado}{10-20 min}{lo~que el~cuerpo recuerda}

\questionE{1. Recuerda una~emocin de~tu pasado que~te brind paz. Cmo se~senta en~tu cuerpo?}

\questionE{2. Qu emocin de~tu pasado an vive en~tu cuerpo, aunque no encuentres palabras para describirla?}

\questionE{3. Qu color, sonido o~imagen est conectado con tu~herida emocional del~pasado?}


\cluster{Replanteamiento de~la~Identidad Personal}{15-30 min}{Quin soy despus de~todo lo~que he vivido}

\questionI{1. Quin seras si~dejaras de~intentar satisfacer las expectativas de~los dems?}

\questionI{2. Si~nadie te~conociera y~pudieras empezar desde cero, quin elegiras ser?}

\questionI{3. Si~no existieran las expectativas sociales ni~las limitaciones, qu parte de~ti mismo permitiras que~se manifestara con mayor libertad?}

\questionI{4. Si~te permitieran ser absolutamente honesto por un~da, qu diras?}


\cluster{Visin de~Uno Mismo}{10-20 min}{quin soy realmente}

\questionE{1. De qu te~sientes ms orgulloso/a y~qu hace que~este logro sea especialmente valioso para ti?}

\questionE{2. Cules son los logros de~los que~te sientes orgulloso/a? Sean grandes o~pequeos, todos son importantes.}

\questionE{3. Quin te~inspira ms y~por qu?}


\cluster{Salida del~Pasado}{10-20 min}{cmo soltar y~seguir adelante}

\questionE{1. Qu sientes en~este momento? Qu deseas ms ahora mismo?}

\questionE{2. Qu le quieres dar ms espacio en~tu vida este ao?}

\questionE{3. Hacia dnde puedes dirigir tu~energa para que~beneficie tanto a~ti como a~quienes te~rodean?}

\questionE{4. Si~maana todo perdiera su~significado, cul sera tu~mayor arrepentimiento?}


\cluster{Accin y~Futuro}{15-30 min}{mis pasos despus del~dolor}

\questionI{1. Cul es el~paso ms simple que~puedes dar para integrar esta leccin de~tu pasado en~tu vida?}

\questionI{2. Con quin te~gustara compartir tu~camino de~sanacin? Quines son aquellas personas para quienes es importante conocer tu~historia?}


\newpage

\chapter{Cuerpo y~Emociones}
\programbreak

\cluster{Mapa de~Sensaciones Corporales}{5-10 min}{qu me~dice mi~cuerpo}

\questionF{1. Dnde sientes la~alegra en~tu cuerpo? Describe la~sensacin: calor, ligereza, expansin?}

\questionF{2. Recuerda un~momento en~el~que la~gratitud te~llen por completo. Qu sentiste en~tu cuerpo en~ese momento?}

\questionF{3. Recuerda una~emocin que~te traiga tranquilidad. Cmo la~sientes en~tu cuerpo?}

\questionF{4. En qu parte de~tu cuerpo vive la~emocin ms profunda que~no puedes expresar con palabras?}


\cluster{Seales del~Cuerpo}{10-20 min}{qu siente mi~cuerpo}

\questionE{1. Qu emociones experimentas con ms frecuencia a~lo~largo del~da?}

\questionE{2. Hay alguna emocin que~viva en~tu cuerpo constantemente, como un~ruido de~fondo?}

\questionE{3. Qu le sucede a~tu cuerpo cuando las emociones se~vuelven muy intensas? Qu seales te~enva tu~organismo?}

\questionE{4. Qu le sucede a~tu cuerpo cuando sientes cansancio o~agotamiento emocional? Qu seales te~enva tu~organismo?}


\cluster{Emociones y~Mscaras}{10-20 min}{mis verdaderos sentimientos}

\questionE{1. Qu emociones sientes cuando ests a~solas contigo mismo y~no muestras a~los dems?}

\questionE{2. Usas el~humor o~el~sarcasmo para expresar tus sentimientos o~para ocultarlos?}

\questionE{3. Cuando alguien te~pregunta: «Qu ests sintiendo?», tu primer impulso es responder con honestidad o~decir lo~que esperan escuchar?}

\questionE{4. Qu parte de~ti mismo escondes tensando fsicamente los msculos o~bloquendote?}


\cluster{Actitud hacia los sentimientos}{10-20 min}{qu estoy sintiendo ahora mismo}

\questionE{1. Te permites simplemente sentir lo~que hay, o~a~menudo piensas que~deberas sentir algo diferente?}

\questionE{2. Qu palabras tiernas y~bondadosas puedes ofrecerle a~tus emociones en~lugar de~criticarlas?}

\questionE{3. Qu estado de~nimo te~gustara cultivar por la~maana? Cmo se~sentira esto en~tu cuerpo?}


\cluster{Capas Profundas}{10-20 min}{lo~que mi~cuerpo me~dice}

\questionE{1. Si~tu ansiedad fuera una~maestra sabia, qu te~estara enseando?}

\questionE{2. Qu te~dicen los sentimientos que~experimentas regularmente?}

\questionE{3. Qu sentimientos tiernos y~bondadosos puedes experimentar hacia ti mismo en~este momento?}


\cluster{Regulacin a~travs del~cuerpo}{15-30 min}{cmo el~cuerpo me~sana}

\questionI{1. Cmo puedes ayudar a~tu cuerpo a~manejar emociones intensas?}

\questionI{2. Qu movimiento o~caricia te~ayuda a~reencontrar tu~equilibrio?}

\questionI{3. Te das cuenta de~cmo cambia tu~respiracin con diferentes emociones? Qu le sucede a~tu cuerpo cuando te~sientes ansioso o~alegre?}


\newpage

\chapter{Dinero y~Autoestima}
\programbreak

\cluster{Origen de~las Creencias}{5-10 min}{quin me~influy}

\questionF{1. Qu lecciones financieras aprendiste de~tus padres? Tanto las buenas como las no tan tiles; todo es importante para comprender.}

\questionF{2. Qu creencias sobre el~dinero has heredado de~tu familia y~entorno? Cmo influyen en~tu vida actualmente?}

\questionF{3. Cmo se~relacionaban tus padres con el~dinero? Qu aspectos de~su enfoque notas en~ti ahora?}


\cluster{Creencias Actuales}{5-10 min}{mis reglas sobre el~dinero}

\questionF{1. Qu creencias sobre el~dinero tienes ahora mismo?}

\questionF{2. Qu creencias tienes actualmente sobre el~dinero y~cmo estn influyendo en~tu situacin financiera?}

\questionF{3. Cmo te~relacionas con el~dinero: lo ves ms como una~oportunidad o~como una~limitacin?}

\questionF{4. Cmo te~comportas con el~dinero cuando experimentas estrs? Qu podra revelar esto sobre tus creencias?}


\cluster{Relacin con el~Dinero y~la~Autoestima}{10-20 min}{mi~dinero y~yo}

\questionE{1. Qu influye en~tu sentido de~merecer bienestar financiero?}

\questionE{2. Tienes la~creencia de~que no eres lo~suficientemente bueno para tener dinero? De dnde proviene esa creencia?}

\questionE{3. Qu haras en~la~vida si~el~dinero dejase de~ser una~limitacin?}


\cluster{Miedos Financieros}{10-20 min}{qu temo con el~dinero}

\questionE{1. Qu miedos financieros no te~dejan dormir tranquilo por las noches?}

\questionE{2. Qu es lo~que ms te~asusta: el~fracaso financiero o~el~xito que~cambiara todo en~tu vida?}

\questionE{3. Si~el~dinero desapareciera, qu perderas adems del~dinero mismo?}


\cluster{Situacin Actual}{5-10 min}{mi~dinero hoy}

\questionF{1. Cmo describiras tu~situacin financiera actual?}

\questionF{2. En qu ests invirtiendo especficamente tu~dinero ahora mismo: en~tu futuro (educacin, activos, inversiones) o~en~el~presente (comodidad, entretenimiento, compras de~estatus)?}

\questionF{3. Qu parte de~tus ingresos destinas al~desarrollo personal?}


\cluster{Sueos y~Motivacin}{10-20 min}{qu sueo con comprar}

\questionE{1. Con qu sueas al~imaginar la~libertad financiera?}

\questionE{2. Qu haras si~tuvieras suficiente dinero para realizar cualquier plan que~quisieras?}

\questionE{3. Si~tuvieras todos los recursos necesarios (tiempo, dinero, oportunidades), a qu te~gustara dedicarte?}


\cluster{Objetivos y~Estrategia}{10-20 min}{a~dnde quiero llegar}

\questionE{1. Qu meta financiera especfica deseas alcanzar en~un~ao?}

\questionE{2. Qu metas financieras te~gustara alcanzar en~cinco aos?}

\questionE{3. Quin puede ayudarte en~tu desarrollo financiero: un~mentor, un~libro o~un~curso?}

\questionE{4. Cul es el~primer paso concreto que~puedes dar esta semana para acercarte a~tus metas financieras?}


\cluster{xito y~Transformacin}{15-30 min}{hacia dnde me~lleva el~xito}

\questionI{1. A qu te~dedicaras si~supieras que, en~cualquier cosa que~hagas, te~espera el~xito?}

\questionI{2. Cmo cambiar tu~vida si~logras la~libertad financiera?}


\newpage

\chapter{Crisis 27 / 45}
\programbreak

\cluster{Expectativas vs Realidad}{5-10 min}{mi~verdad sobre el~pasado}

\questionF{1. Quin imaginabas que~seras a~esta edad cuando eras joven?}

\questionF{2. Quin eres realmente en~este momento? Coincide con tus expectativas?}

\questionF{3. Qu ocurri de~manera diferente a~como lo~habas planeado?}

\questionF{4. Por qu ests agradecido de~que tu~vida no haya salido como la~planeaste?}


\cluster{Sabidura de~los Errores}{10-20 min}{lo~que me~ha enseado la~vida}

\questionE{1. Qu sabidura hay en~tus errores?}

\questionE{2. Cules son las tres lecciones de~vida ms importantes que~transmitiras a~quienes vienen despus de~ti?}

\questionE{3. Si~pudieras cambiar algo de~tu pasado, qu cambiaras?}

\questionE{4. Qu agradeces que~haya hecho tu~yo~ms joven?}


\cluster{Crisis Actual}{5-10 min}{lo~que no me~satisface}

\questionF{1. Qu en~tu vida actual te~est causando una~crisis o~una~sensacin de~estancamiento?}

\questionF{2. Has perdido el~sentido en~aquello que~antes te~daba energa?}

\questionF{3. Qu cambios en~la~vida ests enfrentando actualmente, ya sea en~relaciones, trabajo, tu~visin de~ti mismo o~planes para el~futuro?}


\cluster{Reevaluacin de~Valores}{10-20 min}{qu se~ha vuelto ms importante}

\questionE{1. Qu valores te~parecan importantes antes pero~han perdido su~significado?}

\questionE{2. Qu valores se~han vuelto ms importantes para ti que~antes?}

\questionE{3. Si~pudieras vivir los prximos diez aos de~manera diferente, qu cambiaras en~tus prioridades y~por qu?}


\cluster{Posibilidad de~Cambio}{10-20 min}{lo~que est en~mi poder}

\questionE{1. Hay aspectos de~tu vida que~quieras cambiar de~manera radical?}

\questionE{2. Qu te~impide cambiar eso que~deseas cambiar?}

\questionE{3. De qu estaras dispuesto a~desprenderte para renovar tu~vida?}

\questionE{4. Si~supieras que~todo va a~salir bien, qu cambios te~atreveras a~hacer en~tu vida?}


\cluster{Nuevo Sentido}{15-30 min}{quin soy ahora}

\questionI{1. Cmo te~ves a~ti mismo/a en~10 o~15 aos? Imagina y~describe un~da tpico de~esa vida futura.}

\questionI{2. En qu aspectos deseas reinventarte?}

\questionI{3. Qu nuevo sentido de~la~vida se~revela ante ti despus de~la~crisis?}


\cluster{Accin y~Transicin}{15-30 min}{mi~camino para salir de~la~crisis}

\questionI{1. Qu paso concreto puedes dar esta semana para comenzar tu~transicin?}

\questionI{2. A quin le contars acerca de~tu crisis y~buscars apoyo?}


\newpage

\chapter{Ansiedad por la~IA y~el~Futuro del~Trabajo}
\programbreak

\cluster{Mi~Valor}{5-10 min}{lo~que s hacer}

\questionF{1. Qu habilidades profesionales te~hacen valioso e~indispensable en~tu trabajo actual?}

\questionF{2. Qu aspectos de~tu trabajo no pueden ser reemplazados por una~mquina o~la~IA?}

\questionF{3. Consideras que~la~inteligencia emocional es tu~principal activo profesional? Por qu?}

\questionF{4. Qu haces, ms all de~tu descripcin del~puesto, que~valoran tus colegas?}


\cluster{Identidad Profesional}{10-20 min}{en~qu trabajo y~de~lo~que me~siento orgulloso}

\questionE{1. Quin eres profesionalmente ms all de~tus habilidades tcnicas? Qu te~define como profesional?}

\questionE{2. Qu te~distingue de~otras personas en~tu rea profesional?}

\questionE{3. Qu habilidades o~cualidades tuyas podran volverse especialmente valiosas en~un~mundo donde la~tecnologa est transformando el~entorno profesional?}


\cluster{Signos de~Ansiedad}{10-20 min}{Me gusta mi~trabajo?}

\questionE{1. Qu te~preocupa ms: la~posibilidad de~perder tu~trabajo o~volverte obsoleto profesionalmente?}

\questionE{2. Con qu frecuencia experimentas ansiedad profesional respecto a~la~inteligencia artificial y~las tecnologas?}

\questionE{3. Qu situaciones en~el~trabajo intensifican tu~miedo a~no estar a~la~altura?}

\questionE{4. Cules mtodos sueles utilizar para lidiar con el~miedo a~la~incompetencia profesional?}


\cluster{Reentrenamiento y~Desarrollo}{10-20 min}{cmo aprendo cosas nuevas}

\questionE{1. Cuntas horas a~la~semana dedicas a~la~recapacitacin y~al desarrollo profesional?}

\questionE{2. Qu nuevas habilidades o~cualidades quieres desarrollar el~prximo ao para sentirte ms seguro de~ti mismo en~un~mundo cambiante?}

\questionE{3. Tienes un~plan de~desarrollo profesional para el~caso de~que tu~rol actual sea automatizado?}

\questionE{4. Qu persona, mentor o~comunidad podra ayudarte a~desarrollarte en~el~trabajo con nuevas tecnologas?}


\cluster{Repensar el~Trabajo}{15-30 min}{cmo veo el~trabajo}

\questionI{1. Si~la~IA pudiera hacer tu~trabajo, qu te~gustara hacer en~su lugar? Qu valor nico podras aportar al~mundo?}

\questionI{2. Si~tuvieras que~empezar tu~carrera desde cero en~este momento, qu estudiaras?}

\questionI{3. Qu cualidades profesionales o~experiencias prefieres mantener en~privado? Qu queda fuera del~enfoque pblico de~tu imagen profesional?}


\cluster{Mercado Laboral y~Oportunidades}{10-20 min}{nuevas oportunidades para m}

\questionE{1. Qu nuevos nichos profesionales estn surgiendo gracias a~la~IA y~la~automatizacin?}

\questionE{2. Cmo puedes usar la~IA como aliada para desarrollar tu~dominio en~la~profesin?}

\questionE{3. Quines de~tus colegas o~personas en~tu profesin se~han adaptado exitosamente a~las nuevas tecnologas? Qu puedes aprender de~ellos?}


\cluster{Accin y~Estrategia}{15-30 min}{mi~plan de~accin}

\questionI{1. Cul es el~primer paso concreto que~dars esta semana para prepararte mejor ante los cambios en~el~mundo laboral?}

\questionI{2. Qu nueva habilidad o~conocimiento quieres adquirir en~los prximos tres meses?}

\questionI{3. Cmo hars seguimiento de~tu progreso en~la~adaptacin a~los cambios profesionales?}


\newpage

\chapter{Info-obesidad}
\programbreak

\cluster{Conciencia del~Consumo}{5-10 min}{a~dnde se~va mi~tiempo}

\questionF{1. Cunto tiempo al~da pasas en~plataformas digitales como redes sociales, noticias y~aplicaciones de~mensajera?}

\questionF{2. Cules son las tres primeras aplicaciones que~abres por la~maana?}

\questionF{3. Cundo fue la~ltima vez que~pasaste un~da sin notificaciones, apagndolas completamente o~activando el~modo No molestar?}

\questionF{4. Abres las aplicaciones con un~propsito consciente o~simplemente por costumbre?}


\cluster{Sentimientos y~Seales}{10-20 min}{cmo me~siento}

\questionE{1. Qu le sucede a~tu cuerpo cuando la~pantalla se~llena de~notificaciones? Qu seales te~enva tu~organismo?}

\questionE{2. Qu emociones surgen despus de~una~hora en~el~flujo de~informacin: tranquilidad, confusin o~sobrecarga?}

\questionE{3. Cuando sientes la~necesidad de~revisar algo en~tu telfono o~computadora, qu es lo~que realmente necesitas en~ese momento?}


\cluster{Miedos y~Motivos}{10-20 min}{de~qu tengo miedo y~qu quiero}

\questionE{1. De qu tienes miedo de~perderte en~el~flujo de~informacin?}

\questionE{2. Si~dejaras de~revisar las noticias durante una~semana, qu perderas y~qu ganaras?}

\questionE{3. Cmo te~ayuda el~consumo digital a~evitar el~aburrimiento, la~soledad o~la~autorreflexin?}


\cluster{Lmite y~Saturacin}{10-20 min}{cuando hay suficiente informacin}

\questionE{1. Dnde estableces el~lmite entre mantenerte bien informado y~experimentar una~sobrecarga de~informacin?}

\questionE{2. Cunta informacin realmente necesitas en~un~da? Cmo te~das cuenta de~que es suficiente?}

\questionE{3. Qu contenido es realmente til para ti y~cul consumes en~piloto automtico?}


\cluster{Del~Consumo a~la~Creacin}{10-20 min}{cmo convertirse en~un~creador}

\questionE{1. Qu pasara si~dejaras de~consumir y~comenzaras a~crear?}

\questionE{2. Qu contenido o~idea te~gustara crear en~vez de~consumir sin parar?}

\questionE{3. Con qu frecuencia escribes, dibujas o~creas algo propio en~comparacin con el~tiempo que~dedicas a~consumir contenido?}


\cluster{Identidad sin dgitos}{10-20 min}{mi~verdadero rostro}

\questionE{1. Cmo te~ves a~ti mismo sin tu~huella digital, sin valoraciones ni~«likes»?}

\questionE{2. Qu partes de~tu personalidad muestras en~lnea y~cules mantienes ocultas?}

\questionE{3. Si~maana desaparecieran todas tus cuentas digitales, qu cambiara respecto a~quin eres?}


\cluster{Independencia}{10-20 min}{quin soy sin internet}

\questionE{1. Puedes pasar un~da sin tu~smartphone? Cmo te~sientes en~ese caso?}

\questionE{2. Cundo fue la~ltima vez que~reflexionaste profundamente sobre algo confiando nicamente en~tus propios pensamientos?}

\questionE{3. Qu decisin has tomado en~el~ltimo mes basndote nicamente en~tu propia experiencia, sin consultar Google ni~internet?}


\cluster{Accin y~Equilibrio}{15-30 min}{mis primeros pasos}

\questionI{1. Qu lmite digital vas a~establecer para ti mismo/a esta semana?}

\questionI{2. Qu aplicacin o~hbito estaras dispuesto a~eliminar o~limitar?}

\questionI{3. Qu actividad fuera de~lnea quieres retomar en~tu vida?}

\questionI{4. Cmo hars seguimiento de~tu progreso para crear un~equilibrio digital?}


\newpage

\chapter{Indefensin Aprendida 2.0}
\programbreak

\cluster{Donde Pierdo Mi~Poder}{5-10 min}{qu drena mi~energa}

\questionF{1. Qu es lo~que ms suele paralizarte en~los momentos en~que necesitas actuar?}

\questionF{2. Dnde termina tu~verdadero crculo de~influencia? Qu queda fuera de~tu control?}

\questionF{3. Qu situaciones en~la~vida te~provocan una~sensacin de~impotencia o~desamparo?}

\questionF{4. Qu le sucede a~tu cuerpo cuando te~sientes impotente? Qu seales te~enva?}


\cluster{Trampa informacional}{10-20 min}{cuando dejo de~intentar}

\questionE{1. Cules son las fuentes de~noticias e~informacin que~estn moldeando tu~visin del~mundo?}

\questionE{2. Cunto tiempo al~da dedicas a~consumir informacin que~te paraliza?}

\questionE{3. De quin adoptas puntos de~vista y~creencias como propios, sin verificar si~realmente corresponden a~tu propia experiencia?}


\cluster{Miedos y~Eleccin}{10-20 min}{lo~que me~detiene}

\questionE{1. Qu te~da ms miedo: el~fracaso o~la~inaccin?}

\questionE{2. Qu te~dice la~voz interior que~no puedes hacer algo? De dnde crees que~surgieron estas creencias?}

\questionE{3. De qu esperanza ya te~has dado por vencido? Quieres recuperarla?}


\cluster{Ilusin de~Control}{10-20 min}{lo~que quiero tener en~mis manos}

\questionE{1. Qu pasara si~dejaras de~intentar controlar lo~que est fuera de~tu alcance?}

\questionE{2. En qu reas de~tu vida sientes que~te esfuerzas mucho por cambiar algo, pero~los resultados siguen siendo los mismos?}

\questionE{3. Si~te enfocaras nicamente en~lo~que realmente est en~tu poder, cmo cambiara esto tu~actitud hacia la~vida?}


\cluster{Partes de~la~Personalidad y~Resistencia}{10-20 min}{quin dentro de~m tiene miedo}

\questionE{1. Qu parte de~ti tiene miedo de~actuar? Qu est protegiendo?}

\questionE{2. Qu obtienes al~mantenerte indefenso? Qu te~aporta la~inaccin?}

\questionE{3. Si~una~parte de~ti tuviese miedo mientras otra estuviese lista para actuar, qu dira esa parte preparada?}


\cluster{Competencia e~Historial de~xitos}{10-20 min}{en~qu soy bueno/a}

\questionE{1. Cundo fue la~ltima vez que~te sentiste competente y~capaz de~cambiar algo?}

\questionE{2. Qu pequeas acciones o~hbitos te~han ayudado a~lograr resultados?}

\questionE{3. En qu rea de~tu vida te~sientes ms influyente y~capaz?}


\cluster{De~la~Indefensin a~la~Agencia}{10-20 min}{hacia dnde me~lleva el~xito}

\questionE{1. Quin eres cuando no te~paraliza la~impotencia?}

\questionE{2. Qu decisiones puedes tomar hoy para tener un~mayor control sobre tu~vida?}

\questionE{3. Si~tomaras completa responsabilidad por aquellas partes de~tu vida que~estn en~tus manos, cmo podra esto cambiar tus decisiones y~acciones?}


\cluster{Accin}{15-30 min}{primeros pasos hacia el~cambio}

\questionI{1. Qu paso concreto puedes dar hoy para sentirte ms seguro/a de~tus capacidades?}

\questionI{2. De qu preocupacin habitual ests dispuesto/a a~liberarte esta semana?}

\questionI{3. Qu nuevo hbito introducirs para fortalecer tu~sentido de~competencia?}


\newpage

\chapter{Dependencia Parasocial}
\programbreak

\cluster{Cartografa de~Mi Mundo}{5-10 min}{mis relaciones reales}

\questionF{1. Cuntas personas en~tu vida te~conocen realmente tal como eres, sin mscaras ni~filtros?}

\questionF{2. Cuntas horas a~la~semana pasas con personas en~persona? Te parece suficiente?}

\questionF{3. A cules de~las personas que~conoces solo virtualmente te~gustara conocer en~la~vida real?}

\questionF{4. Qu tipo de~contenido te~ayuda a~sentirte menos solo?}


\cluster{Fuentes de~Cercana}{10-20 min}{quin me~apoya}

\questionE{1. De quin, entre las personas reales de~tu vida, sientes que~recibes el~mayor apoyo emocional?}

\questionE{2. De qu influencer, personaje o~IA sientes que~recibes apoyo y~qu tienen especficamente que~te hace sentir as?}

\questionE{3. Qu necesidades de~cercana y~comprensin satisface en~ti la~interaccin con compaeros virtuales?}

\questionE{4. Qu necesidad real satisface el~contenido virtual para ti?}


\cluster{Escape y~Evitacin}{10-20 min}{lo~que evito en~la~vida}

\questionE{1. De qu ests huyendo cuando caes en~el~desplazamiento infinito?}

\questionE{2. Qu reemplaza para ti la~virtualidad con respecto al~contacto humano real y~la~intimidad?}

\questionE{3. Qu personas reales o~situaciones ests evitando a~travs de~la~pantalla?}

\questionE{4. Si~desconectaras todas tus relaciones virtuales por una~semana, qu extraaras?}


\cluster{La~Paradoja de~la~Proximidad}{10-20 min}{cuando la~pantalla est ms cerca que~las personas}

\questionE{1. Por qu las relaciones virtuales te~resultan ms fciles que~las reales?}

\questionE{2. Cules son las ventajas de~las relaciones virtuales para ti?}

\questionE{3. Cules son los riesgos y~dificultades de~la~intimidad virtual?}


\cluster{Yo~Autntico}{10-20 min}{mi~verdadero rostro}

\questionE{1. Cundo fue la~ltima vez que~sentiste una~cercana genuina, sin pantallas ni~filtros?}

\questionE{2. Quin eres cuando nadie ve tus publicaciones, reacciones y~estado en~lnea?}

\questionE{3. Qu aspectos de~ti mismo muestras en~lnea pero~ocultas en~las interacciones cara a~cara?}

\questionE{4. Si~maana desaparecieran todas tus relaciones virtuales, cambiara algo de~quin realmente eres?}


\cluster{Recuperacin de~la~Autenticidad}{10-20 min}{regresando a~mi verdadero yo}

\questionE{1. Qu necesitas cambiar en~tu vida para tener relaciones ms autnticas?}

\questionE{2. Qu necesidad real podras empezar a~satisfacer mediante relaciones humanas genuinas?}

\questionE{3. A cul de~las personas cercanas te~gustara contarle aquello que~normalmente solo compartes con interlocutores virtuales?}

\questionE{4. Qu paso podras dar hoy para recuperar una~verdadera cercana con alguien importante para ti?}


\cluster{Lmites y~Equilibrio}{15-30 min}{cmo no perderse en~lnea}

\questionI{1. Qu relaciones virtuales o~contenidos puedes conservar porque realmente te~ayudan?}

\questionI{2. Cmo puedes disfrutar de~Internet sin perder tus conexiones autnticas?}

\questionI{3. Qu lmites quieres establecer entre el~mundo en~lnea y~fuera de~lnea?}


\newpage

\chapter{Vida Hbrida --- Trabajo Remoto}
\programbreak

\cluster{Transformacin del~Espacio}{5-10 min}{mi~mundo antes y~despus}

\questionF{1. Cmo ha cambiado tu~relacin con el~hogar tras adoptar el~modo hbrido?}

\questionF{2. Qu ha llegado a~tu vida gracias al~trabajo remoto?}

\questionF{3. Qu perdiste cuando tu~oficina se~convirti en~tu dormitorio?}

\questionF{4. Cmo ha cambiado tu~espacio personal y~tu energa tras la~digitalizacin total?}


\cluster{Fatiga y~Desencadenantes}{10-20 min}{qu roba mis fuerzas}

\questionE{1. Cuntas reuniones virtuales a~la~semana comienzan a~agotar tus recursos internos?}

\questionE{2. Qu le sucede a~tu cuerpo en~momentos de~cansancio? Qu seales te~est enviando?}

\questionE{3. Qu hbitos digitales se~han convertido en~cadenas invisibles para ti?}

\questionE{4. Cmo te~sealan tu~cuerpo y~tu mente que~ests agotado?}


\cluster{Sacrificios y~Prioridades}{10-20 min}{de~qu renuncio}

\questionE{1. De qu te~ests privando al~pasar cada vez ms tiempo en~el~mundo digital?}

\questionE{2. Qu cosas importantes en~tu vida ests relegando por el~trabajo?}

\questionE{3. Cunta energa consumes al~mantener tu~mscara profesional en~diferentes situaciones?}


\cluster{Lmites}{10-20 min}{cuando hay suficiente informacin}

\questionE{1. Dnde estn tus lmites entre el~trabajo y~la~vida personal en~este momento?}

\questionE{2. Qu pasara si~desactivas las notificaciones despus de~las 18:00?}

\questionE{3. Qu ritual o~rutina marca tu~transicin del~trabajo a~la~vida personal?}

\questionE{4. Cmo se~sienten las personas cercanas a~ti con el~hecho de~que siempre ests conectado/a?}


\cluster{Separacin de~Roles}{10-20 min}{qu es mo y~qu es trabajo}

\questionE{1. Dnde termina tu~rol profesional y~comienza tu~identidad personal?}

\questionE{2. Cmo haces la~transicin de~«trabajador» a~«persona» en~casa?}

\questionE{3. Qu partes de~ti no muestras en~el~trabajo?}


\cluster{Identidad y~Huella Digital}{10-20 min}{mi~imagen online y~offline}

\questionE{1. Quin eres cuando apagas todas las notificaciones y~guardas el~telfono lejos?}

\questionE{2. Qu versin de~ti creas en~el~mundo digital y~qu queda tras bambalinas?}

\questionE{3. Si~maana perdieras el~acceso a~internet por una~semana, quin descubriras que~eres?}


\cluster{Oportunidades de~Hibridez}{10-20 min}{mis ventajas hoy}

\questionE{1. Qu ventajas de~la~vida hbrida valoras ya?}

\questionE{2. Cmo puedes aprovechar mejor las oportunidades de~un~estilo de~vida hbrido para tu~desarrollo personal?}

\questionE{3. Qu libertad te~ha brindado el~trabajo remoto?}


\cluster{Accin e~Integracin}{15-30 min}{cmo lo~reunir todo}

\questionI{1. Qu lmite establecers esta semana entre el~trabajo y~la~vida personal?}

\questionI{2. Qu ritual de~desconexin del~trabajo vas a~incorporar en~tu da?}

\questionI{3. Cmo sabrs si~el~equilibrio se~ha restablecido?}


\newpage

\chapter{Autenticidad vs Algoritmos}
\programbreak

\cluster{Cartografa de~Mscaras}{5-10 min}{mis diferentes rostros}

\questionF{1. Qu versiones diferentes de~ti mismo creas en~distintos contextos: redes sociales, trabajo, familia, grupos de~amigos?}

\questionF{2. Qu plataformas usas? Eres la~misma persona en~todas ellas?}

\questionF{3. Cuntas veces al~da editas, filtras u~ocultas tu~imagen?}

\questionF{4. Qu cambiara en~tu vida si~ya no te~esforzaras por adaptar tu~imagen a~las expectativas de~los dems?}


\cluster{Contenido y~Autenticidad}{10-20 min}{mis publicaciones y~yo}

\questionE{1. Qu contenido creas por aburrimiento, presin social o~por el~deseo de~agradar a~los dems?}

\questionE{2. Qu tipo de~contenido creas sinceramente, desde el~corazn?}

\questionE{3. Cmo se~diferencia tu~personalidad en~internet de~quin eres en~la~vida real?}


\cluster{Riesgos de~Autenticidad}{10-20 min}{lo~que temo en~m mismo/a}

\questionE{1. Qu partes de~tu personalidad se~vuelven invisibles en~el~mbito digital?}

\questionE{2. Qu parte de~ti mismo ocultas por miedo al~juicio ajeno?}

\questionE{3. Qu consideras arriesgado al~mostrar tus verdaderas emociones en~lnea?}

\questionE{4. Qu personas o~situaciones te~llevan a~usar una~mscara?}


\cluster{La~Frontera Entre Presentacin y~Censura}{10-20 min}{cuando hay suficiente informacin}

\questionE{1. Dnde trazas el~lmite entre una~autopresentacin saludable y~la~autocensura?}

\questionE{2. Qu aspectos de~tu vida prefieres mantener solo para ti y~qu influye en~esta decisin?}

\questionE{3. Tienes personas o~comunidades con las que~puedas ser completamente t~mismo/a?}


\cluster{Eleccin entre aceptacin y~autenticidad}{10-20 min}{ser uno mismo o~gustar a~otros}

\questionE{1. Cuando tienes que~elegir entre autenticidad y~aceptacin, cmo decides?}

\questionE{2. Qu contenido crearas si~no existieran los «me gusta» ni~las valoraciones?}

\questionE{3. Si~fueras completamente annimo en~lnea, quin seras?}


\cluster{Autenticidad en~la~Era de~los Algoritmos}{10-20 min}{mi~verdadero rostro}

\questionE{1. Dnde est tu~autenticidad en~este momento: en~lnea o~fuera de~lnea?}

\questionE{2. Qu partes de~ti se~sienten autnticas y~no necesitan ningn filtro?}

\questionE{3. Si~tu versin en~lnea y~tu versin en~la~vida real se~encontraran, qu se~diran la~una a~la~otra?}


\cluster{Recuperacin de~la~Autenticidad}{10-20 min}{cmo encontrar tu~verdadero yo}

\questionE{1. Qu faceta de~ti mismo ests dispuesto a~mostrar ms abiertamente?}

\questionE{2. Con quin te~gustara compartir tu~versin ms autntica? Quines son esas personas?}

\questionE{3. Cul sera ese nico contenido autntico que~crearas si~no tuvieras miedo?}


\cluster{Accin y~Equilibrio}{15-30 min}{cmo ser t~mismo en~lnea}

\questionI{1. Qu verdad sobre ti mismo ests dispuesto a~aceptar en~este momento?}

\questionI{2. Cmo medirs tu~autenticidad, por los likes o~por tu~propia sensacin interior?}

\questionI{3. Qu ritual de~desintoxicacin digital podras introducir en~tu vida?}


\newpage

\chapter{Eco-culpa y~Ansiedad Climtica}
\programbreak

\cluster{Mi~Consumo}{5-10 min}{mis compras y~hbitos}

\questionF{1. Qu productos y~servicios consumes a~diario? Te has planteado alguna vez cul es su~origen?}

\questionF{2. Qu hbitos de~consumo heredaste de~tus padres?}

\questionF{3. Qu hbitos ecolgicos practicas actualmente? Qu te~gustara agregar o~cambiar en~tu consumo?}

\questionF{4. Cul es el~costo ecolgico de~tu da tpico?}


\cluster{Culpa y~Responsabilidad}{10-20 min}{Me gusta mi~trabajo?}

\questionE{1. Cunta energa desperdicias sintiendo culpa ecolgica en~lugar de~pasar a~la~accin?}

\questionE{2. Qu te~asusta ms: la~incomodidad de~cambiar tu~estilo de~vida o~las consecuencias de~la~crisis climtica?}

\questionE{3. Dnde termina tu~responsabilidad personal y~comienza la~responsabilidad sistmica?}

\questionE{4. Con qu frecuencia te~das cuenta de~que ests paralizado por la~culpa en~vez de~actuar?}


\cluster{Emociones y~Negacin}{10-20 min}{lo~que me~oculto a~m mismo/a}

\questionE{1. Qu dolor acerca de~la~naturaleza y~su sufrimiento reprimes dentro de~ti?}

\questionE{2. Cundo fue la~ltima vez que~te permitiste sentir ansiedad ecolgica sin juzgarte?}

\questionE{3. Qu racionalizas o~qu niegas cuando se~trata de~temas ecolgicos?}


\cluster{Eleccin Personal}{10-20 min}{lo~que est en~mis manos}

\questionE{1. Qu decisin personal puedes tomar que~tenga un~impacto real?}

\questionE{2. Qu pequeas acciones ests llevando a~cabo? Tienen valor?}

\questionE{3. Si~supieras con certeza que~tus acciones ayudarn al~planeta, qu cambiaras en~tu vida?}


\cluster{Identidad y~Sistema}{10-20 min}{mi~lugar en~el~mundo}

\questionE{1. Quin eres ms all de~tus hbitos de~consumo?}

\questionE{2. Cmo influye el~sistema (corporaciones, estado, economa) en~tus decisiones?}

\questionE{3. Se puede ser ecolgico dentro de~un~sistema no ecolgico? Cmo?}


\cluster{De~la~Culpa a~la~Accin}{10-20 min}{qu hago por el~planeta}

\questionE{1. De qu culpa ecolgica ests dispuesto a~liberarte?}

\questionE{2. Qu pequea accin podras incorporar a~tu vida desde hoy para sentirte ms ntegro y~completo en~cuanto a~temas ecolgicos?}

\questionE{3. Quin o~qu puede ayudarte con tus decisiones ecolgicas?}

\questionE{4. Cmo vas a~hacer seguimiento de~tu progreso, por materiales o~por tu~sensacin interior?}


\cluster{Activismo y~Coherencia}{15-30 min}{lo~que hago por el~planeta}

\questionI{1. Hay alguna causa ambiental que~te inspire y~por qu?}

\questionI{2. Cmo deseas participar: a~travs de~acciones personales, informando a~otros o~mediante activismo?}

\questionI{3. Qu papel quieres desempear en~la~creacin de~un~mundo ms saludable?}

\questionI{4. De qu logros te~sientes orgulloso/a? Qu aspectos de~tu vida te~llenan de~satisfaccin?}

\questionI{5. De qu logros te~sientes ms orgulloso?}

\questionI{6. Cmo puedes disfrutar an ms del~proceso de~alcanzar tus objetivos ambientales?}

\questionI{7. Qu hay de~valioso en~ti, independientemente de~tus acciones y~logros personales?}


\newpage

\chapter{Sndrome del~Impostor en~la~Era de~LinkedIn}
\programbreak

\cluster{Sntomas y~desencadenantes}{5-10 min}{cuando te~invade la~duda}

\questionF{1. En qu momentos sientes que~no eres lo~suficientemente competente para desempear tu~rol?}

\questionF{2. Qu le sucede a~tu autoestima cuando ves las publicaciones de~xito de~tus colegas en~las redes sociales?}

\questionF{3. Temes que~un~da alguien te~toque el~hombro y~te diga que~no perteneces aqu y~que te~vayas?}


\cluster{Atribucin de~Mritos}{10-20 min}{mis logros o~suerte}

\questionE{1. Con qu frecuencia atribuyes tus xitos a~simplemente haber tenido suerte, a~que fue casualidad o~a~la~ayuda del~equipo?}

\questionE{2. Anota tres hechos indiscutibles sobre tu~carrera profesional (cifras, proyectos completados).}

\questionE{3. Cules son los logros de~los que~te sientes genuinamente orgulloso/a, aunque nadie le haya dado «me gusta» a~tu publicacin?}


\cluster{Ruptura con la~Realidad}{10-20 min}{mi~mscara en~las redes sociales}

\questionE{1. Si~comparas tu~perfil pblico (currculum) con tu~percepcin interna, qu tan grande es la~brecha?}

\questionE{2. Qu inseguridad profesional prefieres no compartir con tus colegas?}

\questionE{3. Qu es lo~peor que~podra pasar si~dijeras que~no sabes la~respuesta a~esa pregunta?}


\cluster{Valor e~Identidad}{10-20 min}{lo~que es importante para m}

\questionE{1. Cmo piensas sobre ti mismo ms frecuentemente: a~travs de~tus logros (soy quien logr esto) o~a~travs de~tus cualidades (soy de~esta manera)?}

\questionE{2. Quin eres sin tus cargos, proyectos y~estatus profesional?}

\questionE{3. Cmo puedes encontrar satisfaccin en~el~proceso mismo de~trabajar hacia tus metas y~no solo en~el~momento de~alcanzarlas?}


\cluster{Salir de~la~Carrera}{15-30 min}{cmo vivir sin comparaciones}

\questionI{1. La opinin o~aprobacin de~quin te~parece todava necesaria para sentirte como un~profesional autntico?}

\questionI{2. Qu consejo le daras a~un~amigo si~se sintiera tan inseguro como t~te sientes ahora mismo?}

\questionI{3. Qu ests dispuesto a~hacer hoy, sabiendo que~eres lo~suficientemente bueno para hacerlo?}

\questionI{4. En qu rea de~tu vida inviertes la~mayor parte de~tu tiempo y~energa?}

\questionI{5. Por qu creo que~esto va a~funcionar? En qu se~basa mi~confianza en~este camino? Quizs haya una~manera ms efectiva; cmo puedo averiguarlo?}

\questionI{6. Si~pudieras elegir sin limitaciones, dnde te~gustara trabajar?}

\questionI{7. Qu te~resulta ms doloroso en~este momento: recibir rechazos en~las entrevistas de~trabajo o~seguir trabajando donde no te~valoran como mereces?}

\questionI{8. Te criticas por descansar cuando tienes mucho trabajo y~muchas responsabilidades?}

\questionI{9. Cunto tiempo te~duraran tus ahorros si~decidieras hacer una~pausa en~el~trabajo?}

\questionI{10. Cmo te~ayuda tu~trabajo actual a~lograr lo~que realmente es importante para ti?}

\questionI{11. Qu aspectos de~tu trabajo actual no se~alinean con las metas que~realmente consideras importantes?}

\questionI{12. Qu tan satisfecho te~sientes con tu~trabajo actual? Calficalo en~una~escala del~1 al~10.}

\questionI{13. Cul sera tu~trabajo ideal?}


\newpage

\chapter{Ansiedad Dominical}
\programbreak

\cluster{Conexin a~tierra}{5-10 min}{lo~que siente mi~cuerpo}

\questionF{1. Cmo describiras tu~estado actual en~tres palabras? Por ejemplo, cansancio, inquietud, expectativa.}

\questionF{2. Cundo el~descanso deja de~ser descanso? Has notado alguna vez ese momento en~que la~relajacin se~convierte en~una~expectativa interna antes del~lunes?}

\questionF{3. Cmo te~sientes respecto a~cmo pasaste este fin de~semana?}

\questionF{4. Si~maana no fuera lunes, cmo pasaras la~noche?}


\cluster{Anlisis de~Monstruos}{10-20 min}{mis principales miedos}

\questionE{1. Qu evento o~tarea especfica de~la~prxima semana te~causa mayor tensin?}

\questionE{2. Esta tensin se~debe al~miedo a~no poder con todo, al~aburrimiento o~a~tener que~lidiar con personas desagradables?}

\questionE{3. En~una~escala del~1 al~10, qu tan satisfecho ests con tu~trabajo en~general?}

\questionE{4. Existe una~amenaza real en~lo~que te~preocupa, o~es solo un~hbito de~preocuparte?}


\cluster{Lmites y~Recursos}{10-20 min}{qu me~ayudar}

\questionE{1. Qu es lo~ms amable y~carioso que~puedes hacer por ti mismo en~este momento?}

\questionE{2. Qu pequeo ritual maana por la~maana te~ayudar a~comenzar el~da con ms calma?}

\questionE{3. Quin o~qu puede ser tu~apoyo en~esta semana?}


\cluster{Verificacin Global}{15-30 min}{cuando la~ansiedad no suelta}

\questionI{1. Si~colocas en~la~balanza el~miedo al~cambio y~el~dolor de~trabajar donde no te~valoran, qu se~siente ms pesado en~este momento?}

\questionI{2. El~trabajo de~tus sueos: hay lugar para la~ansiedad dominical?}

\questionI{3. Qu paso dars esta semana para acercar un~poco tu~realidad a~ese sueo?}

\questionI{4. Si~tu ansiedad dominical pudiera hablar, qu tratara de~decirte sobre tu~vida?}

\questionI{5. Seguirs sintiendo esto cada domingo dentro de~5 aos?}

\questionI{6. Qu necesitas cambiar en~tu vida para que~el~domingo vuelva a~ser simplemente domingo?}


\newpage

\chapter{Culpa Parental por el~Tiempo de~Pantalla}
\programbreak

\cluster{Diagnstico Honesto}{5-10 min}{lo~que realmente est pasando}

\questionF{1. Cuntas horas al~da pasa tu~hijo frente a~una~pantalla y~cuntas de~esas horas son para que~t puedas descansar?}

\questionF{2. Conoces la~trama del~juego o~de~los dibujos animados favoritos de~tu hijo/a?}

\questionF{3. Cunto tiempo pasas en~tu telfono cuando tu~hijo/a est en~la~misma habitacin que~t?}


\cluster{Anatoma de~la~Culpa}{10-20 min}{qu me~est pasando}

\questionE{1. Qu sientes en~ese momento cuando le das la~tablet a~tu hijo: alivio, irritacin o~vergenza?}

\questionE{2. Qu miedo te~invade ms profundamente cuando piensas en~el~tiempo de~pantalla de~tu hijo?}

\questionE{3. Crees que~un~buen padre debe entretener a~su hijo 24/7 sin ayuda de~la~tecnologa?}


\cluster{Significados Ocultos}{15-30 min}{lo~que se~esconde detrs de~la~culpa}

\questionI{1. Cuando prohbes los dispositivos o~te enojas por ellos, qu ansiedad ests tratando de~calmar?}

\questionI{2. Si~recuerdas tu~infancia, qu te~falt que~ahora intentas compensar (o prohibir) a~tu hijo?}

\questionI{3. Qu hay realmente detrs de~tus discusiones con tu~hijo sobre el~telfono? Podra ser esta tu~manera de~conectar con l~cuando no encuentras otros motivos para conversar?}


\cluster{Puente Digital}{15-30 min}{qu hacer a~continuacin}

\questionI{1. Y si~la~pantalla no es una~pared entre ustedes sino una~puerta? Cmo puedes atravesar esa puerta junto con tu~hijo?}

\questionI{2. En quin te~convertirs para tu~hijo si~dejas de~ser censor y~te conviertes en~gua hacia el~mundo digital?}

\questionI{3. Qu ritual familiar sin dispositivos podras introducir que~les traiga verdadera alegra a~ambos, en~lugar de~sentirse como una~obligacin saludable?}

\questionI{4. Cunto tiempo al~da dedicas a~monitorear los grficos de~criptomonedas?}

\questionI{5. Qu canales y~blogueros determinan tus decisiones de~inversin?}

\questionI{6. Qu creencias sobre el~dinero y~la~tecnologa influyen en~tus decisiones al~comprar dispositivos para tu~hijo?}

\questionI{7. Qu realmente impulsa tu~deseo de~ganar dinero rpido?}

\questionI{8. Qu vacos internos intentas llenar al~seguir las tendencias de~criptomonedas?}

\questionI{9. De qu sentimiento incmodo sobre ti mismo huyes al~pasar tiempo en~el~telfono?}

\questionI{10. Qu parte de~tu vulnerabilidad ocultas detrs de~tus hbitos digitales?}

\questionI{11. Quin sientes que~eres sin pantallas, notificaciones ni~logros digitales?}


\newpage

\chapter{FOMO de~Criptomonedas}
\programbreak

\cluster{Pulso Digital}{5-10 min}{dnde estoy con las criptos}

\questionF{1. Con qu frecuencia revisas tu~portafolio o~los grficos? S honesto: una vez al~da, cada hora o~con cada notificacin?}

\questionF{2. El estado del~mercado afecta tu~estado de~nimo? Cuando el~mercado est en~rojo, cmo se~refleja esto en~tu comunicacin con tus seres queridos?}

\questionF{3. Duermes con el~mvil en~la~cama por miedo a~perderte los cambios de~precios durante la~noche?}


\cluster{Espejo Social}{10-20 min}{lo~que piensan otras personas}

\questionE{1. Los xitos de~quin en~criptomonedas te~generan sentimientos encontrados? A quin sigues?}

\questionE{2. Crees que~otros estn ganando dinero fcil y~rpidamente mientras t~ests perdiendo oportunidades?}

\questionE{3. Qu sentimiento es ms fuerte: la~alegra por tus ganancias o~la~molestia de~que podras haber ganado ms?}


\cluster{Pensamiento Mgico}{10-20 min}{mis sueos de~dinero fcil}

\questionE{1. Entraste al~mundo cripto para preservar tu~capital o~para cambiar rpidamente tu~vida y~escapar de~la~realidad?}

\questionE{2. Cmo te~relacionas con la~inversin en~criptomonedas: como una~estrategia bien pensada o~como un~juego de~azar?}

\questionE{3. Qu te~da esa sensacin de~estar «en el~mercado»: conexin con el~futuro, emocin, sentimiento de~superioridad?}


\cluster{Identidad}{10-20 min}{quin soy sin el~hype}

\questionE{1. Quin eres sin tus activos virtuales ni~logros digitales?}

\questionE{2. Si~todo el~mercado cripto desapareciera maana, qu habilidades reales y~valores autnticos te~quedaran?}


\cluster{Desintoxicacin Cripto}{15-30 min}{cmo dejar de~hacer seguimiento}

\questionI{1. Cunto dinero ests dispuesto a~perder sin arrepentimientos y~sin que~afecte tu~calidad de~vida? (Tu verdadera tolerancia al~riesgo)}

\questionI{2. Qu reglas de~bienestar digital ests dispuesto/a a~implementar para recuperar tu~sueo reparador?}



\newpage
\thispagestyle{empty}
\vspace*{0.2\textheight}

{\centering\sffamily\bfseries\fontsize{24pt}{28pt}\selectfont Para cerrar\par}

\vspace{2em}

Has recorrido un largo camino.

No importa si respondiste todas las preguntas o solo algunas. Lo que importa es que comenzaste esta conversacion contigo mismo.

\vspace{1.5em}
{\sffamily\bfseries Que hacer a continuacion}

\textbf{Relee tus notas.} Despues de una semana, despues de un mes. Te sorprendera como cambia tu percepcion.

\textbf{Observa los patrones.} Que temas se repiten? Que preguntas fueron las mas dificiles de responder? Esos son tus puntos de crecimiento.

\textbf{Regresa.} Este libro no es de un solo uso. Tu cambias. Tus respuestas cambiaran contigo.

\textbf{Actua.} Comprender es el primer paso. Pero el cambio real ocurre a traves de la accion.

\vspace{2em}
{\centering\itshape El autoconocimiento no es un destino, sino un camino.\par}

\vspace{1em}
{\centering\itshape Ya estas en el.\par}

\vspace{3em}
{\centering\sffamily Selfology --- el arte de entenderte a ti mismo.\par}

\end{document}