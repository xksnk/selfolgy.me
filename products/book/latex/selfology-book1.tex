% Selfology Book 1: Тематические программы
% Полная книга - 29 программ, 190 кластеров
% Сгенерировано: 2025-12-03 15:11

\documentclass[11pt,a4paper,oneside]{book}
\usepackage{selfology-book}

\begin{document}


\thispagestyle{empty}
\vspace*{0.2\textheight}

{\centering\sffamily\bfseries\fontsize{24pt}{28pt}\selectfont Добро пожаловать\par}

\vspace{2em}

Эта книга — не тест и не анкета. Это пространство для встречи с собой.

Здесь нет правильных или неправильных ответов. Есть только ваши — честные, настоящие, живые.

\vspace{1.5em}
{\sffamily\bfseries Как устроена эта книга}

Вы увидите три уровня вопросов:

\textbf{Поверхность} — лёгкие вопросы для разминки. Они помогут настроиться и войти в контакт с собой.

\textbf{Исследование} — вопросы глубже. Здесь начинается настоящая работа. Может быть некомфортно — это нормально.

\textbf{Глубина} — самые важные вопросы. Они требуют сил, времени и безопасного пространства.

\vspace{1.5em}
{\sffamily\bfseries Метафора аквариума}

Представьте, что ваш разум — это аквариум.

Наверху — кристально чистая вода. Там вы ясно видите свои мысли, желания, решения. Но эта часть составляет лишь 10 – 20\%. Это ваше сознание.

Остальные 80 – 90\% — мутная вода на глубине. Там трудно что-то разглядеть. Это ваше подсознание — место, где живут настоящие причины ваших решений, страхов и желаний.

Эта книга — как сито. Вы будете доставать со дна мысли и чувства, которые обычно остаются невидимыми.

\newpage
\thispagestyle{empty}
\vspace*{0.1\textheight}

{\sffamily\bfseries Как проходить}

\textbf{Создайте пространство.} Выберите время, когда вас никто не потревожит.

\textbf{Настройтесь.} Налейте себе любимый напиток. Устройтесь удобно. Дайте себе разрешение на честность.

\textbf{Пишите от руки.} Когда вы пишете рукой, мозг обрабатывает информацию глубже.

\textbf{Не торопитесь.} У каждого вопроса указано рекомендуемое время. Но это лишь ориентир.

\vspace{1.5em}
{\sffamily\bfseries Техника безопасности}

Вам понадобятся силы на это путешествие.

Не пытайтесь пройти всё за один раз. Двигайтесь в своём темпе. Делайте перерывы.

Если какой-то вопрос вызывает сильные эмоции — это знак того, что вы коснулись чего-то важного. Побудьте с этим бережно.

\vspace{1.5em}
{\sffamily\bfseries Время на разделы}

\textbf{5 – 10 минут} — вопросы уровня Поверхность

\textbf{10 – 20 минут} — вопросы уровня Исследование

\textbf{15 – 30 минут} — вопросы уровня Глубина

\vspace{2em}
{\centering\itshape Готовы? Тогда начнём.\par}

\chapter{Подумать о~жизни}
\programbreak

\cluster{Здесь и~сейчас}{5 – 10 мин}{где я~сейчас}

\questionF{1. Как~бы~ты описал своё состояние прямо сейчас тремя словами?}

\questionF{2. Как~обычно выглядит твоё утро?}

\questionF{3. Что~ты делаешь, когда у~тебя есть свободное время и~никто ничего от~тебя не~ждёт?}

\questionF{4. Когда последний раз~ты чувствовал полную энергию и~жизненную силу? Что~происходило вокруг?}


\cluster{Люди рядом}{10 – 20 мин}{кто~влияет на~меня}

\questionE{1. Кто~из~людей оказал на~тебя особенно сильное влияние? Что~именно этот человек сделал или дал тебе?}

\questionE{2. С~кем ты~проводишь больше всего времени в~последнее время?}

\questionE{3. Помогают ли~тебе эти люди расти и~развиваться, или скорее удерживают на~привычном месте?}

\questionE{4. О~чём ты~никогда не~говоришь с~семьёй?}

\questionE{5. В~чём ты~чувствуешь свою уникальность по~сравнению с~семьёй?}


\cluster{Мечты и~амбиции}{10 – 20 мин}{куда хочу прийти}

\questionE{1. Каким ты~представляешь свой идеальный день отдыха? Где проводишь время, что~делаешь, с~кем?}

\questionE{2. О~чём ты~мечтаешь, когда представляешь финансовую свободу?}

\questionE{3. Чему ты~хочешь научиться в~первую очередь?}

\questionE{4. Где хочешь побывать и~почему именно там?}

\questionE{5. Ради чего ты~готов вставать в~самые ранние часы?}

\questionE{6. Представь: ты~изменил мир к~лучшему. Что~конкретно изменилось?}


\cluster{Ценности и~убеждения}{10 – 20 мин}{что~для меня важно}

\questionE{1. Какие ценности ты~для~себя сформулировал? Выпиши хотя бы~три.}

\questionE{2. В~чём ты~видишь свою уникальность по~сравнению с~близкими людьми?}

\questionE{3. Закончи фразу: «Если бы~люди знали, что~я на~самом деле..., они бы...»}

\questionE{4. Любишь ли~ты себя?}

\questionE{5. Как~ты относишься к~себе? С~теплотой или критично? Что~влияет на~это отношение?}


\cluster{Защита и~самообман}{10 – 20 мин}{от~чего я~прячусь}

\questionE{1. От~чего ты~пытаешься защитить себя, когда обманываешь себя?}

\questionE{2. От~чего тебе важно освободиться перед новыми свершениями?}

\questionE{3. С~какими глубинными страхами ты~встречаешься в~своей жизни?}

\questionE{4. Что~тебя больше всего пугает в~близости с~другими людьми?}


\cluster{Прошлое}{10 – 20 мин}{мой путь до~сегодня}

\questionE{1. Кем ты~был год назад? Кто~ты~сейчас? Как~изменился?}

\questionE{2. Как~бы~ты оценил свой прошедший год по~основным сферам жизни по~шкале от~1 до~10? Какие важные события в~каждой сфере повлияли на~твою оценку?}

\questionE{3. Вспомни 3 привычки, которые помогли тебе в~этом году. Как~именно они работали?}

\questionE{4. Вспомни 2-3 случая, когда всё пошло не~по~плану. К~чему хорошему это~тебя в~итоге привело?}

\questionE{5. Чего ты~не~смог достичь в~этом году и~почему? Что~ты об~этом думаешь сейчас?}

\questionE{6. Есть ли~что-то, за~что~ты до~сих пор злишься на~себя или не~можешь себя простить?}


\cluster{Гордость и~смысл}{15 – 30 мин}{чем я~горжусь}

\questionI{1. Чем ты~гордишься больше всего?}

\questionI{2. Если бы~у тебя был всего один год жизни, как~бы~ты хотел его прожить?}

\questionI{3. Если завтра всё потеряет значение, что~стало бы~твоим самым большим сожалением?}


\cluster{Образ будущего}{15 – 30 мин}{кем я~хочу стать}

\questionI{1. Представь себя в~будущем --- таким, каким ты~хочешь быть в~том, что~для тебя важно. Какие советы ты~бы~дал себе сегодняшнему?}

\questionI{2. Если бы~родители действительно видели тебя, что~бы~они поняли?}

\questionI{3. Чего ты~хочешь прямо сейчас больше всего?}


\newpage

\chapter{Подумать о~карьере или бизнесе}
\programbreak

\cluster{Где я~сейчас}{5 – 10 мин}{мои дела на~работе}

\questionF{1. В~какой сфере ты~сейчас работаешь или учишься?}

\questionF{2. Сколько лет ты~уже в~этой сфере?}

\questionF{3. Есть ли~у тебя профильное образование в~твоей текущей сфере?}

\questionF{4. Работаешь ли~ты в~офисе, удалённо или гибридно?}

\questionF{5. Какой у~тебя сейчас режим работы?}


\cluster{Удовлетворённость}{10 – 20 мин}{мне нравится работа?}

\questionE{1. Насколько по~шкале от~1 до~10 тебя удовлетворяет твоя работа прямо сейчас?}

\questionE{2. Что~ты чувствуешь к~своей работе? Что~в ней откликается тебе, а~что --- нет?}

\questionE{3. Насколько твои личные ценности совпадают с~корпоративной культурой организации, где ты~работаешь?}

\questionE{4. Случается ли, что~ты критикуешь себя за~отдых, даже когда много работаешь?}


\cluster{Цели и~мечты}{10 – 20 мин}{мои мечты о~работе}

\questionE{1. Какая она --- работа твоей мечты?}

\questionE{2. Чем бы~ты занимался, если бы~вся работа оплачивалась одинаково?}

\questionE{3. Что~бы~ты делал, если бы~деньги не~были ограничением?}

\questionE{4. Представь: через 5 лет встречаешь старого друга. Что~о своей работе ты~расскажешь с~гордостью?}

\questionE{5. Если бы~ты знал, что~точно добьёшься успеха, чем бы~занялся прямо сейчас?}


\cluster{Преграды и~ресурсы}{10 – 20 мин}{что~тормозит и~что помогает}

\questionE{1. Что~именно в~твоей текущей работе может мешать тебе двигаться к~профессиональным целям?}

\questionE{2. Если тебе нужен наставник в~работе, кто~это~мог бы~быть и~как ты~можешь найти такого человека?}

\questionE{3. Если завтра ты~решишь не~работать, сколько времени сможешь прожить на~свои сбережения?}

\questionE{4. Куда конкретно сейчас идут твои деньги: в~будущее (инвестиции, обучение, развитие) или в~настоящее (покупки, развлечения, комфорт)?}


\cluster{Выборы и~действия}{10 – 20 мин}{мои следующие шаги}

\questionE{1. Куда бы~ты хотел пойти работать, если бы~решил что-то изменить в~своей карьере?}

\questionE{2. Если представить весы: на~одной чаше --- отказы на~собеседованиях, на~другой --- работа там, где тебя не~ценят. Что~для тебя важнее избежать?}

\questionE{3. На~чём основана твоя уверенность в~том, что~твоя идея сработает? Какие данные, опыт или наблюдения поддерживают твой выбор?}

\questionE{4. Если бы~у тебя были все~ресурсы (время, деньги, связи), чем бы~ты хотел заниматься и~как бы~проводил своё время?}


\cluster{Бизнес-метрики (для предпринимателей)}{15 – 30 мин}{}

\questionI{1. Сколько стоит привлечение одного нового клиента в~твой бизнес?}

\questionI{2. Какой главный показатель эффективности (метрика) ты~отслеживаешь в~своём бизнесе?}


\newpage

\chapter{Задуматься о~здоровье}
\programbreak

\cluster{Сон и~восстановление}{5 – 10 мин}{мой сон и~отдых}

\questionF{1. Во сколько ты~обычно ложишься спать и~во сколько встаёшь?}

\questionF{2. Сколько часов ты~обычно спишь за~ночь?}

\questionF{3. Как~ты обычно чувствуешь себя по~утрам: бодро, вяло или сонливо?}

\questionF{4. Что~ты обычно делаешь за~час до~сна? Помогает ли~это тебе заснуть?}

\questionF{5. Ты~скорее жаворонок или сова? И~что это~даёт тебе в~жизни?}


\cluster{Физиология}{10 – 20 мин}{что~ем и~как двигаюсь}

\questionE{1. Как~ты понимаешь, что~нужно твоему телу, а~что --- нет?}

\questionE{2. Сколько воды ты~выпиваешь в~день? Замечаешь ли~связь между количеством выпитой воды и~тем, как~ты себя чувствуешь?}

\questionE{3. Какие виды физической активности приносят тебе радость?}

\questionE{4. Как~ты относишься к~своему физическому здоровью? Что~для тебя значит быть здоровым?}


\cluster{Сигналы тела}{10 – 20 мин}{что~чувствует моё тело}

\questionE{1. Что~происходит с~твоим телом, когда ты~истощён? Какие первые сигналы подаёт организм?}

\questionE{2. Где в~твоём теле чаще всего накапливается напряжение? Что~твоё тело пытается тебе сказать?}

\questionE{3. Когда ты~последний раз~прислушивался к~своему телу и~следовал его сигналам вместо того, чтобы «пересиливать»?}


\cluster{Смысл здоровья}{15 – 30 мин}{зачем мне быть здоровым}

\questionI{1. Почему для~тебя важно иметь хорошее здоровье? Что~оно даёт тебе?}

\questionI{2. Если бы~у тебя было идеальное здоровье, что~бы~изменилось в~твоей жизни?}


\cluster{Здоровье в~будущем}{15 – 30 мин}{куда хочу прийти}

\questionI{1. Какие изменения в~здоровье ты~хотел бы~увидеть в~своей жизни через год?}

\questionI{2. Где ты~видишь себя в~плане здоровья через 1 год? Как~ты будешь себя чувствовать?}

\questionI{3. Какой отдых нужен твоему телу и~душе? Где и~как ты~хотела бы~восстановить силы?}


\newpage

\chapter{Изучить себя}
\programbreak

\cluster{Архетипы и~модели}{5 – 10 мин}{кто~на~меня влиял}

\questionF{1. Кто~оказал на~тебя самое сильное влияние? Какие его качества или подходы ты~находишь в~себе?}

\questionF{2. Кем ты~восхищаешься больше всего? Почему именно этим человеком?}


\cluster{Энергия и~ресурсы}{10 – 20 мин}{откуда беру силы}

\questionE{1. Как~бы~ты описал своё энергетическое состояние прямо сейчас тремя словами?}

\questionE{2. Когда в~последний раз~ты чувствовал полную жизненную силу и~энергию? Что~происходило в~твоей жизни в~тот момент?}

\questionE{3. Какие 3 действия или ситуации возвращают тебе энергию быстрее всего?}


\cluster{Ценности через горделость}{10 – 20 мин}{чем я~горжусь}

\questionE{1. Чем ты~гордишься больше всего? Почему это~важно именно для~тебя?}

\questionE{2. Какие свои качества или особенности ты~знаешь лучше всех, но~редко о~них говоришь?}


\cluster{Желания и~страхи}{10 – 20 мин}{чего хочу и~боюсь}

\questionE{1. Чего ты~хочешь прямо сейчас больше всего?}

\questionE{2. Какой страх живёт в~самой глубине твоей души? Что~он пытается защитить?}


\cluster{Идеальное будущее}{15 – 30 мин}{кем хочу стать}

\questionI{1. Опиши свой идеальный день отдыха: где, с~кем, что~делаешь?}

\questionI{2. Представь: ты~изменил мир к~лучшему. Что~конкретно изменилось благодаря тебе?}

\questionI{3. Если бы~ты знал, что~не можешь потерпеть неудачу, чем бы~занялся?}


\cluster{Развитие и~путь}{15 – 30 мин}{куда я~расту}

\questionI{1. Чему ты~хочешь научиться в~первую очередь и~почему это~важно для~тебя сейчас?}

\questionI{2. Куда в~мире тебе хочется отправиться и~что тебя там привлекает?}

\questionI{3. Какой навык или качество ты~хотел бы~развить в~ближайший год?}


\newpage

\chapter{Улучшить эмоциональное состояние}
\programbreak

\cluster{Карта эмоций}{5 – 10 мин}{мне нравится работа?}

\questionF{1. Какие эмоции ты~чаще всего замечаешь у~себя в~течение дня?}

\questionF{2. Назови одну эмоцию, которая тебя удивляет своей частотой.}

\questionF{3. Есть ли~эмоция, которая живет в~твоем теле постоянно, как~фоновый шум?}


\cluster{Телесность эмоций}{10 – 20 мин}{что~тело мне говорит}

\questionE{1. Где в~теле ты~чувствуешь радость? Опиши ощущение: тепло, легкость, расширение?}

\questionE{2. Вспомни момент, когда тебя переполняла благодарность. Что~ты чувствовал в~теле?}

\questionE{3. Какие эмоции приносят тебе ощущение покоя и~гармонии? Как~они проявляются в~твоём теле?}

\questionE{4. Где в~твоем теле живет эмоция, которую ты~не~можешь выразить словами?}


\cluster{Осознание чувств}{10 – 20 мин}{что~я сейчас чувствую}

\questionE{1. Насколько хорошо ты~замечаешь свои чувства, когда они появляются? Оцени от~1 до~10.}

\questionE{2. Ты~позволяешь себе просто чувствовать то, что~есть, или часто думаешь «я не~должен это~чувствовать»?}

\questionE{3. Какие эмоции ты~позволяешь себе чувствовать только наедине с~собой?}

\questionE{4. Когда кто-то спрашивает «что ты~чувствуешь?», твой первый импульс --- ответить честно или сказать то, что~ожидают?}


\cluster{Защиты и~маски}{10 – 20 мин}{мои маски и~щиты}

\questionE{1. Используешь ли~ты юмор или сарказм, чтобы выразить чувства или чтобы их спрятать?}

\questionE{2. Есть ли~«безопасные» эмоции, которые ты~показываешь вместо настоящих чувств?}


\cluster{Эмоции как~информация}{10 – 20 мин}{что~чувства мне шепчут}

\questionE{1. О~чём тебе говорят чувства, которые ты~регулярно испытываешь?}

\questionE{2. Если бы~твоя тревога была мудрым учителем, чему бы~она тебя учила?}

\questionE{3. Какую эмоцию ты~часто расцениваешь как~«плохую», но~она на~самом деле пытается тебе помочь?}


\cluster{Настроение как~выбор}{15 – 30 мин}{как~управлять своим настроением}

\questionI{1. Какое настроение ты~хотел бы~создать у~себя утром перед началом дня?}

\questionI{2. Что~помогает тебе возвращаться в~равновесие, когда эмоции захлёстывают?}


\newpage

\chapter{Перебрать цели}
\programbreak

\cluster{Инвентаризация целей}{5 – 10 мин}{мои цели на~сегодня}

\questionF{1. Перечисли все~цели, которые ты~сейчас преследуешь (профессиональные, личные, финансовые).}

\questionF{2. Откуда появилась каждая из~этих целей: это~твой выбор или ожидание других людей?}

\questionF{3. От~какой цели пора отказаться --- она уже не~служит тебе и~не вдохновляет?}

\questionF{4. Может быть, эти цели, которые давят на~тебя, уже перестали быть актуальными?}


\cluster{Проверка целей}{10 – 20 мин}{что~из целей живое}

\questionE{1. Выбери одну цель, которая для~тебя важна. На~чём основана твоя уверенность, что~именно этот путь приведёт тебя к~результату?}

\questionE{2. На~каких конкретных данных или опыте ты~основываешь веру в~эту цель? Что~даёт тебе уверенность, что~она реально достижима?}

\questionE{3. Может ли~быть более эффективный путь к~этой же~цели? Что~поможет тебе это~выяснить?}

\questionE{4. Вдохновляют ли~тебя твои цели или скорее вызывают тревогу?}


\cluster{Профессиональные цели}{10 – 20 мин}{что~хочу от~работы}

\questionE{1. Какая твоя работа мечты? Представь её детально: где происходит, что~включает, с~какими людьми взаимодействуешь.}

\questionE{2. Что~в твоей текущей работе уже помогает тебе двигаться к~профессиональным целям, и~что ты~можешь сделать ещё?}

\questionE{3. Что~именно в~твоей текущей работе мешает тебе достичь твоих профессиональных целей?}

\questionE{4. Насколько твоя нынешняя работа помогает тебе двигаться к~профессиональным целям?}


\cluster{Достижения и~гордость}{15 – 30 мин}{чем я~горжусь}

\questionI{1. Какими своими достижениями ты~гордишься больше всего?}

\questionI{2. Какое достижение тебя удивило больше всего (ты не~ожидал, что~получится)?}


\cluster{Приоритизация}{10 – 20 мин}{что~важнее всего сейчас}

\questionE{1. Если бы~ты мог реализовать только 3 цели из~всех своих --- какие бы~это были?}

\questionE{2. Как~ты поймёшь, что~движешься к~своим целям в~правильном направлении?}


\cluster{Действие}{15 – 30 мин}{мой план действий}

\questionI{1. Какой один конкретный шаг ты~можешь сделать на~этой неделе, чтобы приблизиться к~своей главной цели?}

\questionI{2. Кто~или что~может помочь тебе приблизиться к~этой цели? Какие ресурсы и~поддержка тебе нужны?}


\newpage

\chapter{Мечтатели}
\programbreak

\cluster{Что~такое мечта для~меня}{5 – 10 мин}{что~значит мечтать}

\questionF{1. Вспомни детство: о~чём ты~мечтал тогда? Что~изменилось с~тех пор?}

\questionF{2. Какая твоя главная мечта сейчас?}

\questionF{3. Это~действительно твоя мечта или что-то, что~ты когда-то услышал от~других и~принял за~своё?}

\questionF{4. Насколько ты~живёшь сейчас жизнью своей мечты?}


\cluster{Проверка реальности}{10 – 20 мин}{что~реально осуществимо}

\questionE{1. Знаешь ли~ты кого-то, кто~уже живёт похожей жизнью или работает в~интересной тебе сфере? Что~помогло им достичь этого?}

\questionE{2. Может быть, твоя мечта основана на~ограниченной информации? Что~помогло бы~тебе глубже понять её реальность?}

\questionE{3. Если бы~у тебя были все~возможности и~ресурсы, что~бы~ты сделал в~первую очередь?}


\cluster{Дом мечты}{10 – 20 мин}{как~выглядит моя мечта}

\questionE{1. Если представить, что~все проблемы решены и~в жизни полный порядок --- где бы~ты жил и~как выглядел бы~твой дом мечты?}

\questionE{2. Представь свой дом мечты. Что~в нём особенного? Опиши 3-5 деталей, которые делают его именно твоим.}

\questionE{3. Какой небольшой и~безопасный шаг ты~можешь сделать сегодня, чтобы приблизиться к~дому своей мечты?}


\cluster{Работа мечты}{10 – 20 мин}{где работать с~радостью}

\questionE{1. Какой ты~видишь свою работу мечты? Попробуй описать её как~можно конкретнее.}

\questionE{2. Что~могло бы~измениться в~твоей жизни, чтобы ты~смог заняться работой своей мечты?}

\questionE{3. Кто~может помочь тебе приблизиться к~работе мечты? Какие контакты, знания или ресурсы тебе нужны?}


\cluster{Деньги и~инвестиции}{10 – 20 мин}{куда уходят мои деньги}

\questionE{1. Куда конкретно сейчас направляются твои деньги: в~будущее (инвестиции, обучение, активы) или в~настоящее (комфорт, развлечения, покупки)?}

\questionE{2. Какой процент твоего дохода ты~инвестируешь в~реализацию своей мечты?}

\questionE{3. Если бы~ты мог выделить 10\% от~своего дохода на~воплощение мечты, на~что~бы~ты их потратил?}


\cluster{Действие}{15 – 30 мин}{мой первый шаг сегодня}

\questionI{1. Какой один конкретный шаг ты~можешь сделать на~этой неделе, чтобы приблизиться к~своей мечте?}

\questionI{2. Если ты~не~предпримешь действий, какой ты~видишь свою жизнь через несколько лет? Что~для тебя будет самым болезненным упущением?}

\questionI{3. Когда ты~хочешь начать? Может быть, уже на~этой неделе есть подходящий день?}


\newpage

\chapter{Рефлексия}
\programbreak

\cluster{Что~я узнал о~себе}{5 – 10 мин}{что~я про~себя понял}

\questionF{1. Какое открытие о~себе оказалось самым неожиданным?}

\questionF{2. Что~ты узнал о~себе, чего не~знал до~рефлексии?}

\questionF{3. Какие убеждения о~себе ты~переосмыслил или изменил за~последнее время?}


\cluster{Свобода и~изменения}{10 – 20 мин}{что~во мне изменилось}

\questionE{1. Какая свобода приходит с~новым пониманием себя?}

\questionE{2. Что~ты теперь позволяешь себе делать, чего раньше не~позволял?}

\questionE{3. От~чего ты~готов отказаться? Что~в твоей жизни больше не~служит тебе?}


\cluster{Действие}{10 – 20 мин}{что~делаю дальше}

\questionE{1. Какое одно действие ты~готов предпринять сегодня, опираясь на~своё новое понимание?}

\questionE{2. С~кем ты~можешь поделиться своими новыми открытиями о~себе? Кто~готов поддержать тебя в~этом?}


\cluster{Продолжение пути}{15 – 30 мин}{куда иду теперь}

\questionI{1. Какой вопрос к~себе ты~будешь задавать регулярно?}

\questionI{2. Когда ты~снова пройдёшь эту рефлексию? Через месяц, квартал, год?}


\newpage

\chapter{3 кита очищения}
\programbreak

\cluster{Принятие}{5 – 10 мин}{что~я готов принять}

\questionF{1. Какую часть себя ты~когда-то отвергал, но~со временем смог принять?}

\questionF{2. Что~изменилось в~твоей жизни, когда ты~принял то, с~чем раньше боролся?}

\questionF{3. Какая часть тебя пока ещё ждёт принятия?}


\cluster{Отпускание}{10 – 20 мин}{что~готов отпустить}

\questionE{1. От~чего ты~смог отпустить в~этом году --- убеждений, привычек или отношений?}

\questionE{2. Что~происходит с~тобой, когда ты~что-то отпускаешь? Как~это меняет твоё внутреннее состояние?}

\questionE{3. Что~ты держишь сейчас, зная, что~пора отпустить?}


\cluster{Преобразование}{15 – 30 мин}{как~я меняюсь}

\questionI{1. Когда последний раз~твой рост происходил не~через достижения, а~через принятие и~отпускание того, что~не получалось?}

\questionI{2. Что~нового в~себе ты~открыл благодаря этому процессу очищения?}

\questionI{3. Каким ты~стал после этого цикла очищения --- что~в тебе изменилось?}


\cluster{Интеграция}{15 – 30 мин}{что~изменится в~жизни}

\questionI{1. Как~это новое понимание может повлиять на~твои решения в~ближайшем и~далёком будущем?}

\questionI{2. Какой один конкретный шаг ты~готов сделать уже сегодня, опираясь на~это очищение?}

\questionI{3. Какие новые знания или навыки могли бы~углубить твой опыт очищения и~сделать его более целостным?}


\newpage

\chapter{Ресурс}
\programbreak

\cluster{Люди как~ресурс}{5 – 10 мин}{кто~меня поддерживает}

\questionF{1. Кто~оказал на~тебя особенно сильное влияние? Что~именно этот человек сделал или сказал?}

\questionF{2. Кто~сейчас --- твой главный источник поддержки?}

\questionF{3. На~кого ты~можешь положиться в~сложный момент? Почему именно на~этих людей?}

\questionF{4. Кто~верит в~тебя, даже когда ты~сам в~себе сомневаешься?}


\cluster{Внутренние ресурсы}{10 – 20 мин}{что~даёт мне силы}

\questionE{1. Чем ты~гордишься больше всего? Почему это~важно для~тебя?}

\questionE{2. Какая твоя главная сила? Как~ты её используешь?}

\questionE{3. Какой опыт из~прошлого помогает тебе сейчас? Чему он~тебя научил?}

\questionE{4. Какие 3 качества в~себе ты~считаешь своим преимуществом?}


\cluster{Материальные ресурсы}{10 – 20 мин}{что~у меня есть}

\questionE{1. Какие материальные ресурсы у~тебя есть? (деньги, имущество, образование, связи)}

\questionE{2. Как~ты сейчас используешь свои ресурсы --- время, энергию, финансы? Что~работает хорошо, а~что хочется изменить?}

\questionE{3. Какие ресурсы (время, деньги, энергия) у~тебя в~избытке, а~каких не~хватает для~того, что~действительно важно?}


\cluster{Времяи энергия}{10 – 20 мин}{куда уходят мои силы}

\questionE{1. На~что~у тебя уходит больше всего времени? Приносит ли~это пользу тебе?}

\questionE{2. Какие занятия дают тебе энергию? Как~часто ты~к~ним обращаешься?}

\questionE{3. От~какой активности ты~теряешь энергию? Можешь ли~ты это~изменить?}


\cluster{Развитие ресурсов}{15 – 30 мин}{как~стать сильнее}

\questionI{1. Какой ресурс ты~хотел бы~развить в~первую очередь?}

\questionI{2. Что~нужно сделать, чтобы получить доступ к~новым ресурсам?}

\questionI{3. К~кому или к~чему ты~можешь обратиться, чтобы расширить свои возможности для~развития?}


\newpage

\chapter{Границы личности}
\programbreak

\cluster{Что~такое граница для~меня}{5 – 10 мин}{что~значит мечтать}

\questionF{1. Где ты~часто позволяешь людям переступать через свои границы?}

\questionF{2. Почему ты~позволяешь это? Из~страха, вежливости, привычки?}

\questionF{3. С~кем тебе легче выстроить границы, а~с кем сложнее? Почему?}


\cluster{Личные границы}{10 – 20 мин}{мои «да» и~«нет}

\questionE{1. Какое из~твоих желаний больше не~служит тебе?}

\questionE{2. От~какой цели пора отказаться --- она уже не~служит тебе?}

\questionE{3. Какой ритуал или привычку ты~бы~хотел ввести или изменить? Зачем?}

\questionE{4. Испытываешь ли~ты чувство вины, когда нарушаешь собственные правила?}


\cluster{Границы в~отношениях}{10 – 20 мин}{где я~открываюсь людям}

\questionE{1. С~кем из~близких тебе людей ты~можешь позволить себе быть уязвимым, а~с кем чувствуешь потребность казаться сильным?}

\questionE{2. Кому ты~позволяешь видеть твою слабость и~уязвимость?}

\questionE{3. Перед кем ты~надеваешь маску и~почему?}

\questionE{4. Какие люди требуют от~тебя быть всегда «на высоте»? Хочешь ли~ты это~менять?}


\cluster{Границы в~работе}{10 – 20 мин}{где работа и~где я}

\questionE{1. Насколько рабочие обязанности совпадают с~твоими интересами и~ценностями?}

\questionE{2. Если поставить на~чашу весов отказы на~собеседованиях и~работу там, где тебя не~ценят --- что~для тебя перевесит?}

\questionE{3. Что~делает именно тебя подходящим человеком для~решения этой задачи? Кто~ещё мог бы~с ней справиться?}

\questionE{4. Когда ты~последний раз~сказал «нет» дополнительной работе? Что~помешало раньше?}


\cluster{Границы с~технологией и~временем}{10 – 20 мин}{где уходит мое время}

\questionE{1. Когда последний раз~ты проводил день без~телефона и~интернета?}

\questionE{2. Установил ли~ты границы для~интернета в~своей жизни? Какие?}

\questionE{3. Что~из того, чем ты~занимался сегодня, приблизило тебя к~твоим главным целям? А~что, возможно, было не~так~важно?}


\cluster{Граница между работой и~личной жизнью}{10 – 20 мин}{}

\questionE{1. Есть ли~у тебя чёткое время, когда работа заканчивается?}

\questionE{2. Если завтра всё в~жизни потеряет значение, о~чём ты~больше всего сожалел бы?}


\cluster{Физические и~телесные границы}{10 – 20 мин}{что~говорит моё тело}

\questionE{1. Какие физические границы ты~четко соблюдаешь? А~какие часто нарушаешь?}

\questionE{2. Полюбил ли~ты себя настолько, чтобы отказаться от~того, что~тебя разрушает?}


\cluster{Действие}{15 – 30 мин}{мои первые шаги}

\questionI{1. Какую одну границу ты~хочешь выстроить в~ближайший месяц?}

\questionI{2. Как~ты поймёшь, что~эта граница работает? Какие признаки покажут тебе это?}

\questionI{3. Кого ты~попросишь поддержать тебя в~выстраивании этой границы?}


\newpage

\chapter{Работа со страхами}
\programbreak

\cluster{Знакомство со страхом}{5 – 10 мин}{мне нравится работа?}

\questionF{1. Какой страх живет в~тебе так~давно, что~стал частью тебя?}

\questionF{2. Как~часто этот страх себя проявляет? Когда он~активизируется?}

\questionF{3. Где в~теле ты~чувствуешь свой страх?}

\questionF{4. Какой возможный сценарий будущего вызывает у~тебя больше всего тревоги или беспокойства?}


\cluster{Встреча со страхом}{10 – 20 мин}{что~чувствую когда страшно}

\questionE{1. Вспомни момент, когда ты~встретил свой страх лицом к~лицу. Что~ты почувствовал?}

\questionE{2. Что~происходит с~твоим страхом, когда ты~дышишь через него? Становится ли~легче?}

\questionE{3. Что~тёплого и~поддерживающего ты~мог бы~сказать своему страху?}


\cluster{Глубокий анализ страха}{10 – 20 мин}{что~скрывается за~страхом}

\questionE{1. Если бы~твой страх мог говорить, что~бы~он сказал о~том, что~защищает?}

\questionE{2. Если бы~твоя тревога была мудрым учителем, чему бы~она тебя учила?}

\questionE{3. Какая часть тебя уже не~боится того, что~пугало раньше?}


\cluster{Специфичные страхи}{10 – 20 мин}{мои главные страхи}

\questionE{1. Какие финансовые страхи не~дают тебе спать спокойно?}

\questionE{2. Чего ты~хочешь, но~боишься захотеть по-настоящему?}

\questionE{3. Как~думаешь, могут ли~другие люди чувствовать неловкость или страх в~общении с~тобой? Если да, то~с чем это~может быть связано?}


\cluster{Страх и~мотивация}{10 – 20 мин}{что~страх делает со мной}

\questionE{1. Твой страх мотивирует тебя двигаться вперёд или парализует тебя?}

\questionE{2. Когда ты~избегаешь определённых ситуаций --- что~именно тебя в~них пугает: реальная опасность или возможность узнать что-то новое о~себе?}

\questionE{3. Что~ты избегаешь так~тонко, что~даже сам этого не~замечаешь?}


\cluster{Страх в~отношениях}{10 – 20 мин}{близость пугает меня}

\questionE{1. Ты~даришь людям время и~внимание из~щедрости или из~страха быть отвергнутым?}

\questionE{2. Что~тебя больше пугает --- показать свою уязвимость или риск того, что~её могут использовать против тебя?}

\questionE{3. Что~тебя больше пугает --- остаться одному или встретиться с~собой наедине?}

\questionE{4. Если бы~ты не~боялся одиночества, какие отношения ты~бы~закончил?}


\cluster{Экзистенциальные страхи}{10 – 20 мин}{чего боюсь по-настоящему}

\questionE{1. Чего ты~боишься больше --- интенсивности своих эмоций или их полного отсутствия?}

\questionE{2. Кем ты~станешь, когда перестанешь бояться?}

\questionE{3. Кем ты~становишься, когда выбираешь любовь вместо страха?}


\cluster{Бессознательные страхи}{10 – 20 мин}{чего боюсь втайне}

\questionE{1. Какую правду о~себе ты~уже чувствуешь, но~пока не~готов с~ней встретиться?}

\questionE{2. Если бы~твои близкие отношения могли говорить, о~каких твоих скрытых переживаниях они бы~рассказали?}


\cluster{Действие}{15 – 30 мин}{что~я делаю сейчас}

\questionI{1. Какой один небольшой шаг ты~мог бы~сделать навстречу тому, что~тебя тревожит?}

\questionI{2. Какой человек или ресурс поможет тебе в~этой работе?}


\newpage

\chapter{Разобраться в~отношениях}
\programbreak

\cluster{Карта моих отношений}{5 – 10 мин}{мои близкие и~дальние}

\questionF{1. Сколько людей знают тебя настоящего? Кто~они?}

\questionF{2. С~кем в~твоей жизни ты~чувствуешь безопасность быть собой?}

\questionF{3. Кто~для~тебя пример гармоничных и~здоровых отношений?}

\questionF{4. Насколько ты~доволен своими отношениями с~друзьями сейчас? Оцени по~шкале от~1 до~10.}


\cluster{Успехи и~ресурсы в~отношениях}{10 – 20 мин}{мои победы в~любви}

\questionE{1. Каким успехом в~отношениях ты~можешь гордиться?}

\questionE{2. Вспомни отношения, где ты~чувствовал себя по-настоящему увиденным. Что~в них было особенного?}

\questionE{3. Вспомни человека, который принимает тебя полностью. Что~ты чувствуешь рядом с~ним?}

\questionE{4. Вспомни отношения, которые помогли тебе стать лучше. Что~в них было целительного?}

\questionE{5. Чем ты~помог другому человеку? Что~это дало тебе?}


\cluster{Паттерны и~циклы}{10 – 20 мин}{мои повторяющиеся ошибки}

\questionE{1. Какие паттерны или поведение разрушали твои прошлые отношения?}

\questionE{2. Какие модели поведения повторяются в~твоих отношениях?}

\questionE{3. Ты~притягиваешь людей, которые нуждаются в~помощи, или сам выбираешь тех, кого можешь помогать?}

\questionE{4. Ты~чаще прощаешь людям или притворяешься, что~простил, чтобы избежать конфронтации?}


\cluster{Я~в~отношениях}{10 – 20 мин}{кто~я~рядом с~другими}

\questionE{1. Есть ли~части твоей личности, которые ты~предпочитаешь не~показывать в~отношениях?}

\questionE{2. С~кем из~близких тебе людей ты~можешь быть уязвимым и~настоящим, а~рядом с~кем чувствуешь потребность казаться сильнее?}

\questionE{3. Где в~теле ты~чувствуешь, когда отношения не~правильные для~тебя?}

\questionE{4. Ты~выбираешь близость с~людьми или близость к~образу, который они о~тебе создали?}

\questionE{5. Какую часть себя ты~скрываешь или не~показываешь в~отношениях с~людьми?}


\cluster{Любовь и~выражение}{10 – 20 мин}{мой способ любить}

\questionE{1. Как~ты проявляешь любовь к~близким? А~как хочешь её получать?}

\questionE{2. Как~ты выражаешь свою любовь людям, которых ты~любишь?}

\questionE{3. Ты~даришь людям время и~внимание из~щедрости или из~страха быть отвергнутым?}


\cluster{Влияние происхождения}{10 – 20 мин}{что~мне дали родители}

\questionE{1. Как~твои отношения с~родителями влияют на~твои романтические связи?}

\questionE{2. Какие черты отношений из~твоего детства ты~хотел бы~сохранить в~своей взрослой жизни? А~что хочешь изменить?}

\questionE{3. Как~часто и~качественно ты~общаешься с~близкими родственниками? Это~тебя обогащает?}


\cluster{Уязвимость и~близость}{10 – 20 мин}{открыться или защищаться}

\questionE{1. Что~мешает тебе быть открытым с~близкими людьми?}

\questionE{2. Ты~ищешь в~отношениях понимание или подтверждение того, что~ты уже думаешь о~себе?}

\questionE{3. Если бы~ты не~боялся одиночества, какие отношения ты~бы~закончил?}


\cluster{Трансформация}{10 – 20 мин}{кто~я~теперь}

\questionE{1. Кем ты~становишься, когда выбираешь любовь вместо страха?}

\questionE{2. Если бы~твои близкие отношения могли говорить, что~бы~они рассказали о~том, что~тебе важно скрывать или защищать?}


\cluster{Действие}{15 – 30 мин}{что~буду делать дальше}

\questionI{1. Каким образом ты~можешь улучшить свои отношения с~кем-то из~близких? Какой первый шаг ты~готов сделать?}

\questionI{2. С~кем ты~готов поделиться чем-то более личным о~себе на~этой неделе?}

\questionI{3. Какие качества в~себе ты~хочешь развить для~более близких и~доверительных отношений?}


\newpage

\chapter{Выгорание  Ресурс}
\programbreak

\cluster{Признаки выгорания}{5 – 10 мин}{что~чувствую и~замечаю}

\questionF{1. Что~происходит с~твоим телом, когда ты~истощён? Какие сигналы подаёт организм?}

\questionF{2. Какая эмоция прячется за~твоей хронической усталостью?}

\questionF{3. В~какие моменты ты~чувствуешь злость, но~называешь это~усталостью или раздражением?}

\questionF{4. Когда твои цели питают тебя энергией, а~когда начинают истощать?}


\cluster{Триггеры и~паттерны}{10 – 20 мин}{что~меня истощает}

\questionE{1. Что~в твоей жизни забирает больше всего энергии?}

\questionE{2. Когда ты~последний раз~чувствовал настоящий отдых --- полный, без~вины и~срочности?}

\questionE{3. Какие обязанности ты~берешь на~себя из~чувства долга, хотя они тебя не~вдохновляют?}


\cluster{Восстановление --- быстрые способы}{10 – 20 мин}{скорая помощь себе}

\questionE{1. Что~помогает тебе быстро восстановиться после стресса и~пополнить энергию?}

\questionE{2. Какие активности помогают тебе лучше всего восстановить энергию? (музыка, общение, движение, природа, творчество)}

\questionE{3. Что~помогает тебе быстро расслабиться и~восстановить внутренний баланс?}


\cluster{Восстановление --- глубокие способы}{10 – 20 мин}{что~даёт мне силы}

\questionE{1. Какой вид отдыха восстанавливает тебя наиболее полно --- физический, эмоциональный, творческий или социальный?}

\questionE{2. Как~выглядит твой идеальный выходной? Что~наполняет тебя энергией и~даёт настоящий отдых?}

\questionE{3. Какие люди помогают тебе восстанавливать энергию? Как~ты понимаешь, что~рядом с~ними становишься более живым и~наполненным?}


\cluster{Самоподдержка}{10 – 20 мин}{как~я себя обнимаю}

\questionE{1. Как~ты поддерживаешь себя в~моменты усталости или упадка сил?}

\questionE{2. Что~ты говоришь себе, когда чувствуешь выгорание? Поддерживаешь или критикуешь?}

\questionE{3. Какой ритуал восстановления ты~хотел бы~ввести в~свою жизнь?}


\cluster{Баланс и~границы}{10 – 20 мин}{что~забираю и~отдаю}

\questionE{1. От~чего ты~готов отказаться, чтобы освободить место для~восстановления?}

\questionE{2. Какую границу ты~должен выстроить, чтобы не~доходить до~выгорания?}

\questionE{3. Как~бы~изменилась твоя жизнь, если бы~ты мог отдыхать столько же, сколько работаешь?}


\cluster{Действие}{15 – 30 мин}{мои первые шаги}

\questionI{1. Какой один ритуал восстановления ты~можешь ввести на~этой неделе?}

\questionI{2. Кому ты~расскажешь о~своём выгорании и~попросишь поддержки?}

\questionI{3. Если бы~у тебя был день полного отдыха без~всяких ограничений, как~бы~ты его провёл?}


\newpage

\chapter{Исцеление прошлого}
\programbreak

\cluster{Признание того, что~было}{5 – 10 мин}{моя правда о~прошлом}

\questionF{1. Какое событие из~твоего прошлого до~сих пор откликается в~тебе?}

\questionF{2. Какой урок ты~извлёк из~самого сложного периода своей жизни?}

\questionF{3. Чему важному ты~научился за~последнее время?}

\questionF{4. В~чём ты~стал сильнее за~последнее время? Что~это дало тебе?}


\cluster{Благодарность}{10 – 20 мин}{что~мне дало прошлое}

\questionE{1. Вспомни момент, когда тебя переполняла благодарность. Что~ты чувствовал в~теле?}

\questionE{2. За~что~ты можешь быть благодарен своему прошлому опыту?}

\questionE{3. Чем ты~помог другому человеку, и~что это~дало тебе?}


\cluster{Люди, которые помогали исцелять}{10 – 20 мин}{кто~был рядом}

\questionE{1. Вспомни отношения, которые помогли тебе стать лучше. Что~в них было целительного?}

\questionE{2. Вспомни отношения, где ты~чувствовал себя по-настоящему увиденным и~принятым. Что~делало эти отношения особенными для~твоего исцеления?}

\questionE{3. Вспомни человека, который принимает тебя полностью. Что~ты чувствуешь рядом с~ним?}

\questionE{4. С~кем ты~можешь молчать и~чувствовать себя понятым?}

\questionE{5. С~кем в~твоей жизни ты~чувствуешь безопасность быть собой?}


\cluster{Прощение}{10 – 20 мин}{как~отпустить боль}

\questionE{1. Кого ты~ещё не~простил --- других или себя?}

\questionE{2. Ты~чаще прощаешь людям или притворяешься, что~простил, чтобы избежать конфронтации?}

\questionE{3. Если бы~твоя грусть могла говорить, что~бы~она сказала о~твоей жизни?}

\questionE{4. Что~из твоего прошлого ты~готов отпустить, чтобы освободить место для~того, что~действительно важно для~тебя сейчас?}


\cluster{Эмоции прошлого}{10 – 20 мин}{что~тело помнит}

\questionE{1. Вспомни эмоцию из~прошлого, которая приносила тебе покой. Как~она ощущалась в~твоём теле?}

\questionE{2. Какая эмоция из~прошлого всё ещё живёт в~твоём теле, даже если ты~не~можешь найти для~неё слова?}

\questionE{3. Какой цвет, звук или образ связан с~твоим прошлым ранением?}


\cluster{Переосмысление личности}{15 – 30 мин}{кто~я~после всего пережитого}

\questionI{1. Кем бы~ты был, если бы~перестал соответствовать ожиданиям других людей?}

\questionI{2. Если бы~тебя никто не~знал и~ты мог начать с~чистого листа --- кем бы~ты решил быть?}

\questionI{3. Если бы~не было социальных ожиданий и~ограничений --- какую часть себя ты~бы~позволил проявиться свободнее?}

\questionI{4. Если бы~тебе разрешили быть абсолютно честным на~один день --- что~бы~ты сказал?}


\cluster{Видение себя}{10 – 20 мин}{какой я~на~самом деле}

\questionE{1. Чем ты~гордишься больше всего? Что~делает это~достижение особенно ценным для~тебя?}

\questionE{2. Какими своими достижениями ты~гордишься? Большими или маленькими --- всё важно.}

\questionE{3. Кто~тебя вдохновляет больше всего и~почему?}


\cluster{Выход из~прошлого}{10 – 20 мин}{как~отпустить и~идти}

\questionE{1. Что~ты чувствуешь прямо сейчас? Чего тебе больше всего хочется в~этот момент?}

\questionE{2. Чему в~своей жизни ты~хочешь дать больше места в~этом году?}

\questionE{3. Куда ты~можешь направить свою энергию, чтобы она принесла тебе и~окружающим наибольшую пользу?}

\questionE{4. Если завтра всё потеряет значение, какое было бы~твоё самое большое сожаление?}


\cluster{Действие и~будущее}{15 – 30 мин}{мои шаги после боли}

\questionI{1. Какой самый простой первый шаг ты~можешь сделать, чтобы интегрировать этот урок из~прошлого в~свою жизнь?}

\questionI{2. Кому ты~хотел бы~рассказать о~своём пути исцеления? Кто~эти люди, которым важно знать твою историю?}


\newpage

\chapter{Тело и~эмоции}
\programbreak

\cluster{Карта телесных ощущений}{5 – 10 мин}{что~тело мне говорит}

\questionF{1. Где в~теле ты~чувствуешь радость? Опиши ощущение: тепло, легкость, расширение.}

\questionF{2. Вспомни момент, когда тебя переполняла благодарность. Что~ты чувствовал в~теле в~этот момент?}

\questionF{3. Вспомни эмоцию, которая приносит тебе покой. Как~она ощущается в~теле?}

\questionF{4. Где в~твоём теле живёт самая глубокая эмоция, которую ты~не~можешь выразить словами?}


\cluster{Сигналы тела}{10 – 20 мин}{что~чувствует моё тело}

\questionE{1. Какие эмоции ты~чаще всего замечаешь у~себя в~течение дня?}

\questionE{2. Есть ли~эмоция, которая живет в~твоем теле постоянно, как~фоновый шум?}

\questionE{3. Что~происходит с~твоим телом, когда эмоции становятся очень сильными? Какие сигналы тебе подаёт организм?}

\questionE{4. Что~происходит с~твоим телом, когда ты~чувствуешь усталость или эмоциональное истощение? Какие сигналы подаёт тебе организм?}


\cluster{Эмоции и~маски}{10 – 20 мин}{мои настоящие чувства}

\questionE{1. Какие эмоции ты~чувствуешь наедине с~собой, но~не показываешь другим?}

\questionE{2. Ты~используешь юмор или сарказм, чтобы выразить чувства или чтобы их спрятать?}

\questionE{3. Когда кто-то спрашивает «что ты~чувствуешь?», твой первый импульс --- ответить честно или сказать то, что~ожидают?}

\questionE{4. Какую часть себя ты~прячешь в~теле, физически напрягая мышцы или замораживая?}


\cluster{Отношение к~чувствам}{10 – 20 мин}{что~я сейчас чувствую}

\questionE{1. Ты~позволяешь себе просто чувствовать то, что~есть, или часто думаешь «я должен это~чувствовать»?}

\questionE{2. Что~нежного и~доброго ты~можешь сказать своим эмоциям вместо критики?}

\questionE{3. Какое настроение ты~хотел бы~создавать у~себя по~утрам? Как~это может ощущаться в~теле?}


\cluster{Глубокие слои}{10 – 20 мин}{что~тело мне говорит}

\questionE{1. Если бы~твоя тревога была мудрым учителем, чему бы~она тебя учила?}

\questionE{2. О~чём тебе говорят чувства, которые ты~регулярно испытываешь?}

\questionE{3. Что~нежного и~доброго ты~можешь почувствовать к~себе прямо сейчас?}


\cluster{Регуляция через тело}{15 – 30 мин}{как~тело меня лечит}

\questionI{1. Как~ты можешь помочь своему телу справиться с~сильными эмоциями?}

\questionI{2. Какое движение или прикосновение помогает тебе вернуться в~баланс?}

\questionI{3. Замечаешь ли~ты, как~меняется твоё дыхание при~разных эмоциях? Что~происходит с~телом, когда ты~волнуешься или радуешься?}


\newpage

\chapter{Деньги и~самоценность}
\programbreak

\cluster{Происхождение убеждений}{5 – 10 мин}{кто~на~меня влиял}

\questionF{1. Какие финансовые уроки ты~усвоил от~родителей? И~хорошие, и~не очень полезные --- всё важно для~понимания.}

\questionF{2. Какие убеждения о~деньгах ты~унаследовал от~семьи и~окружения? Как~они сейчас влияют на~твою жизнь?}

\questionF{3. Как~твои родители относились к~деньгам? Что~из их подхода ты~замечаешь в~себе сейчас?}


\cluster{Текущие убеждения}{5 – 10 мин}{мои правила про~деньги}

\questionF{1. Какие убеждения о~деньгах живут в~тебе прямо сейчас?}

\questionF{2. Какие убеждения о~деньгах у~тебя есть сейчас, и~как они влияют на~твоё финансовое положение?}

\questionF{3. Как~ты относишься к~деньгам --- видишь в~них больше возможностей или ограничений?}

\questionF{4. Как~ты ведёшь себя с~деньгами, когда испытываешь стресс? Что~это может рассказать о~твоих убеждениях?}


\cluster{Отношение к~деньгам и~самоценность}{10 – 20 мин}{мои деньги и~я}

\questionE{1. Что~влияет на~твоё ощущение собственной достойности финансового благополучия?}

\questionE{2. Есть ли~в тебе убеждение «я недостаточно хорош для~денег»? Откуда оно?}

\questionE{3. Что~бы~ты делал в~жизни, если бы~деньги перестали быть ограничением?}


\cluster{Финансовые страхи}{10 – 20 мин}{чего боюсь с~деньгами}

\questionE{1. Какие финансовые страхи не~дают тебе спать спокойно?}

\questionE{2. Чего ты~боишься больше --- финансовой неудачи или успеха, который изменит всё?}

\questionE{3. Если бы~деньги исчезли --- что~ты потеряешь помимо самих денег?}


\cluster{Текущее положение}{5 – 10 мин}{мои деньги сегодня}

\questionF{1. Как~бы~ты описал своё текущее финансовое положение?}

\questionF{2. Куда конкретно ты~сейчас инвестируешь свои деньги: в~будущее (образование, активы, инвестиции) или в~настоящее (комфорт, развлечения, статусные покупки)?}

\questionF{3. Какую часть своего дохода ты~выделяешь на~собственное развитие?}


\cluster{Мечты и~мотивация}{10 – 20 мин}{о~чём мечтаю купить}

\questionE{1. О~чём ты~мечтаешь, когда представляешь финансовую свободу?}

\questionE{2. Что~бы~ты делал(-а), если бы~денег у~тебя было достаточно для~любых планов?}

\questionE{3. Если бы~у тебя были все~необходимые ресурсы (время, деньги, возможности) --- чем бы~ты хотел заниматься?}


\cluster{Цели и~стратегия}{10 – 20 мин}{куда хочу прийти}

\questionE{1. Какую конкретную финансовую цель ты~хочешь достичь через год?}

\questionE{2. Какие финансовые цели ты~хотел бы~достичь через 5 лет?}

\questionE{3. Кто~может помочь тебе в~финансовом развитии? Наставник, книга, курс?}

\questionE{4. Какой первый конкретный шаг ты~можешь сделать уже на~этой неделе, чтобы приблизиться к~своим финансовым целям?}


\cluster{Успех и~трансформация}{15 – 30 мин}{куда меня ведет успех}

\questionI{1. Чем бы~ты занимался, если бы~знал, что~в любом деле тебя ждёт успех?}

\questionI{2. Как~изменится твоя жизнь, если ты~достигнешь финансовой свободы?}


\newpage

\chapter{Кризис 30/40/50}
\programbreak

\cluster{Ожидания vs реальность}{5 – 10 мин}{моя правда о~прошлом}

\questionF{1. Кем ты~представлял себя в~этом возрасте, когда был молодым?}

\questionF{2. Кто~ты~на~самом деле сейчас? Совпадает ли~это с~ожиданиями?}

\questionF{3. Что~произошло по-другому, чем ты~планировал?}

\questionF{4. За~что~ты благодарен, что~жизнь пошла не~так, как~ты планировал?}


\cluster{Мудрость из~ошибок}{10 – 20 мин}{чему меня научила жизнь}

\questionE{1. Какая мудрость есть в~твоих ошибках?}

\questionE{2. Какие 3 главных жизненных урока ты~бы~передал тем, кто~идёт следом за~тобой?}

\questionE{3. Если бы~у тебя была возможность что-то изменить в~своём прошлом, что~бы~это было?}

\questionE{4. Чему ты~благодарен своему молодому «я» за~то, что~оно сделало?}


\cluster{Текущий кризис}{5 – 10 мин}{что~меня не~устраивает}

\questionF{1. Что~в твоей жизни сейчас вызывает кризис или ощущение застоя?}

\questionF{2. Потерял ли~ты смысл в~том, что~раньше питало тебя?}

\questionF{3. С~какими переменами в~жизни ты~сейчас имеешь дело --- в~отношениях, работе, взглядах на~себя или планах на~будущее?}


\cluster{Переоценка ценностей}{10 – 20 мин}{что~стало важнее всего}

\questionE{1. Какие ценности тебе казались важными раньше, но~потеряли значение?}

\questionE{2. Какие ценности стали для~тебя важнее, чем были раньше?}

\questionE{3. Если бы~ты мог прожить следующие 10 лет по-другому --- что~бы~изменил в~своих приоритетах и~почему?}


\cluster{Возможность изменения}{10 – 20 мин}{что~в моих силах}

\questionE{1. Есть ли~аспекты твоей жизни, которые ты~хочешь кардинально изменить?}

\questionE{2. Что~мешает тебе изменить то, что~хочется изменить?}

\questionE{3. Чем ты~был бы~готов пожертвовать, чтобы обновить свою жизнь?}

\questionE{4. Если бы~ты знал, что~всё получится --- какие перемены ты~бы~осмелился сделать в~своей жизни?}


\cluster{Новый смысл}{15 – 30 мин}{кто~я~теперь}

\questionI{1. Каким ты~видишь себя через 10-15 лет? Представь и~опиши один обычный день из~этой будущей жизни}

\questionI{2. В~чём ты~хочешь себя переизобрести?}

\questionI{3. Какой новый смысл жизни открывается перед тобой после кризиса?}


\cluster{Действие и~переход}{15 – 30 мин}{мой путь из~кризиса}

\questionI{1. Какой один конкретный шаг ты~можешь сделать на~этой неделе, чтобы начать переход?}

\questionI{2. Кому ты~расскажешь о~своём кризисе и~поищешь поддержки?}


\newpage

\chapter{AI-тревожность и~будущее работы}
\programbreak

\cluster{Моя ценность}{5 – 10 мин}{что~я умею делать}

\questionF{1. Какие профессиональные навыки делают тебя незаменимым на~текущей работе?}

\questionF{2. Что~в твоей работе невозможно заменить машиной или AI?}

\questionF{3. Твоя эмоциональная интеллигентность --- это~твой главный профессиональный актив? Как?}

\questionF{4. Чем ты~занимаешься сверх должностной инструкции, что~ценят твои коллеги?}


\cluster{Профессиональная идентичность}{10 – 20 мин}{кем работаю и~горжусь}

\questionE{1. Кто~ты~профессионально помимо технических навыков? Что~определяет тебя как~специалиста?}

\questionE{2. Чем ты~отличаешься от~других людей на~твоей позиции?}

\questionE{3. Какие твои навыки или качества могли бы~стать особенно ценными в~мире, где технологии меняют профессиональную среду?}


\cluster{Признаки тревоги}{10 – 20 мин}{мне нравится работа?}

\questionE{1. Что~тебя больше беспокоит: возможность потерять работу или стать профессионально неактуальным?}

\questionE{2. Как~часто ты~испытываешь профессиональное беспокойство по~поводу AI и~технологий?}

\questionE{3. Какие ситуации в~работе усиливают твой страх несоответствия?}

\questionE{4. Какими способами ты~обычно справляешься со страхом профессиональной некомпетентности?}


\cluster{Переобучение и~развитие}{10 – 20 мин}{как~я учусь новому}

\questionE{1. Сколько времени в~неделю ты~тратишь на~переобучение и~развитие?}

\questionE{2. Какие новые навыки или качества ты~хочешь развить в~ближайший год, чтобы чувствовать себя увереннее в~меняющемся мире?}

\questionE{3. Есть ли~у тебя план профессионального развития на~случай, если твоя текущая роль автоматизируется?}

\questionE{4. Какой человек, наставник или сообщество может помочь тебе развиваться в~работе с~новыми технологиями?}


\cluster{Переосмысление работы}{15 – 30 мин}{как~я вижу работу}

\questionI{1. Если бы~ИИ мог выполнять твою работу, чем бы~ты хотел заниматься? Какую уникальную ценность ты~можешь привнести в~мир?}

\questionI{2. Если бы~тебе пришлось начинать карьеру с~нуля прямо сейчас --- что~бы~ты изучал?}

\questionI{3. Какие свои профессиональные качества или опыт ты~предпочитаешь не~выставлять напоказ? Что~остаётся за~кадром твоего публичного образа?}


\cluster{Рынок труда и~возможности}{10 – 20 мин}{новые возможности для~меня}

\questionE{1. Какие новые профессиональные ниши открываются благодаря AI и~автоматизации?}

\questionE{2. Как~ты можешь использовать AI как~союзника, чтобы развить своё мастерство в~профессии?}

\questionE{3. Кто~из~твоих коллег или людей в~профессии уже успешно адаптировался к~новым технологиям? Чему ты~можешь у~них научиться?}


\cluster{Действие и~стратегия}{15 – 30 мин}{мой план действий}

\questionI{1. Какой первый конкретный шаг ты~сделаешь на~этой неделе, чтобы лучше подготовиться к~изменениям в~мире работы?}

\questionI{2. Какой новый навык или знание ты~хочешь приобрести в~ближайшие 3 месяца?}

\questionI{3. Как~ты будешь отслеживать прогресс в~адаптации к~профессиональным изменениям?}


\newpage

\chapter{Инфо-ожирение}
\programbreak

\cluster{Осознание потребления}{5 – 10 мин}{куда уходит мое время}

\questionF{1. Сколько времени в~день ты~проводишь в~цифровых потоках (социальные сети, новости, мессенджеры)?}

\questionF{2. Какие три приложения первыми открываются утром?}

\questionF{3. Когда последний раз~ты проводил день без~уведомлений --- полностью отключив их или включив режим «не беспокоить»?}

\questionF{4. Открываешь ли~ты приложения с~осознанной целью или просто «по привычке»?}


\cluster{Чувства и~сигналы}{10 – 20 мин}{как~я себя чувствую}

\questionE{1. Что~происходит с~твоим телом, когда экран переполняют уведомления? Какие сигналы подаёт организм?}

\questionE{2. Какие эмоции возникают после часа в~информационном потоке: спокойствие, растерянность, перегруз?}

\questionE{3. Когда ты~чувствуешь потребность «проверить что-то» в~телефоне или компьютере --- что~на самом деле тебе нужно в~этот момент?}


\cluster{Страхи и~мотивы}{10 – 20 мин}{чего боюсь и~хочу}

\questionE{1. Чего ты~боишься пропустить в~информационном потоке?}

\questionE{2. Если бы~ты перестал проверять новости неделю --- что~бы~ты потерял? Что~бы~приобрел?}

\questionE{3. Как~цифровое потребление помогает тебе избежать чего-то (скуки, одиночества, размышлений)?}


\cluster{Граница и~сатурация}{10 – 20 мин}{когда хватит информации}

\questionE{1. Где проходит для~тебя граница между полезной информированностью и~информационной перегрузкой?}

\questionE{2. Сколько информации в~день твоему мозгу действительно нужно? Как~ты понимаешь, что~уже достаточно?}

\questionE{3. Какой контент действительно полезен тебе, а~какой ты~потребляешь на~автопилоте?}


\cluster{От~потребления к~созданию}{10 – 20 мин}{как~стать творцом}

\questionE{1. Что~случится, если ты~перестанешь только потреблять, а~начнешь создавать?}

\questionE{2. Какой контент или идею ты~бы~хотел создавать вместо бесконечного потребления?}

\questionE{3. Как~часто ты~пишешь, рисуешь, создаёшь что-то своё против времени, которое тратишь на~потребление?}


\cluster{Идентичность без~цифры}{10 – 20 мин}{мое настоящее лицо}

\questionE{1. Каким ты~видишь себя без~цифрового следа, оценок и~лайков?}

\questionE{2. Какие части твоей личности ты~показываешь онлайн, а~какие скрываешь?}

\questionE{3. Если бы~завтра все~твои цифровые аккаунты исчезли --- что~бы~изменилось в~том, кто~ты~есть?}


\cluster{Независимость}{10 – 20 мин}{кто~я~без~интернета}

\questionE{1. Можешь ли~ты провести день без~смартфона? Что~ты чувствуешь в~этом случае?}

\questionE{2. Когда ты~последний раз~размышлял о~чём-то глубоко, полагаясь только на~свои мысли?}

\questionE{3. Какое решение ты~принял в~последний месяц, опираясь только на~собственный опыт, без~консультации «интернета»?}


\cluster{Действие и~баланс}{15 – 30 мин}{мои первые шаги}

\questionI{1. Какой один конкретный цифровой лимит ты~установишь для~себя на~этой неделе?}

\questionI{2. Какое приложение или привычку ты~готов удалить или ограничить?}

\questionI{3. Какую деятельность offline ты~хочешь вернуть в~свою жизнь?}

\questionI{4. Как~ты будешь отслеживать прогресс в~создании цифрового баланса?}


\newpage

\chapter{Выученная беспомощность 2.0}
\programbreak

\cluster{Где я~теряю силу}{5 – 10 мин}{что~забирает мою энергию}

\questionF{1. Что~тебя парализует больше всего в~моменты, когда нужно действовать?}

\questionF{2. Где заканчивается твой реальный круг влияния? Что~находится вне твоего контроля?}

\questionF{3. Какие ситуации в~жизни вызывают у~тебя чувство бессилия или беспомощности?}

\questionF{4. Что~происходит с~твоим телом, когда ты~чувствуешь бессилие? Какие сигналы подаёт организм?}


\cluster{Информационная ловушка}{10 – 20 мин}{когда я~перестаю пытаться}

\questionE{1. Какие источники новостей и~информации формируют твою картину мира?}

\questionE{2. Сколько времени в~день ты~тратишь на~потребление информации, которая тебя парализует?}

\questionE{3. Чьи взгляды и~убеждения ты~принимаешь как~свои собственные, не~проверяя их на~соответствие твоему опыту?}


\cluster{Страхи и~выбор}{10 – 20 мин}{что~меня останавливает}

\questionE{1. Чего ты~боишься больше: неудачи или бездействия?}

\questionE{2. Какой внутренний голос говорит тебе, что~ты не~можешь что-то сделать? Откуда, как~ты думаешь, появились эти убеждения?}

\questionE{3. От~какой надежды ты~уже отказался? Хочешь ли~ты её вернуть?}


\cluster{Иллюзия контроля}{10 – 20 мин}{что~хочу держать в~руках}

\questionE{1. Что~будет, если ты~перестанешь пытаться контролировать то, что~находится вне твоего контроля?}

\questionE{2. В~каких областях жизни ты~чувствуешь, что~усердно стараешься что-то изменить, но~результат остаётся прежним?}

\questionE{3. Если бы~ты сосредоточился только на~том, что~действительно в~твоей власти --- как~бы~это изменило твоё отношение к~жизни?}


\cluster{Части личности и~сопротивление}{10 – 20 мин}{кто~внутри меня боится}

\questionE{1. Какая часть тебя боится действовать? Что~она защищает?}

\questionE{2. Что~ты получаешь, оставаясь в~беспомощности? Какая выгода в~бездействии?}

\questionE{3. Если бы~одна часть тебя боялась, а~другая была готова действовать --- что~бы~сказала готовая часть?}


\cluster{Компетентность и~история успеха}{10 – 20 мин}{что~у меня получается}

\questionE{1. Когда последний раз~ты почувствовал себя компетентным и~способным что-то изменить?}

\questionE{2. Какие маленькие действия или привычки уже помогли тебе добиться результата?}

\questionE{3. В~какой сфере жизни ты~чувствуешь себя наиболее влиятельным и~способным?}


\cluster{От~беспомощности к~агентности}{10 – 20 мин}{куда меня ведет успех}

\questionE{1. Кто~ты, когда ты~не~парализован беспомощностью?}

\questionE{2. Какие решения ты~можешь принять сегодня, чтобы почувствовать больше контроля над~своей жизнью?}

\questionE{3. Если бы~ты взял полную ответственность за~те части жизни, которые в~твоих руках --- как~это могло бы~изменить твои решения и~действия?}


\cluster{Действие}{15 – 30 мин}{первые шаги к~переменам}

\questionI{1. Какой один конкретный шаг ты~можешь сделать сегодня, чтобы почувствовать себё увереннее в~своих силах?}

\questionI{2. От~какого привычного беспокойства ты~готов отказаться на~этой неделе?}

\questionI{3. Какую новую привычку ты~введёшь, чтобы укрепить своё чувство компетентности?}


\newpage

\chapter{Паразоциальная зависимость}
\programbreak

\cluster{Картография моего мира}{5 – 10 мин}{мои реальные отношения}

\questionF{1. Сколько люди в~твоей жизни знают тебя по-настоящему, без~фильтров?}

\questionF{2. Сколько часов в~неделю ты~проводишь с~живыми людьми? Это~достаточно?}

\questionF{3. Кого из~людей, с~которыми ты~знаком только виртуально, хотелось бы~встретить в~реальной жизни?}

\questionF{4. Какой контент помогает тебе почувствовать себя менее одиноко?}


\cluster{Источники близости}{10 – 20 мин}{кто~меня поддерживает}

\questionE{1. От~кого из~реальных людей ты~чувствуешь максимальную эмоциональную поддержку?}

\questionE{2. От~какого блогера, персонажа или AI ты~чувствуешь поддержку? Почему именно они?}

\questionE{3. Какие потребности в~близости и~понимании удовлетворяет для~тебя общение с~виртуальными собеседниками?}

\questionE{4. Какую настоящую потребность закрывает для~тебя виртуальный контент?}


\cluster{Побег и~избегание}{10 – 20 мин}{что~я избегаю в~жизни}

\questionE{1. От~чего ты~убегаешь, когда проваливаешься в~бесконечный скролл?}

\questionE{2. Чем заменяет для~тебя виртуальность живое человеческое прикосновение и~близость?}

\questionE{3. Какие реальные люди или ситуации ты~избегаешь через экран?}

\questionE{4. Если бы~ты на~неделю отключил все~виртуальные отношения --- что~бы~тебе не~хватало?}


\cluster{Парадокс близости}{10 – 20 мин}{когда экран ближе людей}

\questionE{1. Почему виртуальные отношения кажутся тебе проще, чем реальные?}

\questionE{2. Какие преимущества имеют виртуальные отношения для~тебя?}

\questionE{3. Какие риски и~сложности есть в~виртуальной близости?}


\cluster{Подлинное «я»}{10 – 20 мин}{мое настоящее лицо}

\questionE{1. Когда последний раз~ты чувствовал настоящую близость --- без~экрана, без~фильтров?}

\questionE{2. Кто~ты, когда никто не~видит твоих постов, лайков и~онлайн-статусов?}

\questionE{3. Какие грани себя ты~показываешь онлайн, но~скрываешь в~живом общении?}

\questionE{4. Если бы~завтра все~твои виртуальные отношения исчезли --- изменилось бы~что-то в~том, кто~ты~есть?}


\cluster{Восстановление подлинности}{10 – 20 мин}{возвращаюсь к~себе настоящему}

\questionE{1. Что~нужно изменить в~твоей жизни, чтобы иметь более подлинные отношения?}

\questionE{2. Какую реальную потребность ты~можешь начать закрывать через живые отношения?}

\questionE{3. Кому из~близких тебе людей ты~бы~хотел рассказать то, что~обычно делишься только с~виртуальными собеседниками?}

\questionE{4. Какой один шаг ты~можешь сделать сегодня, чтобы восстановить настоящую близость с~кем-то важным для~тебя?}


\cluster{Границы и~баланс}{15 – 30 мин}{как~не потеряться онлайн}

\questionI{1. Какие виртуальные отношения или контент можно оставить, потому что~они тебе действительно помогают?}

\questionI{2. Как~ты можешь наслаждаться интернетом без~потери подлинных связей?}

\questionI{3. Какие границы между онлайн и~офлайн-миром ты~хочешь для~себя установить?}


\newpage

\chapter{Гибридная жизнь}
\programbreak

\cluster{Трансформация пространства}{5 – 10 мин}{мой мир до~и~после}

\questionF{1. Как~изменилось твоё отношение к~дому после перехода на~гибридный режим?}

\questionF{2. Что~появилось в~твоей жизни благодаря удалённой работе?}

\questionF{3. Что~ты потерял, когда офис стал твоей спальней?}

\questionF{4. Как~изменилось твоё личное пространство и~энергия после тотальной цифровизации?}


\cluster{Усталость и~триггеры}{10 – 20 мин}{что~крадёт мои силы}

\questionE{1. Сколько виртуальных встреч в~неделю начинают истощать твои внутренние ресурсы?}

\questionE{2. Что~происходит с~твоим телом в~моменты усталости? Какие сигналы оно тебе подаёт?}

\questionE{3. Какие цифровые привычки стали для~тебя невидимыми оковами?}

\questionE{4. Как~твоё тело и~разум сигнализируют о~том, что~ты переутомился?}


\cluster{Жертвы и~приоритеты}{10 – 20 мин}{от~чего я~отказываюсь}

\questionE{1. От~чего ты~отказываешься, проводя всё больше времени в~цифровом мире?}

\questionE{2. Какие важные вещи в~жизни ты~отодвигаешь из-за работы?}

\questionE{3. Сколько энергии ты~тратишь на~поддержание профессиональной маски в~разных ситуациях?}


\cluster{Границы}{10 – 20 мин}{когда хватит информации}

\questionE{1. Где проходят твои границы между работой и~личной жизнью сейчас?}

\questionE{2. Что~произойдёт, если ты~отключишь уведомления после 18:00?}

\questionE{3. Какой ритуал или обряд обозначает переход от~работы к~личной жизни?}

\questionE{4. Как~твои близкие люди относятся к~тому, что~ты всегда онлайн?}


\cluster{Разделение ролей}{10 – 20 мин}{что~моё, а~что работа}

\questionE{1. Где заканчивается твоя профессиональная роль и~начинается личность?}

\questionE{2. Как~ты переходишь от~«работника» к~«человеку» дома?}

\questionE{3. Какие части себя ты~не~показываешь на~работе?}


\cluster{Идентичность и~цифровой след}{10 – 20 мин}{мой образ онлайн и~офлайн}

\questionE{1. Кто~ты, когда отключаешь все~уведомления и~убираешь телефон подальше?}

\questionE{2. Какую версию себя ты~создаёшь в~онлайн-мире, и~что остаётся за~кадром?}

\questionE{3. Если бы~завтра ты~потерял интернет на~неделю --- кем бы~ты себя обнаружил?}


\cluster{Возможности гибридности}{10 – 20 мин}{мои плюсы сегодня}

\questionE{1. Какие преимущества гибридной жизни ты~уже ценишь?}

\questionE{2. Как~ты можешь лучше использовать возможности гибридного режима для~своего развития?}

\questionE{3. Какая свобода появилась благодаря удалённой работе?}


\cluster{Действие и~интеграция}{15 – 30 мин}{как~я всё соберу}

\questionI{1. Какую одну границу ты~установишь на~этой неделе между работой и~жизнью?}

\questionI{2. Какой ритуал «отключения» от~работы ты~введешь в~свой день?}

\questionI{3. Как~ты поймёшь, что~баланс восстановился?}


\newpage

\chapter{Аутентичность vs Алгоритмы}
\programbreak

\cluster{Картография масок}{5 – 10 мин}{мои разные лица}

\questionF{1. Какие разные версии себя ты~создаёшь в~разных контекстах --- соцсетях, работе, семье, дружеских компаниях?}

\questionF{2. Какие платформы ты~используешь? Одинаковый ли~ты везде?}

\questionF{3. Сколько раз~в день ты~редактируешь, фильтруешь или скрываешь свой образ?}

\questionF{4. Что~бы~изменилось в~твоей жизни, если бы~ты больше не~старался подстраивать свой образ под~ожидания других?}


\cluster{Контент и~искренность}{10 – 20 мин}{мои посты и~я}

\questionE{1. Какой контент ты~создаёшь от~скуки, давления или желания нравиться?}

\questionE{2. Какой контент ты~создаёшь искренне, от~сердца?}

\questionE{3. Чем твоя личность в~интернете отличается от~того, кто~ты~в~обычной жизни?}


\cluster{Риски аутентичности}{10 – 20 мин}{чего боюсь в~себе}

\questionE{1. Какие части твоей личности становятся невидимыми в~цифровом пространстве?}

\questionE{2. Какую часть себя ты~скрываешь, опасаясь осуждения?}

\questionE{3. Что~для тебя рискованного в~том, чтобы показывать свои настоящие эмоции онлайн?}

\questionE{4. Какие люди или ситуации заставляют тебя носить маску?}


\cluster{Граница между презентацией и~цензурой}{10 – 20 мин}{когда хватит информации}

\questionE{1. Где граница между здоровой самопрезентацией и~самоцензурой для~тебя?}

\questionE{2. Что~из твоей жизни ты~предпочитаешь держать только для~себя? Что~влияет на~это решение?}

\questionE{3. Есть ли~у тебя люди или сообщества, с~которыми ты~можешь быть полностью собой?}


\cluster{Выбор между принятием и~аутентичностью}{10 – 20 мин}{быть собой или нравиться}

\questionE{1. Когда тебе приходится выбирать между аутентичностью и~принятием --- как~ты выбираешь?}

\questionE{2. Какой контент бы~ты создавал, если бы~не было лайков и~оценок?}

\questionE{3. Если бы~ты был совершенно анонимен онлайн --- кем бы~ты была/был?}


\cluster{Подлинность в~эпоху алгоритмов}{10 – 20 мин}{мое настоящее лицо}

\questionE{1. Где твоя подлинность сейчас --- онлайн или офлайн?}

\questionE{2. Какие части тебя чувствуют себя «настоящими» и~не требуют фильтра?}

\questionE{3. Если бы~твоя онлайн-версия и~реальная версия встретились --- что~бы~они сказали друг другу?}


\cluster{Восстановление аутентичности}{10 – 20 мин}{как~найти себя настоящего}

\questionE{1. Какую грань себя ты~готов показать более открыто?}

\questionE{2. С~кем бы~ты хотел поделиться своей подлинной версией? Кто~эти люди?}

\questionE{3. Какой один «настоящий» контент ты~бы~создал, если бы~не боялся?}


\cluster{Действие и~баланс}{15 – 30 мин}{как~быть собой онлайн}

\questionI{1. Какую правду о~себе ты~готов принять прямо сейчас?}

\questionI{2. Как~ты будешь измерять свою аутентичность? По~лайкам или по~собственному ощущению?}

\questionI{3. Какой ритуал «цифрового детокса» ты~могла бы~ввести в~свою жизнь?}


\newpage

\chapter{Эко-вина и~климат-тревога}
\programbreak

\cluster{Моё потребление}{5 – 10 мин}{мои покупки и~привычки}

\questionF{1. Какие товары и~услуги ты~потребляешь ежедневно? Задумываешься ли~об их источнике?}

\questionF{2. Какие потребительские привычки ты~унаследовал от~родителей?}

\questionF{3. Какие экологичные привычки у~тебя уже есть? Что~хотелось бы~добавить или изменить в~своём потреблении?}

\questionF{4. Сколько «экологической стоимости» имеет твой типичный день?}


\cluster{Вина и~ответственность}{10 – 20 мин}{мне нравится работа?}

\questionE{1. Сколько энергии ты~тратишь на~экологическую вину вместо действия?}

\questionE{2. Что~тебя пугает больше: собственный дискомфорт от~изменений в~образе жизни или последствия климатического кризиса?}

\questionE{3. Где заканчивается твоя личная ответственность и~начинается системная?}

\questionE{4. Как~часто ты~ловишь себя на~парализующей вине вместо действия?}


\cluster{Эмоции и~отрицание}{10 – 20 мин}{что~скрываю от~себя}

\questionE{1. Какую боль о~природе, о~её страдании ты~прячешь внутри себя?}

\questionE{2. Когда ты~последний раз~позволил себе почувствовать экологическую тревогу без~осуждения?}

\questionE{3. Что~ты рационализируешь или отрицаешь, когда дело касается экологии?}


\cluster{Личный выбор}{10 – 20 мин}{что~в моих руках}

\questionE{1. Какой один личный выбор ты~можешь сделать, который имеет реальное значение?}

\questionE{2. Какие малые действия ты~уже делаешь? Это~ценно?}

\questionE{3. Если бы~ты знал, что~твои действия точно помогут планете --- что~бы~ты изменил в~своей жизни?}


\cluster{Идентичность и~система}{10 – 20 мин}{моё место в~мире}

\questionE{1. Кто~ты~помимо своих потребительских привычек?}

\questionE{2. Как~система (корпорации, государство, экономика) влияет на~твой выбор?}

\questionE{3. Можешь ли~ты быть экологичным внутри неэкологичной системы? Как?}


\cluster{От~вины к~действию}{10 – 20 мин}{что~делаю для~планеты}

\questionE{1. От~какой экологической вины ты~готова отпустить себя?}

\questionE{2. Какое одно небольшое действие ты~можешь добавить в~свою жизнь уже сегодня, чтобы почувствовать себя более целостным в~вопросах экологии?}

\questionE{3. Кто~или что~может помочь тебе в~экологических выборах?}

\questionE{4. Как~ты будешь отслеживать прогресс? По~материалам или по~внутреннему ощущению?}


\cluster{Активизм и~согласованность}{15 – 30 мин}{что~я делаю для~планеты}

\questionI{1. Есть ли~экологическая причина, которая тебя вдохновляет? Почему?}

\questionI{2. Как~ты хочешь участвовать: личными действиями, информированием других, активизмом?}

\questionI{3. Какую роль ты~хочешь играть в~создании более здорового мира?}

\questionI{4. Какими своими достижениями ты~гордишься? Что~в твоей жизни наполняет тебя чувством гордости?}

\questionI{5. Какими своими достижениями ты~гордишься больше всего?}

\questionI{6. Как~ты можешь получать ещё больше удовольствия в~процессе достижения своих экологических целей?}

\questionI{7. Что~в тебе ценно независимо от~твоих экологических действий и~достижений?}


\newpage

\chapter{Синдром самозванца в~эпоху LinkedIn}
\programbreak

\cluster{Симптомы и~триггеры}{5 – 10 мин}{когда накрывает сомнение}

\questionF{1. В~какие моменты ты~чувствуешь, что~недостаточно компетентен для~своей роли?}

\questionF{2. Что~происходит с~твоей самооценкой, когда ты~читаешь «истории успеха» коллег в~соцсетях?}

\questionF{3. Боишься ли~ты, что~однажды кто-то постучит по~плечу и~скажет: «Ты здесь случайно, уходи»?}


\cluster{Присвоение заслуг}{10 – 20 мин}{мои успехи или везение}

\questionE{1. Как~часто ты~списываешь свои успехи на~«просто повезло», «совпадение» или «помощь команды»?}

\questionE{2. Выпиши 3 факта о~своей карьере, которые невозможно оспорить (цифры, завершенные проекты).}

\questionE{3. Какими своими достижениями ты~гордишься по-настоящему, даже если никто не~поставил лайк?}


\cluster{Разрыв с~реальностью}{10 – 20 мин}{моя маска в~соцсетях}

\questionE{1. Если сравнить твой публичный профиль (резюме) и~твое внутреннее ощущение --- насколько велик разрыв?}

\questionE{2. Какую свою профессиональную неуверенность ты~предпочитаешь не~обсуждать с~коллегами?}

\questionE{3. Что~самое страшное может произойти, если ты~скажешь: «Я не~знаю ответа на~этот вопрос»?}


\cluster{Ценность и~личность}{10 – 20 мин}{что~для меня важно}

\questionE{1. Как~ты чаще всего думаешь о~себе --- через свои достижения («я тот, кто~сделал это») или через свои качества («я тот, кто~такой-то»)?}

\questionE{2. Кто~ты~без~своих должностей, проектов и~карьерных статусов?}

\questionE{3. Как~ты можешь находить удовольствие в~самом процессе работы над~целями, а~не только в~моменте их достижения?}


\cluster{Выход из~гонки}{15 – 30 мин}{как~жить без~сравнений}

\questionI{1. Чьё мнение или одобрение до~сих пор кажется тебе необходимым, чтобы почувствовать себя настоящим профессионалом?}

\questionI{2. Какой совет ты~бы~дал другу, если бы~он чувствовал себя так~же~неуверенно, как~ты сейчас?}

\questionI{3. Что~ты готов сделать уже сегодня, зная, что~ты достаточно хорош для~этого?}

\questionI{4. В~какой сфере ты~работаешь или учишься?}

\questionI{5. Почему я~верю, что~это сработает? На~чём основана моя уверенность в~этом пути? Возможно, есть более эффективный способ --- как~я могу это~выяснить?}

\questionI{6. Если бы~ты мог выбирать без~ограничений --- куда бы~тебе хотелось пойти работать?}

\questionI{7. Что~для тебя сейчас кажется более болезненным: получать отказы на~собеседованиях или продолжать работать там, где тебя не~ценят по~достоинству?}

\questionI{8. Критикуешь ли~ты себя за~отдых? И~при этом у~тебя много работы и~обязанностей?}

\questionI{9. На~сколько времени хватит твоих сбережений, если ты~решишь сделать паузу в~работе?}

\questionI{10. Как~твоя нынешняя работа помогает тебе достичь того, что~действительно важно для~тебя?}

\questionI{11. Что~в твоей нынешней работе не~совпадает с~важными для~тебя целями?}

\questionI{12. Насколько твоя текущая работа приносит тебе удовлетворение? Оцени по~шкале от~1 до~10.}

\questionI{13. Какой была бы~твоя работа мечты?}


\newpage

\chapter{Воскресная тревога}
\programbreak

\cluster{Заземление}{5 – 10 мин}{что~чувствует моё тело}

\questionF{1. Как~бы~ты описал своё состояние прямо сейчас тремя словами? (например: усталость, беспокойство, ожидание)}

\questionF{2. Когда отдых перестаёт быть отдыхом? Замечал ли~ты тот момент, когда расслабление сменяется внутренним ожиданием понедельника?}

\questionF{3. Что~ты чувствуешь по~поводу того, как~провёл эти выходные?}

\questionF{4. Если бы~завтра не~был понедельником, как~бы~ты провёл этот вечер?}


\cluster{Анализ монстров}{10 – 20 мин}{мои главные страхи}

\questionE{1. Какое одно конкретное событие или задача предстоящей недели вызывает у~тебя наибольшее напряжение?}

\questionE{2. Это~напряжение --- про~страх не~справиться, про~скуку или про~неприятных людей?}

\questionE{3. Насколько по~шкале от~1 до~10 тебя удовлетворяет твоя работа в~целом?}

\questionE{4. Есть ли~реальная угроза в~том, что~тебя беспокоит, или это~привычка тревожиться?}


\cluster{Границы и~ресурсы}{10 – 20 мин}{что~мне поможет}

\questionE{1. Что~самое доброе и~заботливое ты~можешь сделать для~себя прямо сейчас?}

\questionE{2. Какой маленький ритуал завтра утром поможет тебе начать день мягче?}

\questionE{3. Кто~или что~может стать твоей поддержкой на~этой неделе?}


\cluster{Глобальная сверка}{15 – 30 мин}{когда тревога не~отпускает}

\questionI{1. Если поставить на~весы страх перемен и~боль от~работы там, где тебя не~ценят --- что~сейчас ощущается тяжелее?}

\questionI{2. Работа твоей мечты --- в~ней есть место такому чувству по~воскресеньям?}

\questionI{3. Какой один шаг ты~сделаешь на~неделе, чтобы немного приблизить реальность к~этой мечте?}

\questionI{4. Сколько часов в~день твой ребенок проводит перед экраном?}

\questionI{5. Какие приложения чаще всего выбирает твой ребенок?}

\questionI{6. Что~ты чувствуешь, когда видишь ребенка с~гаджетом?}

\questionI{7. Какие эмоции скрываются за~твоим запретом на~гаджеты?}

\questionI{8. Как~твой страх технологий влияет на~отношения с~ребенком?}

\questionI{9. Что~на самом деле вызывает тревогу, когда ты~думаешь об~ограничении экранного времени?}

\questionI{10. Какую травму собственного детства ты~проецируешь на~ребенка?}

\questionI{11. Кем ты~станешь как~родитель, приняв цифровой мир ребенка?}


\newpage

\chapter{Родительская вина за~экранное время}
\programbreak

\cluster{Честная диагностика}{5 – 10 мин}{что~происходит на~самом деле}

\questionF{1. Сколько часов в~день твой ребенок проводит перед экраном, и~сколько из~них --- для~твоего отдыха?}

\questionF{2. Знаешь ли~ты сюжет любимой игры или мультика своего ребенка?}

\questionF{3. Сколько времени в~телефоне проводишь ты~сам, когда ребенок находится в~одной комнате с~тобой?}


\cluster{Анатомия вины}{10 – 20 мин}{что~со мной происходит}

\questionE{1. Что~ты чувствуешь в~тот момент, когда вручаешь ребенку планшет: облегчение, раздражение или стыд?}

\questionE{2. Какой страх живёт в~тебе глубже всего, когда ты~думаешь об~экранном времени ребёнка?}

\questionE{3. Есть ли~у тебя убеждение, что~«хороший родитель» должен развлекать ребенка 24/7 без~помощи техники?}


\cluster{Скрытые смыслы}{15 – 30 мин}{что~скрывается за~виной}

\questionI{1. Когда ты~запрещаешь гаджет или сердишься из-за него --- какую свою тревогу ты~пытаешься успокоить?}

\questionI{2. Если вспомнить твое детство: чего тебе не~хватало, что~ты теперь пытаешься компенсировать (или запретить) ребенку?}

\questionI{3. Что~стоит за~твоими спорами о~телефоне с~ребёнком? Возможно, это~способ наладить контакт, когда других поводов для~разговора не~находится?}


\cluster{Цифровой мост}{15 – 30 мин}{что~делать дальше}

\questionI{1. Что~если экран --- это~не~стена между вами, а~дверь? Как~ты можешь войти в~эту дверь вместе с~ребенком?}

\questionI{2. Кем ты~станешь для~своего ребенка, если перестанешь быть «цензором» и~станешь «проводником» в~цифровой мир?}

\questionI{3. Какой один семейный ритуал *без гаджетов* ты~можешь ввести, чтобы он~был в~радость вам обоим, а~не «полезной обязанностью»?}

\questionI{4. Сколько времени в~день ты~тратишь на~мониторинг криптовалютных графиков?}

\questionI{5. Какие каналы и~блогеры определяют твои инвестиционные решения?}

\questionI{6. Какие убеждения о~деньгах и~технологиях влияют на~твои решения о~покупке гаджетов для~ребёнка?}

\questionI{7. Что~на самом деле движет твоим желанием быстро заработать?}

\questionI{8. Какие внутренние пустоты ты~пытаешься заполнить погоней за~криптовалютными трендами?}

\questionI{9. От~какого неприятного чувства о~себе ты~убегаешь, проводя время в~телефоне?}

\questionI{10. Какую часть своей уязвимости ты~прячешь за~цифровыми привычками?}

\questionI{11. Кем ты~ощущаешь себя без~экранов, уведомлений и~цифровых достижений?}


\newpage

\chapter{Криптовалютное FOMO}
\programbreak

\cluster{Цифровой пульс}{5 – 10 мин}{где я~с~криптой}

\questionF{1. Как~часто ты~проверяешь портфель или графики? (Честно: раз~в день, каждый час, при~каждом уведомлении?)}

\questionF{2. Влияет ли~состояние рынка на~твое настроение? Если рынок «красный», как~это отражается на~твоем общении с~близкими?}

\questionF{3. Спишь ли~ты с~телефоном, боясь пропустить движение цены ночью?}


\cluster{Социальное зеркало}{10 – 20 мин}{что~думают другие люди}

\questionE{1. Чьи успехи в~криптовалюте вызывают у~тебя сложные чувства? На~кого ты~подписан?}

\questionE{2. Веришь ли~ты, что~другие зарабатывают «легко и~быстро», пока ты~«упускаешь возможности»?}

\questionE{3. Какое чувство сильнее: радость от~своей прибыли или досада, что~«мог бы~заработать больше»?}


\cluster{Магическое мышление}{10 – 20 мин}{мои мечты о~лёгких деньгах}

\questionE{1. Ты~пришел в~крипту, чтобы сохранить капитал или чтобы «быстро изменить свою жизнь» (сбежать из~реальности)?}

\questionE{2. Как~ты относишься к~криптоинвестированию --- как~к продуманной стратегии или как~к азартной игре?}

\questionE{3. Что~дает тебе ощущение «быть в~рынке» (причастность к~будущему, азарт, чувство превосходства)?}


\cluster{Identity}{10 – 20 мин}{кто~я~без~хайпа}

\questionE{1. Кто~ты~без~своих виртуальных активов и~цифровых достижений?}

\questionE{2. Если завтра весь крипторынок исчезнет, какие твердые навыки и~ценности останутся у~тебя?}


\cluster{Крипто-детокс}{15 – 30 мин}{как~перестать следить}

\questionI{1. Сколько денег ты~готов потерять без~сожаления и~влияния на~качество жизни? (Твоя реальная толерантность к~риску)}

\questionI{2. Какие правила цифровой гигиены ты~готов ввести, чтобы вернуть себе спокойный сон?}



\newpage
\thispagestyle{empty}
\vspace*{0.2\textheight}

{\centering\sffamily\bfseries\fontsize{24pt}{28pt}\selectfont В завершение\par}

\vspace{2em}

Вы прошли долгий путь.

Не важно, ответили вы на все вопросы или только на часть. Важно, что вы начали этот разговор с собой.

\vspace{1.5em}
{\sffamily\bfseries Что делать дальше}

\textbf{Перечитайте записи.} Через неделю, через месяц. Вы удивитесь, как изменится восприятие.

\textbf{Замечайте паттерны.} Какие темы повторяются? Какие вопросы было сложнее всего отвечать? Там — ваши точки роста.

\textbf{Возвращайтесь.} Эта книга не одноразовая. Вы меняетесь. Ваши ответы будут меняться вместе с вами.

\textbf{Действуйте.} Понимание — это первый шаг. Но настоящие изменения происходят через действие.

\vspace{2em}
{\centering\itshape Самопознание — это не пункт назначения, а путь.\par}

\vspace{1em}
{\centering\itshape Вы уже на нём.\par}

\vspace{3em}
{\centering\sffamily Selfology — искусство понимать себя.\par}

\end{document}