% Selfology Book 2: Путь в глубину
% Кластеры организованы по уровню глубины
% Сгенерировано: 2025-12-01 22:06

\documentclass[11pt,a4paper,oneside]{book}
\usepackage{selfology-book}

\begin{document}


\thispagestyle{empty}
\vspace*{0.2\textheight}

{\centering\sffamily\bfseries\fontsize{24pt}{28pt}\selectfont Добро пожаловать\par}

\vspace{2em}

Эта книга — путешествие вглубь себя.

Вы будете двигаться постепенно: от простого к сложному, от поверхности к глубине.

\vspace{1.5em}
{\sffamily\bfseries Три уровня погружения}

\textbf{Часть I: Первые шаги} — мягкое начало. Знакомство с собой через простые вопросы.

\textbf{Часть II: Исследование} — погружение глубже. Здесь начинается настоящая работа.

\textbf{Часть III: Глубинная работа} — самые важные вопросы. Они требуют сил и безопасного пространства.

\vspace{1.5em}
{\sffamily\bfseries Метафора аквариума}

Представьте, что ваш разум — это аквариум.

Наверху — кристально чистая вода. Это ваше сознание — лишь 10–20\%.

Остальные 80–90\% — мутная вода на глубине. Это ваше подсознание — место, где живут настоящие причины ваших решений.

Эта книга — как сито. Вы будете доставать со дна мысли и чувства, которые обычно остаются невидимыми.

\newpage
\thispagestyle{empty}
\vspace*{0.1\textheight}

{\sffamily\bfseries Как проходить}

\textbf{Создайте пространство.} Выберите время, когда вас никто не потревожит.

\textbf{Двигайтесь последовательно.} Начните с Части I и постепенно углубляйтесь.

\textbf{Пишите от руки.} Когда вы пишете рукой, мозг обрабатывает информацию глубже.

\textbf{Не торопитесь.} Делайте перерывы между частями.

\vspace{1.5em}
{\sffamily\bfseries Время на разделы}

\textbf{5–10 минут} — вопросы Части I

\textbf{10–20 минут} — вопросы Части II

\textbf{15–30 минут} — вопросы Части III

\vspace{2em}
{\centering\itshape Готовы? Тогда начнём.\par}

\part{Часть I: Первые шаги}
\partsubtitle{Мягкое начало — знакомство с собой}
\programbreak

\clusterWithProgram{Здесь и сейчас}{5--10 мин}{ориентация}{Подумать о жизни}

\questionF{1. Как обычно выглядит твоё утро?}

\questionF{2. Что ты делаешь в выходные, когда никто не требует от тебя ничего?}


\clusterWithProgram{Где я сейчас}{5--10 мин}{текущая точка}{Подумать о карьере или бизнесе}

\questionF{1. В какой сфере ты сейчас работаешь или учишься?}

\questionF{2. Есть ли у тебя профильное образование в твоей текущей сфере?}

\questionF{3. Какой у тебя график работы сейчас?}


\clusterWithProgram{Сон и восстановление}{5--10 мин}{фундамент}{Задуматься о здоровье}

\questionF{1. Во сколько обычно ты ложишься спать и во сколько встаёшь?}

\questionF{2. Как ты обычно чувствуешь себя по утрам: бодро, вяло или сонливо?}

\questionF{3. Что ты обычно делаешь за час до сна? Помогает ли это тебе заснуть?}


\clusterWithProgram{Архетипы и модели}{5--10 мин}{откуда я это взял}{Изучить себя}

\questionF{1. Кто оказал на тебя самое сильное влияние? Какие его качества ты в себе видишь?}


\clusterWithProgram{Карта эмоций}{5--10 мин}{что я чувствую}{Улучшить эмоциональное состояние}

\questionF{1. Какие эмоции ты чувствуешь чаще всего в течение обычного дня?}

\questionF{2. Есть ли эмоция, которая живет в твоем теле постоянно, как фоновый шум?}


\clusterWithProgram{Инвентаризация целей}{5--10 мин}{что у меня есть}{Перебрать цели}

\questionF{1. Перечисли все цели, которые ты сейчас преследуешь (профессиональные, личные, финансовые).}

\questionF{2. Откуда появилась каждая из этих целей: это твой выбор или ожидание других людей?}

\questionF{3. От какой цели пора отказаться  она уже не служит тебе и не вдохновляет?}


\clusterWithProgram{Что такое мечта для меня}{5--10 мин}{определение}{Мечтатели}

\questionF{1. Вспомни детство: о чём ты мечтал тогда? Что изменилось с тех пор?}

\questionF{2. Какая твоя главная мечта сейчас? Опиши её одним предложением.}

\questionF{3. Насколько ты живёшь сейчас жизнью своей мечты? На шкале от 1 до 10.}


\clusterWithProgram{Что я узнал о себе}{5--10 мин}{итоги}{Рефлексия}

\questionF{1. Что ты узнал о себе, чего не знал до рефлексии?}

\questionF{2. Какое убеждение о себе ты переоценил или изменил?}


\clusterWithProgram{Принятие}{5--10 мин}{первый кит}{3 кита очищения}

\questionF{1. Что в себе ты долго отрицал, но в конце концов принял?}

\questionF{2. Как изменилась твоя жизнь, когда ты перестал бороться с этим?}


\clusterWithProgram{Люди как ресурс}{5--10 мин}{моя поддержка}{Ресурс}

\questionF{1. Кто оказал на тебя самое сильное влияние? Что именно он сделал?}

\questionF{2. На кого ты можешь положиться в сложный момент? Почему именно на этих людей?}


\clusterWithProgram{Что такое граница для меня}{5--10 мин}{определение}{Границы личности}

\questionF{1. Где ты часто позволяешь людям переступать через свои границы?}

\questionF{2. С кем тебе легче выстроить границы, а с кем сложнее? Почему?}


\clusterWithProgram{Знакомство со страхом}{5--10 мин}{что я чувствую}{Работа со страхами}

\questionF{1. Какой страх живет в тебе так давно, что стал частью тебя?}

\questionF{2. Где в теле ты чувствуешь свой страх?}

\questionF{3. Какой сценарий будущего пугает тебя больше всего?}


\clusterWithProgram{Карта моих отношений}{5--10 мин}{кто рядом}{Разобраться в отношениях}

\questionF{1. Сколько людей знают тебя настоящего? Кто они?}

\questionF{2. С кем в твоей жизни ты чувствуешь безопасность быть собой?}

\questionF{3. Насколько ты доволен своими отношениями с друзьями? На шкале от 1 до 10.}


\clusterWithProgram{Признаки выгорания}{5--10 мин}{что происходит со мной}{Выгорание  Ресурс}

\questionF{1. Что происходит с твоим телом, когда ты истощен? Какие первые сигналы подает организм?}

\questionF{2. Какая эмоция прячется за твоей хронической усталостью?}

\questionF{3. Когда твои цели питают тебя энергией, а когда начинают истощать?}


\clusterWithProgram{Признание того, что было}{5--10 мин}{что произошло}{Исцеление прошлого}

\questionF{1. Какое событие из прошлого до сих пор влияет на тебя?}

\questionF{2. Какой урок ты извлёк из самого сложного периода своей жизни?}

\questionF{3. Чему важному ты научился за последний год?}


\clusterWithProgram{Карта телесных ощущений}{5--10 мин}{где я чувствую}{Тело и эмоции}

\questionF{1. Где в теле ты чувствуешь радость? Опиши ощущение: тепло, легкость, расширение.}

\questionF{2. Вспомни момент, когда тебя переполняла благодарность. Что ты чувствовал в теле?}

\questionF{3. Вспомни эмоцию, которая приносит тебе покой. Как она ощущается в теле?}

\questionF{4. Где в твоем теле живет самая глубокая эмоция, которую ты не можешь выразить словами?}


\clusterWithProgram{Происхождение убеждений}{5--10 мин}{откуда я это взял}{Деньги и самоценность}

\questionF{1. Какие финансовые уроки усвоил от родителей? Хорошие и плохие.}

\questionF{2. Как твои родители относились к деньгам? Как это повлияло на тебя?}


\clusterWithProgram{Текущие убеждения}{5--10 мин}{что я сейчас верю}{Деньги и самоценность}

\questionF{1. Какие убеждения о деньгах ты имеешь прямо сейчас?}

\questionF{2. Ты веришь, что деньги  это добро или зло? Почему?}


\clusterWithProgram{Текущее положение}{5--10 мин}{где я сейчас}{Деньги и самоценность}

\questionF{1. Где ты находишься в плане финансов прямо сейчас? На шкале от 1 до 10.}

\questionF{2. Куда конкретно ты инвестируешь свои деньги: в будущее (образование, активы, инвестиции) или в настоящее (комфорт, развлечения)?}


\clusterWithProgram{Ожидания vs реальность}{5--10 мин}{что произошло}{Кризис 30/40/50}

\questionF{1. Кем ты представлял себя в этом возрасте, когда был молодым?}

\questionF{2. Кто ты на самом деле сейчас? Совпадает ли это с ожиданиями?}

\questionF{3. Что произошло по-другому, чем ты планировал?}


\clusterWithProgram{Текущий кризис}{5--10 мин}{что не работает}{Кризис 30/40/50}

\questionF{1. Что в твоей жизни сейчас вызывает кризис или ощущение застоя?}

\questionF{2. С какой утратой ты сейчас справляешься: мечты, идентичности, молодости, отношений?}


\clusterWithProgram{Моя ценность}{5--10 мин}{что я уже вношу}{AI-тревожность и будущее работы}

\questionF{1. Какие профессиональные навыки делают тебя незаменимым на текущей работе?}

\questionF{2. Что в твоей работе невозможно заменить машиной или AI?}

\questionF{3. Чем ты занимаешься сверх должностной инструкции, что ценят твои коллеги?}


\clusterWithProgram{Осознание потребления}{5--10 мин}{что я делаю}{Инфо-ожирение}

\questionF{1. Сколько времени в день ты проводишь в цифровых потоках (социальные сети, новости, мессенджеры)?}

\questionF{2. Какие три приложения первыми открываются утром?}

\questionF{3. Открываешь ли ты приложения с осознанной целью или просто по привычке?}


\clusterWithProgram{Где я теряю силу}{5--10 мин}{источники беспомощности}{Выученная беспомощность 2.0}

\questionF{1. Что тебя парализует больше всего в моменты, когда нужно действовать?}

\questionF{2. Где заканчивается твой реальный круг влияния? Что находится вне твоего контроля?}

\questionF{3. Что происходит в твоем теле, когда ты чувствуешь бессилие?}


\clusterWithProgram{Картография моего мира}{5--10 мин}{кто рядом на самом деле}{Паразоциальная зависимость}

\questionF{1. Сколько люди в твоей жизни знают тебя по-настоящему, без фильтров?}

\questionF{2. Кто из твоих виртуальных "знакомых" ты хотел бы встретить в реальности?}

\questionF{3. Какой контент успокаивает тебя, когда становится одиноко?}


\clusterWithProgram{Трансформация пространства}{5--10 мин}{как изменилась моя жизнь}{Гибридная жизнь}

\questionF{1. Как изменилось твоё отношение к дому после перехода на гибридный режим?}

\questionF{2. Что ты потерял, когда офис стал твоей спальней?}


\clusterWithProgram{Картография масок}{5--10 мин}{какие версии я создаю}{Аутентичность vs Алгоритмы}

\questionF{1. Сколько разных версий себя ты создал в разных соцсетях или контекстах?}

\questionF{2. Сколько раз в день ты редактируешь, фильтруешь или скрываешь свой образ?}


\clusterWithProgram{Моё потребление}{5--10 мин}{что я беру из мира}{Эко-вина и климат-тревога}

\questionF{1. Какие товары и услуги ты потребляешь ежедневно? Задумываешься ли об их источнике?}

\questionF{2. Какие потребительские привычки ты унаследовал от родителей?}

\questionF{3. Сколько "экологической стоимости" имеет твой типичный день?}


\clusterWithProgram{Симптомы и триггеры}{5--10 мин}{где болит}{Синдром самозванца в эпоху LinkedIn}

\questionF{1. В какие моменты ты чаще всего чувствуешь, что "обманываешь" окружающих своим профессионализмом?}


\clusterWithProgram{Заземление}{5--10 мин}{выход из головы}{Воскресная тревога}

\questionF{1. Заметил ли ты, в какой конкретно момент отдых перестал быть отдыхом и включилось ожидание понедельника?}

\questionF{2. Критикуешь ли ты себя за то, как провел эти выходные?}


\clusterWithProgram{Честная диагностика}{5--10 мин}{без осуждения}{Родительская вина за экранное время}

\questionF{1. Сколько часов в день твой ребенок проводит перед экраном, и сколько из них  для твоего отдыха?}

\questionF{2. Сколько времени в телефоне проводишь ты сам, когда ребенок находится в одной комнате с тобой?}


\clusterWithProgram{Цифровой пульс}{5--10 мин}{диагностика состояния}{Криптовалютное FOMO}

\questionF{1. Как часто ты проверяешь портфель или графики? (Честно: раз в день, каждый час, при каждом уведомлении?)}


\newpage

\part{Часть II: Исследование}
\partsubtitle{Погружение — исследование паттернов}
\programbreak

\clusterWithProgram{Люди рядом}{10--20 мин}{экосистема влияний}{Подумать о жизни}

\questionE{1. Назови одного человека, который оказал на тебя самое сильное влияние. Что именно он сделал?}

\questionE{2. О чём ты никогда не говоришь с семьёй?}

\questionE{3. В чем ты кардинально отличаешься от своей семьи?}


\clusterWithProgram{Мечты и амбиции}{10--20 мин}{направление}{Подумать о жизни}

\questionE{1. Какой идеальный день отдыха ты можешь представить? Где, с кем и что ты там делаешь?}

\questionE{2. О чём ты мечтаешь, когда представляешь финансовую свободу?}

\questionE{3. Куда хочешь побывать и почему именно туда?}

\questionE{4. Ради чего ты готов вставать в самый ранний час?}


\clusterWithProgram{Ценности и убеждения}{10--20 мин}{кто я}{Подумать о жизни}

\questionE{1. Какие ценности ты для себя сформулировал? Выпиши хотя бы три.}

\questionE{2. Закончи фразу: "Если бы люди знали, что я на самом деле... они бы..."}

\questionE{3. Почему ты себя любишь или не любишь? На что именно это влияет?}


\clusterWithProgram{Защита и самообман}{10--20 мин}{что я избегаю}{Подумать о жизни}

\questionE{1. Что тебе нужно освободить или отпустить перед новыми свершениями?}


\clusterWithProgram{Прошлое}{10--20 мин}{что было}{Подумать о жизни}

\questionE{1. Кем ты был год назад? Кто ты сейчас? Как изменился?}

\questionE{2. Оцени прошедший год по основным сферам жизни от 1 до 10}

\questionE{3. Вспомни 3 привычки, которые тебе помогли в этом году. Как они работали?}

\questionE{4. Чего ты не смог достичь в этом году и почему? Что ты об этом думаешь сейчас?}


\clusterWithProgram{Удовлетворённость}{10--20 мин}{что я чувствую}{Подумать о карьере или бизнесе}

\questionE{1. Насколько по шкале от 1 до 10 тебя удовлетворяет твоя работа сейчас?}

\questionE{2. Любишь ли ты свою работу? Что именно любишь или не любишь?}

\questionE{3. Насколько твои личные ценности совпадают с культурой организации, где ты работаешь?}


\clusterWithProgram{Цели и мечты}{10--20 мин}{куда я хочу}{Подумать о карьере или бизнесе}

\questionE{1. Какая она  работа твоей мечты?}

\questionE{2. Чем бы ты занимался, если бы вся работа оплачивалась одинаково?}

\questionE{3. Что бы ты делал, если бы деньги вообще не были ограничением?}

\questionE{4. Представь: через 5 лет встречаешь старого друга. Что о своей работе расскажешь с гордостью?}


\clusterWithProgram{Преграды и ресурсы}{10--20 мин}{что мешает и что помогает}{Подумать о карьере или бизнесе}

\questionE{1. Что именно в твоей нынешней работе мешает тебе двигаться к профессиональным целям?}

\questionE{2. Если тебе нужен наставник в работе, кто это мог бы быть и как ты можешь его найти?}

\questionE{3. Если завтра ты решишь не работать, сколько времени сможешь прожить на свои сбережения?}

\questionE{4. Куда конкретно сейчас идут твои деньги: в будущее (инвестиции, обучение, активы) или в настоящее (покупки, развлечения, комфорт)?}


\clusterWithProgram{Выборы и действия}{10--20 мин}{что я готов сделать}{Подумать о карьере или бизнесе}

\questionE{1. Если поставить на чашу весов отказы на собеседованиях и бесконечную работу там, где тебя не ценят  что перевесит для тебя?}

\questionE{2. Почему ты думаешь, что твоя текущая идея (проект/путь) сработает? На каких данных или опыте это основано?}

\questionE{3. Если бы у тебя были все ресурсы (время, деньги, связи), чем бы ты занимался и как бы проводил рабочее время?}


\clusterWithProgram{Физиология}{10--20 мин}{питание и движение}{Задуматься о здоровье}

\questionE{1. Что ты обычно ешь на завтрак и почему именно это?}

\questionE{2. Какие регулярные физические активности ты практикуешь?}

\questionE{3. Занимаешься ли ты спортом? Если да  почему это важно для тебя? Если нет  что тебя останавливает?}


\clusterWithProgram{Сигналы тела}{10--20 мин}{что говорит организм}{Задуматься о здоровье}

\questionE{1. Что происходит с твоим телом, когда ты истощен? Какие первые сигналы подает организм?}

\questionE{2. Где в теле ты держишь стресс и напряжение? Как это влияет на твой день?}


\clusterWithProgram{Энергия и ресурсы}{10--20 мин}{как я функционирую}{Изучить себя}

\questionE{1. Когда последний раз ты чувствовал полную жизненную силу и энергию? Что происходило в твоей жизни?}


\clusterWithProgram{Ценности через горделость}{10--20 мин}{что я создал}{Изучить себя}

\questionE{1. Чем ты гордишься больше всего? Почему это важно именно для тебя?}


\clusterWithProgram{Желания и страхи}{10--20 мин}{полюса моего я}{Изучить себя}

\questionE{1. Чего ты хочешь прямо сейчас больше всего?}

\questionE{2. Чего ты боишься больше всего в глубине души? Что это страхование в действительности защищает?}


\clusterWithProgram{Телесность эмоций}{10--20 мин}{где я это чувствую}{Улучшить эмоциональное состояние}

\questionE{1. Где в теле ты чувствуешь радость? Опиши ощущение: тепло, легкость, расширение?}

\questionE{2. Вспомни момент, когда тебя переполняла благодарность. Что ты чувствовал в теле?}

\questionE{3. Какая эмоция приносит тебе состояние покоя? Как она ощущается в твоем теле?}

\questionE{4. Где в твоем теле живет эмоция, которую ты не можешь выразить словами?}


\clusterWithProgram{Осознание чувств}{10--20 мин}{как я к ним отношусь}{Улучшить эмоциональное состояние}

\questionE{1. Насколько по шкале от 1 до 10 ты осознаешь свои чувства в момент их возникновения?}

\questionE{2. Есть ли эмоции, которые ты разрешаешь себе чувствовать только наедине, но скрываешь при других?}


\clusterWithProgram{Защиты и маски}{10--20 мин}{как я скрываюсь}{Улучшить эмоциональное состояние}

\questionE{1. Есть ли "безопасные" эмоции, которые ты показываешь вместо настоящих чувств?}


\clusterWithProgram{Эмоции как информация}{10--20 мин}{что они мне говорят}{Улучшить эмоциональное состояние}

\questionE{1. О чём тебе говорят чувства, которые ты регулярно испытываешь?}

\questionE{2. Если бы твоя тревога была мудрым учителем, чему бы она тебя учила?}


\clusterWithProgram{Проверка целей}{10--20 мин}{что реально работает}{Перебрать цели}

\questionE{1. Выбери одну цель, которая для тебя важна. Почему ты думаешь, что она сработает?}

\questionE{2. Может ли быть более эффективный путь к этой же цели? Как ты можешь об этом узнать?}


\clusterWithProgram{Профессиональные цели}{10--20 мин}{карьера и работа}{Перебрать цели}

\questionE{1. Какая твоя работа мечты? Опиши её конкретно: где, что, с кем.}

\questionE{2. Что именно в твоей текущей работе мешает тебе достичь твоих профессиональных целей?}


\clusterWithProgram{Приоритизация}{10--20 мин}{что действительно важно}{Перебрать цели}

\questionE{1. Если ты сможешь реализовать только 3 цели из всех своих  какие это будут?}


\clusterWithProgram{Проверка реальности}{10--20 мин}{это возможно?}{Мечтатели}

\questionE{1. Знаешь ли ты кого-то, кто уже живёт подобной жизнью? Как они это сделали?}

\questionE{2. Если бы возможности были безограничены и ты мог выбрать, что бы ты сделал в первую очередь?}


\clusterWithProgram{Дом мечты}{10--20 мин}{конкретный образ}{Мечтатели}

\questionE{1. Если представить, что все проблемы решены и в жизни полный порядок  где бы ты жил и как?}

\questionE{2. Что конкретно должно быть в твоём доме мечты? Опиши 3-5 деталей.}


\clusterWithProgram{Работа мечты}{10--20 мин}{профессиональное воплощение}{Мечтатели}

\questionE{1. Какая твоя работа мечты? Опиши её конкретно.}

\questionE{2. Кто может помочь тебе перейти к работе мечты? Какие контакты или знания тебе нужны?}


\clusterWithProgram{Деньги и инвестиции}{10--20 мин}{как я их трачу}{Мечтатели}

\questionE{1. Куда конкретно сейчас идут твои деньги: в будущее (инвестиции, обучение, активы) или в настоящее (комфорт, развлечения)?}

\questionE{2. Если бы ты выделил 10\% от своего месячного дохода на мечту, что бы ты сделал?}


\clusterWithProgram{Свобода и изменения}{10--20 мин}{что сдвинулось}{Рефлексия}

\questionE{1. Какая свобода приходит с новым пониманием себя?}

\questionE{2. От чего ты готов отпустить? Что больше не служит тебе?}


\clusterWithProgram{Действие}{10--20 мин}{как я это воплощу}{Рефлексия}

\questionE{1. Какой один поступок ты сделаешь в ближайший день, основываясь на этом новом понимании?}

\questionE{2. Кому ты можешь поделиться этим новым пониманием? Кто поддержит?}


\clusterWithProgram{Отпускание}{10--20 мин}{второй кит}{3 кита очищения}

\questionE{1. От какого убеждения, отношения или человека ты отпустил в этом году?}

\questionE{2. Что ты держишь сейчас, зная, что пора отпустить?}


\clusterWithProgram{Внутренние ресурсы}{10--20 мин}{мои силы}{Ресурс}

\questionE{1. Чем ты гордишься больше всего? Почему это важно для тебя?}

\questionE{2. Какой опыт из прошлого помогает тебе сейчас? Чему он тебя научил?}


\clusterWithProgram{Материальные ресурсы}{10--20 мин}{что я имею}{Ресурс}

\questionE{1. Какие материальные ресурсы у тебя есть? (деньги, имущество, образование, связи)}

\questionE{2. Какой ресурс у тебя избыточный? А какой недостаточный?}


\clusterWithProgram{Времяи энергия}{10--20 мин}{как я их расходую}{Ресурс}

\questionE{1. Какая активность дарует тебе энергию? Как часто ты её практикуешь?}


\clusterWithProgram{Личные границы}{10--20 мин}{что я позволяю себе}{Границы личности}

\questionE{1. От какого желания давно пора отказаться?}

\questionE{2. От какой цели пора отказаться  она уже не служит тебе?}


\clusterWithProgram{Границы в отношениях}{10--20 мин}{кто видит мою слабость}{Границы личности}

\questionE{1. С кем ты можешь быть слабым, а с кем всегда должен быть сильным?}

\questionE{2. Кому ты позволяешь видеть твою слабость и уязвимость?}

\questionE{3. Какие люди требуют от тебя быть всегда "на высоте"? Хочешь ли ты это менять?}


\clusterWithProgram{Границы в работе}{10--20 мин}{мои профессиональные границы}{Границы личности}

\questionE{1. Насколько рабочие обязанности совпадают с твоими интересами и ценностями?}

\questionE{2. Если поставить на весы отказы на собеседованиях и работу там, где тебя не ценят  что перевесит?}

\questionE{3. Когда ты последний раз сказал "нет" дополнительной работе? Что помешало раньше?}


\clusterWithProgram{Границы с технологией и временем}{10--20 мин}{мой ресурс}{Границы личности}

\questionE{1. Установил ли ты границы для интернета в своей жизни? Какие?}


\clusterWithProgram{Граница между работой и личной жизнью}{10--20 мин}{}{Границы личности}

\questionE{1. Есть ли у тебя чёткое время, когда работа заканчивается?}

\questionE{2. Если завтра всё потеряет значение, какое было бы твоё самое большое сожаление?}


\clusterWithProgram{Физические и телесные границы}{10--20 мин}{мои "нет"}{Границы личности}

\questionE{1. Какие физические границы ты четко соблюдаешь? А какие часто нарушаешь?}

\questionE{2. Полюбил ли ты себя настолько, чтобы отказаться от того, что тебя разрушает?}


\clusterWithProgram{Встреча со страхом}{10--20 мин}{что я испытываю}{Работа со страхами}

\questionE{1. Вспомни момент, когда ты встретил свой страх лицом к лицу. Что ты почувствовал?}

\questionE{2. Что мягкого и сострадательного ты можешь сказать своему страху?}


\clusterWithProgram{Глубокий анализ страха}{10--20 мин}{что он защищает}{Работа со страхами}

\questionE{1. Если бы твой страх мог говорить, что бы он сказал о том, что защищает?}

\questionE{2. Какая часть тебя уже не боится того, что пугало раньше?}


\clusterWithProgram{Специфичные страхи}{10--20 мин}{разные виды}{Работа со страхами}

\questionE{1. Какие финансовые страхи не дают тебе спать спокойно?}


\clusterWithProgram{Страх и мотивация}{10--20 мин}{помогает или парализует}{Работа со страхами}

\questionE{1. Твой страх мотивирует тебя двигаться вперёд или парализует тебя?}

\questionE{2. Ты избегаешь ситуаций, потому что они действительно опасны, или потому что боишься узнать что-то о себе?}


\clusterWithProgram{Страх в отношениях}{10--20 мин}{люди и связь}{Работа со страхами}

\questionE{1. Ты даришь людям время и внимание из щедрости или из страха быть отвергнутым?}

\questionE{2. Ты боишься быть уязвимым или боишься того, что твоя уязвимость будет использована против тебя?}

\questionE{3. Если бы ты не боялся одиночества, какие отношения ты бы закончил?}


\clusterWithProgram{Экзистенциальные страхи}{10--20 мин}{большие вопросы}{Работа со страхами}

\questionE{1. Кем ты станешь, когда перестанешь бояться?}

\questionE{2. Кем ты становишься, когда выбираешь любовь вместо страха?}


\clusterWithProgram{Бессознательные страхи}{10--20 мин}{что я прячу}{Работа со страхами}

\questionE{1. От встречи с какой правдой о себе ты прячешься за постоянной занятостью или отвлечениями?}


\clusterWithProgram{Успехи и ресурсы в отношениях}{10--20 мин}{что я создал хорошего}{Разобраться в отношениях}

\questionE{1. Каким успехом в отношениях ты можешь гордиться?}

\questionE{2. Вспомни отношения, где ты чувствовал себя по-настоящему увиденным. Что в них было особенного?}

\questionE{3. Вспомни человека, который принимает тебя полностью. Что ты чувствуешь рядом с ним?}

\questionE{4. Вспомни отношения, которые помогли тебе стать лучше. Что в них было целительного?}


\clusterWithProgram{Паттерны и циклы}{10--20 мин}{что повторяется}{Разобраться в отношениях}

\questionE{1. Что разрушало твои прошлые отношения? Честно.}

\questionE{2. Какие паттерны повторяются в твоих отношениях снова и снова?}


\clusterWithProgram{Я в отношениях}{10--20 мин}{моя роль, мои защиты}{Разобраться в отношениях}

\questionE{1. Какую часть себя ты прячешь в отношениях?}

\questionE{2. С кем ты можешь быть слабым, а с кем всегда должен быть сильным?}

\questionE{3. Где в теле ты чувствуешь, когда отношения не правильные для тебя?}

\questionE{4. Когда ты чувствуешь себя одиноко в окружении людей  это про них или про то, кем ты притворяешься?}


\clusterWithProgram{Любовь и выражение}{10--20 мин}{как я люблю}{Разобраться в отношениях}

\questionE{1. Как ты проявляешь любовь к близким? А как хочешь её получать?}

\questionE{2. Ты даришь людям время и внимание из щедрости или из страха быть отвергнутым?}


\clusterWithProgram{Влияние происхождения}{10--20 мин}{откуда я это беру}{Разобраться в отношениях}

\questionE{1. Как твои отношения с родителями влияют на твои романтические связи?}

\questionE{2. Как часто и качественно ты общаешься с близкими родственниками? Это тебя обогащает?}


\clusterWithProgram{Уязвимость и близость}{10--20 мин}{риск и доверие}{Разобраться в отношениях}

\questionE{1. Ты боишься быть уязвимым или боишься, что твоя уязвимость будет использована против тебя?}

\questionE{2. Если бы ты не боялся одиночества, какие отношения ты бы закончил?}


\clusterWithProgram{Трансформация}{10--20 мин}{кем я становлюсь}{Разобраться в отношениях}

\questionE{1. Кем ты становишься, когда выбираешь любовь вместо страха?}

\questionE{2. Если бы твои отношения могли говорить, что бы они сказали о твоих самых глубоких страхах?}


\clusterWithProgram{Триггеры и паттерны}{10--20 мин}{когда это происходит}{Выгорание  Ресурс}

\questionE{1. Когда ты последний раз чувствовал настоящий отдых  полный, без вины и срочности?}

\questionE{2. Какие обязанности ты берешь на себя из чувства долга, хотя они тебя не вдохновляют?}


\clusterWithProgram{Восстановление  быстрые способы}{10--20 мин}{что помогает сейчас}{Выгорание  Ресурс}

\questionE{1. Какие активности помогают тебе лучше всего восстановить энергию? (музыка, общение, движение, природа)}


\clusterWithProgram{Восстановление  глубокие способы}{10--20 мин}{медленный ресурс}{Выгорание  Ресурс}

\questionE{1. Какой вид отдыха полностью восстанавливает тебя: физический, эмоциональный, творческий, социальный?}

\questionE{2. Что такое для тебя полноценный выходной? Как он должен выглядеть?}

\questionE{3. Какие люди помогают тебе восстанавливать энергию? Как часто ты с ними видишься?}


\clusterWithProgram{Самоподдержка}{10--20 мин}{как я себя поддерживаю}{Выгорание  Ресурс}

\questionE{1. Как ты поддерживаешь себя в моменты усталости, плохого самочувствия или упадка сил?}

\questionE{2. Какой ритуал восстановления ты бы хотел ввести в свой день?}


\clusterWithProgram{Баланс и границы}{10--20 мин}{что нужно изменить}{Выгорание  Ресурс}

\questionE{1. Что ты готов убрать из своей жизни, чтобы создать пространство для восстановления?}

\questionE{2. Как бы изменилась твоя жизнь, если бы ты отдыхал столько же, сколько работаешь?}


\clusterWithProgram{Благодарность}{10--20 мин}{что я получил}{Исцеление прошлого}

\questionE{1. Вспомни момент, когда тебя переполняла благодарность. Что ты чувствовал в теле?}

\questionE{2. Чем ты помог другому человеку, и что это дало тебе?}


\clusterWithProgram{Люди, которые помогали исцелять}{10--20 мин}{поддержка}{Исцеление прошлого}

\questionE{1. Вспомни отношения, которые помогли тебе стать лучше. Что в них было целительного?}

\questionE{2. Вспомни отношения, где ты чувствовал себя по-настоящему увиденным. Что в них было особенного?}

\questionE{3. Вспомни человека, который принимает тебя полностью. Что ты чувствуешь рядом с ним?}

\questionE{4. С кем ты можешь молчать и чувствовать себя понятым?}

\questionE{5. С кем в твоей жизни ты чувствуешь безопасность быть собой?}


\clusterWithProgram{Прощение}{10--20 мин}{отпускание обиды}{Исцеление прошлого}

\questionE{1. Кого ты ещё не простил  других или себя?}

\questionE{2. Если бы твоя грусть могла говорить, что бы она сказала о твоей жизни?}


\clusterWithProgram{Эмоции прошлого}{10--20 мин}{телесная интеграция}{Исцеление прошлого}

\questionE{1. Вспомни эмоцию, которая приносит тебе покой. Как она ощущается в теле?}

\questionE{2. Где в твоем теле живет самая глубокая эмоция, которую ты не можешь выразить словами?}


\clusterWithProgram{Видение себя}{10--20 мин}{гордость и принятие}{Исцеление прошлого}

\questionE{1. Чем ты гордишься больше всего? Почему это важно?}

\questionE{2. Какими своими достижениями ты гордишься? Даже маленькими.}


\clusterWithProgram{Выход из прошлого}{10--20 мин}{движение вперёд}{Исцеление прошлого}

\questionE{1. Чего ты хочешь прямо сейчас больше всего?}

\questionE{2. Куда ты можешь направить свою энергию, чтобы она принесла тебе и миру наибольшую пользу?}


\clusterWithProgram{Сигналы тела}{10--20 мин}{что говорит организм}{Тело и эмоции}

\questionE{1. Какие эмоции ты чувствуешь чаще всего в течение обычного дня?}

\questionE{2. Что происходит с твоим телом в момент, когда эмоция становится слишком сильной?}


\clusterWithProgram{Эмоции и маски}{10--20 мин}{что я скрываю в теле}{Тело и эмоции}

\questionE{1. Есть ли эмоции, которые ты разрешаешь себе чувствовать только наедине, но скрываешь при других?}

\questionE{2. Какую часть себя ты прячешь в теле, физически напрягая мышцы или замораживая?}


\clusterWithProgram{Отношение к чувствам}{10--20 мин}{как я к ним отношусь}{Тело и эмоции}

\questionE{1. Ты позволяешь себе просто чувствовать то, что есть, или часто думаешь "я должен это чувствовать"?}

\questionE{2. Какое настроение ты хотел бы создавать у себя по утрам? Как это ощущается в теле?}


\clusterWithProgram{Глубокие слои}{10--20 мин}{метафора и смысл}{Тело и эмоции}

\questionE{1. Если бы твоя тревога была мудрым учителем, чему бы она тебя учила?}

\questionE{2. Что нежного и доброго ты чувствуешь к себе прямо сейчас?}


\clusterWithProgram{Отношение к деньгам и самоценность}{10--20 мин}{кто я в отношении денег}{Деньги и самоценность}

\questionE{1. Ты чувствуешь себя достойным финансового успеха и изобилия?}

\questionE{2. Если бы ты был финансово свободен, какую стоимость имела бы твоя жизнь?}


\clusterWithProgram{Финансовые страхи}{10--20 мин}{что пугает}{Деньги и самоценность}

\questionE{1. Какие финансовые страхи не дают тебе спать спокойно?}

\questionE{2. Чего ты боишься больше  финансовой неудачи или успеха, который изменит всё?}


\clusterWithProgram{Мечты и мотивация}{10--20 мин}{зачем мне деньги}{Деньги и самоценность}

\questionE{1. О чём ты мечтаешь, когда представляешь финансовую свободу?}

\questionE{2. Что бы ты делал, если бы деньги не были ограничением?}


\clusterWithProgram{Цели и стратегия}{10--20 мин}{куда я иду}{Деньги и самоценность}

\questionE{1. Какую финансовую цель ты хочешь достичь через 1 год? Конкретное число.}

\questionE{2. Кто может помочь тебе в финансовом развитии? Наставник, книга, курс?}


\clusterWithProgram{Мудрость из ошибок}{10--20 мин}{что я узнал}{Кризис 30/40/50}

\questionE{1. Какая мудрость есть в твоих ошибках?}

\questionE{2. Если бы ты мог вернуться в молодость  что бы ты сделал иначе?}

\questionE{3. Чему ты благодарен своему молодому "я" за то, что оно сделало?}


\clusterWithProgram{Переоценка ценностей}{10--20 мин}{что для меня важно}{Кризис 30/40/50}

\questionE{1. Какие ценности тебе казались важными раньше, но потеряли значение?}

\questionE{2. Если бы ты мог прожить следующие 10 лет по-другому  что бы изменилось в твоих приоритетах?}


\clusterWithProgram{Возможность изменения}{10--20 мин}{что я могу сделать}{Кризис 30/40/50}

\questionE{1. Есть ли аспекты твоей жизни, которые ты хочешь кардинально изменить?}

\questionE{2. Чем ты был бы готов пожертвовать, чтобы обновить свою жизнь?}


\clusterWithProgram{Профессиональная идентичность}{10--20 мин}{кто я как специалист}{AI-тревожность и будущее работы}

\questionE{1. Кто ты профессионально помимо технических навыков? Что определяет тебя как специалиста?}

\questionE{2. Какие мягкие навыки (лидерство, коммуникация, креативность) являются твоей главной силой?}


\clusterWithProgram{Признаки тревоги}{10--20 мин}{что я чувствую}{AI-тревожность и будущее работы}

\questionE{1. Чего ты боишься больше: потерять работу или стать профессионально неактуальным?}

\questionE{2. Какие ситуации в работе усиливают твой страх неадекватности?}


\clusterWithProgram{Переобучение и развитие}{10--20 мин}{как я адаптируюсь}{AI-тревожность и будущее работы}

\questionE{1. Сколько времени в неделю ты тратишь на переобучение и развитие?}

\questionE{2. Есть ли у тебя план профессионального развития на случай, если твоя текущая роль автоматизируется?}


\clusterWithProgram{Рынок труда и возможности}{10--20 мин}{какие двери открываются}{AI-тревожность и будущее работы}

\questionE{1. Как ты можешь использовать AI как инструмент, чтобы стать сильнее в своей области?}

\questionE{2. Кто в твоей профессии уже успешно адаптировался к технологиям? Чему ты можешь у них научиться?}


\clusterWithProgram{Чувства и сигналы}{10--20 мин}{что я ощущаю}{Инфо-ожирение}

\questionE{1. Что происходит с твоим телом, когда экран переполняют уведомления?}

\questionE{2. Когда ты ловишь себя на стремлении "проверить что-то"  что на самом деле тебе нужно?}


\clusterWithProgram{Страхи и мотивы}{10--20 мин}{почему я это делаю}{Инфо-ожирение}

\questionE{1. Чего ты боишься пропустить в информационном потоке?}

\questionE{2. Как цифровое потребление помогает тебе избежать чего-то (скуки, одиночества, размышлений)?}


\clusterWithProgram{Граница и сатурация}{10--20 мин}{где я рисую линию}{Инфо-ожирение}

\questionE{1. Где граница между информированностью и информационным перегрузом для тебя?}

\questionE{2. Какой контент действительно полезен тебе, а какой ты потребляешь на автопилоте?}


\clusterWithProgram{От потребления к созданию}{10--20 мин}{переориентирование энергии}{Инфо-ожирение}

\questionE{1. Что случится, если ты перестанешь только потреблять, а начнешь создавать?}

\questionE{2. Как часто ты пишешь, рисуешь, создаёшь что-то своё против времени, которое тратишь на потребление?}


\clusterWithProgram{Идентичность без цифры}{10--20 мин}{кто я на самом деле}{Инфо-ожирение}

\questionE{1. Кто ты без цифрового следа, оценок и лайков?}

\questionE{2. Если бы завтра все твои цифровые аккаунты исчезли  что бы изменилось в том, кто ты есть?}


\clusterWithProgram{Независимость}{10--20 мин}{существую ли я вне потока}{Инфо-ожирение}

\questionE{1. Когда ты последний раз размышлял о чём-то глубоко, не обращаясь к Google?}


\clusterWithProgram{Информационная ловушка}{10--20 мин}{где я сдаюсь}{Выученная беспомощность 2.0}

\questionE{1. Какие источники новостей и информации формируют твою картину мира?}

\questionE{2. Чьи мнения и нарративы ты принимаешь за правду без проверки?}


\clusterWithProgram{Страхи и выбор}{10--20 мин}{почему я не действую}{Выученная беспомощность 2.0}

\questionE{1. Чего ты боишься больше: неудачи или бездействия?}

\questionE{2. От какой надежды ты уже отказался? Хочешь ли ты её вернуть?}


\clusterWithProgram{Иллюзия контроля}{10--20 мин}{что я пытаюсь контролировать}{Выученная беспомощность 2.0}

\questionE{1. Что будет, если ты перестанешь пытаться контролировать то, что находится вне твоего контроля?}

\questionE{2. Если бы ты сосредоточился только на том, что в твоей власти  как бы это изменило твою жизнь?}


\clusterWithProgram{Части личности и сопротивление}{10--20 мин}{кто боится}{Выученная беспомощность 2.0}

\questionE{1. Какая часть тебя боится действовать? Что она защищает?}

\questionE{2. Если бы одна часть тебя боялась, а другая была готова действовать  что бы сказала готовая часть?}


\clusterWithProgram{Компетентность и история успеха}{10--20 мин}{когда я могу}{Выученная беспомощность 2.0}

\questionE{1. Когда последний раз ты почувствовал себя компетентным и способным что-то изменить?}

\questionE{2. В какой сфере жизни ты чувствуешь себя наиболее влиятельным и способным?}


\clusterWithProgram{От беспомощности к агентности}{10--20 мин}{кто я становлюсь}{Выученная беспомощность 2.0}

\questionE{1. Кто ты, когда ты не парализован беспомощностью?}

\questionE{2. Если бы ты взял ответственность за те части жизни, которые ты можешь контролировать  как это изменило бы твой выбор?}


\clusterWithProgram{Источники близости}{10--20 мин}{откуда я беру поддержку}{Паразоциальная зависимость}

\questionE{1. От какого блогера, персонажа или AI ты чувствуешь поддержку? Почему именно они?}

\questionE{2. Какие эмоции замещает для тебя общение с виртуальным собеседником?}


\clusterWithProgram{Побег и избегание}{10--20 мин}{от чего я убегаю}{Паразоциальная зависимость}

\questionE{1. От чего убегаешь, когда проваливаешься в бесконечный скролл?}

\questionE{2. Какие реальные люди или ситуации ты избегаешь через экран?}


\clusterWithProgram{Парадокс близости}{10--20 мин}{почему виртуальное кажется проще}{Паразоциальная зависимость}

\questionE{1. Почему виртуальные отношения кажутся тебе проще, чем реальные?}

\questionE{2. Какие риски и сложности есть в виртуальной близости?}


\clusterWithProgram{Подлинное "я"}{10--20 мин}{кто я на самом деле}{Паразоциальная зависимость}

\questionE{1. Когда последний раз ты чувствовал настоящую близость  без экрана, без фильтров?}

\questionE{2. Кто ты без цифровых масок, фильтров и виртуальных аватаров?}

\questionE{3. Если бы завтра все твои виртуальные отношения исчезли  изменилось бы что-то в том, кто ты есть?}


\clusterWithProgram{Восстановление подлинности}{10--20 мин}{как я возвращаюсь в реальность}{Паразоциальная зависимость}

\questionE{1. Что нужно изменить в твоей жизни, чтобы иметь более подлинные отношения?}

\questionE{2. Кому из реальных людей ты бы хотел рассказать то, что рассказываешь виртуальному собеседнику?}


\clusterWithProgram{Усталость и триггеры}{10--20 мин}{что меня истощает}{Гибридная жизнь}

\questionE{1. Сколько виртуальных встреч в неделю начинают истощать твои внутренние ресурсы?}

\questionE{2. Какие цифровые привычки стали для тебя невидимыми оковами?}


\clusterWithProgram{Жертвы и приоритеты}{10--20 мин}{что я теряю}{Гибридная жизнь}

\questionE{1. Чем ты жертвуешь, чтобы быть постоянно онлайн?}

\questionE{2. Сколько энергии ты тратишь на поддержание профессионального образа 24/7?}


\clusterWithProgram{Границы}{10--20 мин}{где я рисую линию}{Гибридная жизнь}

\questionE{1. Где твои настоящие границы между работой и личной жизнью?}

\questionE{2. Какой ритуал или обряд обозначает переход от работы к личной жизни?}


\clusterWithProgram{Разделение ролей}{10--20 мин}{где одно, где другое}{Гибридная жизнь}

\questionE{1. Где заканчивается твоя профессиональная роль и начинается личность?}

\questionE{2. Какие части себя ты не показываешь на работе, даже в гибридном режиме?}


\clusterWithProgram{Идентичность и цифровой след}{10--20 мин}{кто я в этой жизни}{Гибридная жизнь}

\questionE{1. Кто ты без статусов, уведомлений и профессиональной роли?}

\questionE{2. Если бы завтра ты потерял интернет на неделю  кем бы ты себя обнаружил?}


\clusterWithProgram{Возможности гибридности}{10--20 мин}{что хорошо в этой жизни}{Гибридная жизнь}

\questionE{1. Какие преимущества гибридной жизни ты уже ценишь?}

\questionE{2. Какая свобода появилась благодаря удалённой работе?}


\clusterWithProgram{Контент и искренность}{10--20 мин}{что я даю миру}{Аутентичность vs Алгоритмы}

\questionE{1. Какой контент ты создаёшь от скуки, давления или желания нравиться?}

\questionE{2. Отличается ли ты в реальности от того, как ты выглядишь в сети? Как?}


\clusterWithProgram{Риски аутентичности}{10--20 мин}{что я боюсь показать}{Аутентичность vs Алгоритмы}

\questionE{1. Какие части твоей личности становятся невидимыми в цифровом пространстве?}

\questionE{2. Чем ты рискуешь, когда показываешь настоящие, уязвимые эмоции онлайн?}


\clusterWithProgram{Граница между презентацией и цензурой}{10--20 мин}{где я рисую линию}{Аутентичность vs Алгоритмы}

\questionE{1. Где граница между здоровой самопрезентацией и самоцензурой для тебя?}

\questionE{2. Есть ли люди или сообщества, перед которыми ты можешь быть полностью собой?}


\clusterWithProgram{Выбор между принятием и аутентичностью}{10--20 мин}{что я выбираю}{Аутентичность vs Алгоритмы}

\questionE{1. Когда тебе приходится выбирать между аутентичностью и принятием  как ты выбираешь?}

\questionE{2. Если бы ты был совершенно анонимен онлайн  кем бы ты была/был?}


\clusterWithProgram{Подлинность в эпоху алгоритмов}{10--20 мин}{кто я на самом деле}{Аутентичность vs Алгоритмы}

\questionE{1. Где твоя подлинность сейчас  онлайн или офлайн?}

\questionE{2. Если бы твоя онлайн-версия и реальная версия встретились  что бы они сказали друг другу?}


\clusterWithProgram{Восстановление аутентичности}{10--20 мин}{как я возвращаюсь к себе}{Аутентичность vs Алгоритмы}

\questionE{1. Какую маску ты готов сбросить прямо сейчас?}

\questionE{2. Какой один "настоящий" контент ты бы создал, если бы не боялся?}


\clusterWithProgram{Вина и ответственность}{10--20 мин}{что я чувствую}{Эко-вина и климат-тревога}

\questionE{1. Сколько энергии ты тратишь на экологическую вину вместо действия?}

\questionE{2. Где заканчивается твоя личная ответственность и начинается системная?}


\clusterWithProgram{Эмоции и отрицание}{10--20 мин}{что я чувствую, но не признаю}{Эко-вина и климат-тревога}

\questionE{1. Какую боль о природе, о её страдании ты прячешь внутри себя?}

\questionE{2. Что ты рационализируешь или отрицаешь, когда дело касается экологии?}


\clusterWithProgram{Личный выбор}{10--20 мин}{какая власть у меня есть}{Эко-вина и климат-тревога}

\questionE{1. Какой один личный выбор ты можешь сделать, который имеет реальное значение?}

\questionE{2. Если бы ты знал, что твои действия точно помогут  что бы ты изменил?}


\clusterWithProgram{Идентичность и система}{10--20 мин}{кто я в этой системе}{Эко-вина и климат-тревога}

\questionE{1. Кто ты помимо своих потребительских привычек?}

\questionE{2. Можешь ли ты быть экологичным внутри неэкологичной системы? Как?}


\clusterWithProgram{От вины к действию}{10--20 мин}{как я меняюсь}{Эко-вина и климат-тревога}

\questionE{1. От какой экологической вины ты готов отпустить?}

\questionE{2. Кто или что может помочь тебе в экологических выборах?}


\clusterWithProgram{Присвоение заслуг}{10--20 мин}{факты vs удача}{Синдром самозванца в эпоху LinkedIn}

\questionE{1. Как часто ты списываешь свои успехи на "просто повезло", "совпадение" или "помощь команды"?}

\questionE{2. Выпиши 3 факта о своей карьере, которые невозможно оспорить (цифры, завершенные проекты).}

\questionE{3. Какими своими достижениями ты гордишься по-настоящему, даже если никто не поставил лайк?}


\clusterWithProgram{Разрыв с реальностью}{10--20 мин}{фасад}{Синдром самозванца в эпоху LinkedIn}

\questionE{1. Что самое страшное случится, если ты скажешь: "Я не знаю ответа на этот вопрос"?}


\clusterWithProgram{Ценность и личность}{10--20 мин}{кто я}{Синдром самозванца в эпоху LinkedIn}

\questionE{1. Ты определяешь себя через достижения (что я сделал) или через качества (кто я есть)?}

\questionE{2. Как я могу получать удовольствие от процесса работы, а не только от "галочки" в списке целей?}


\clusterWithProgram{Анализ монстров}{10--20 мин}{чего я боюсь}{Воскресная тревога}

\questionE{1. Какое одно конкретное событие или задача предстоящей недели вызывает у тебя наибольшее напряжение?}

\questionE{2. Есть ли реальная угроза, или это привычка тревожиться перед стартом?}


\clusterWithProgram{Границы и ресурсы}{10--20 мин}{подготовка}{Воскресная тревога}

\questionE{1. Какой маленький ритуал завтра утром поможет тебе начать день мягче?}

\questionE{2. Кто или что будет твоей поддержкой на этой неделе?}


\clusterWithProgram{Анатомия вины}{10--20 мин}{почему мне плохо}{Родительская вина за экранное время}

\questionE{1. Чего ты боишься больше всего: что испортится здоровье или что ребенок потеряет интерес к реальному миру (и к тебе)?}


\clusterWithProgram{Социальное зеркало}{10--20 мин}{откуда берется FOMO}{Криптовалютное FOMO}

\questionE{1. Чьи результаты заставляют тебя чувствовать себя неудачником? На кого ты подписан?}

\questionE{2. Какое чувство сильнее: радость от своей прибыли или досада, что "мог бы заработать больше"?}


\clusterWithProgram{Магическое мышление}{10--20 мин}{зачем я здесь}{Криптовалютное FOMO}

\questionE{1. Ты пришел в крипту, чтобы сохранить капитал или чтобы "быстро изменить свою жизнь" (сбежать из реальности)?}

\questionE{2. Что дает тебе ощущение "быть в рынке" (причастность к будущему, азарт, чувство превосходства)?}


\clusterWithProgram{Identity}{10--20 мин}{кто я без "иксов"}{Криптовалютное FOMO}

\questionE{1. Кто ты без своих виртуальных активов и цифровых достижений?}



\newpage
\thispagestyle{empty}
\vspace*{0.3\textheight}

{\centering\sffamily\bfseries\fontsize{18pt}{22pt}\selectfont Внимание: глубокая работа\par}

\vspace{2em}

Следующий раздел требует особого внимания.

Перед тем как продолжить, убедитесь:

\begin{itemize}
\item Вы находитесь в безопасном месте
\item У вас есть время и силы
\item Вас никто не потревожит
\end{itemize}

\vspace{1em}

Эти вопросы могут вызвать сильные эмоции. Это нормально — значит, вы касаетесь чего-то важного.

Если станет слишком тяжело — остановитесь. Сделайте перерыв. Вернитесь позже.

\vspace{2em}
{\centering\itshape Бережно относитесь к себе.\par}

\newpage

\part{Часть III: Глубинная работа}
\partsubtitle{Трансформация — интеграция открытий}
\programbreak

\clusterWithProgram{Гордость и смысл}{15--30 мин}{что я создал}{Подумать о жизни}

\questionI{1. Чем ты гордишься больше всего?}

\questionI{2. Если бы тебе осталось жить всего один год, как бы ты его провёл?}

\questionI{3. Если бы завтра всё потеряло значение, какое было бы твоё самое большое сожаление?}


\clusterWithProgram{Образ будущего}{15--30 мин}{кто я могу быть}{Подумать о жизни}

\questionI{1. Представь себя в будущем  идеальным в том, что для тебя важно. Какие советы ты бы дал себе сегодняшнему?}

\questionI{2. Если бы родители действительно тебя видели, что бы они поняли обо мне?}


\clusterWithProgram{Бизнес-метрики (для предпринимателей)}{15--30 мин}{}{Подумать о карьере или бизнесе}


\clusterWithProgram{Смысл здоровья}{15--30 мин}{почему это важно}{Задуматься о здоровье}

\questionI{1. Почему для тебя важно иметь хорошее здоровье? Что оно даёт тебе?}


\clusterWithProgram{Здоровье в будущем}{15--30 мин}{целеполагание}{Задуматься о здоровье}

\questionI{1. Что бы ты хотел достичь в своём здоровье за ближайший год?}

\questionI{2. Где ты видишь себя в плане здоровья через 1 год? Как ты будешь себя чувствовать?}

\questionI{3. Какой отдых нужен твоему телу и психике? Где и как ты хотел бы отдохнуть?}


\clusterWithProgram{Идеальное будущее}{15--30 мин}{версия себя}{Изучить себя}

\questionI{1. Опиши свой идеальный день отдыха: где, с кем, что делаешь?}

\questionI{2. Представь: ты изменил мир к лучшему. Что конкретно изменилось благодаря тебе?}


\clusterWithProgram{Развитие и путь}{15--30 мин}{я становлюсь}{Изучить себя}

\questionI{1. Чему ты хочешь научиться в первую очередь?}

\questionI{2. Куда в мире ты хочешь побывать и почему?}


\clusterWithProgram{Настроение как выбор}{15--30 мин}{как я это меняю}{Улучшить эмоциональное состояние}

\questionI{1. Какое настроение ты хотел бы создавать у себя утром перед днём?}


\clusterWithProgram{Достижения и гордость}{15--30 мин}{что я создал}{Перебрать цели}

\questionI{1. Какими своими достижениями ты гордишься больше всего?}


\clusterWithProgram{Действие}{15--30 мин}{как я это буду делать}{Перебрать цели}

\questionI{1. Какой первый конкретный шаг ты можешь сделать на этой неделе ради своей главной цели?}

\questionI{2. Кто или что может помочь тебе достичь этой цели? Какие ресурсы тебе нужны?}


\clusterWithProgram{Действие}{15--30 мин}{что я буду делать прямо сейчас}{Мечтатели}

\questionI{1. Какой один конкретный шаг ты можешь сделать на этой неделе ради своей мечты?}

\questionI{2. Когда ты начнёшь? Назови конкретную дату или день недели.}


\clusterWithProgram{Продолжение пути}{15--30 мин}{что дальше}{Рефлексия}

\questionI{1. Когда ты снова пройдёшь эту рефлексию? Через месяц, квартал, год?}


\clusterWithProgram{Преобразование}{15--30 мин}{третий кит}{3 кита очищения}

\questionI{1. Когда последний раз твой рост шёл не через достижение, а через принятие и отпускание?}

\questionI{2. Что нового в себе ты открыл благодаря этому процессу очищения?}


\clusterWithProgram{Интеграция}{15--30 мин}{как это влияет на будущее}{3 кита очищения}

\questionI{1. Как это новое понимание повлияет на твои решения дальше?}

\questionI{2. Либо удалить оба модуля}

\questionI{3. Либо расширить до полноценных программ (вариант выше)}

\questionI{4. Либо объединить в одну программу "Рефлексия и очищение" (15-20 вопросов)}


\clusterWithProgram{Развитие ресурсов}{15--30 мин}{как я их увеличу}{Ресурс}

\questionI{1. Какой ресурс ты хочешь развить в первую очередь?}

\questionI{2. Кого или что ты можешь попросить, чтобы расширить свои ресурсы?}


\clusterWithProgram{Действие}{15--30 мин}{как я буду выстраивать границы}{Границы личности}

\questionI{1. Какую одну границу ты хочешь выстроить в ближайший месяц?}

\questionI{2. Кого ты попросишь поддержать тебя в выстраивании этой границы?}


\clusterWithProgram{Действие}{15--30 мин}{как я с этим работаю}{Работа со страхами}

\questionI{1. Какой один маленький шаг ты можешь сделать, чтобы встретить свой страх?}

\questionI{2. Какой человек или ресурс поможет тебе в этой работе?}


\clusterWithProgram{Действие}{15--30 мин}{как я буду улучшать отношения}{Разобраться в отношениях}

\questionI{1. Каким образом ты можешь улучшить свои отношения с одним человеком? Первый шаг?}

\questionI{2. Какие качества в себе ты хочешь развить для более крепких отношений?}


\clusterWithProgram{Действие}{15--30 мин}{что я буду делать}{Выгорание  Ресурс}

\questionI{1. Какой один ритуал восстановления ты введешь на этой неделе?}

\questionI{2. Кому ты расскажешь о своём выгорании и попросишь поддержки?}


\clusterWithProgram{Переосмысление личности}{15--30 мин}{кто я благодаря прошлому}{Исцеление прошлого}

\questionI{1. Кем бы ты был, если бы перестал соответствовать ожиданиям других людей?}

\questionI{2. Если бы тебя никто не знал и ты мог начать с чистого листа  кем бы ты решил быть?}

\questionI{3. Если бы тебе разрешили быть абсолютно честным на один день  что бы ты сказал?}


\clusterWithProgram{Действие и будущее}{15--30 мин}{как я двигаюсь дальше}{Исцеление прошлого}

\questionI{1. Какой первый шаг ты сделаешь, чтобы интегрировать этот урок из прошлого?}

\questionI{2. Кому ты расскажешь свою историю исцеления?}


\clusterWithProgram{Регуляция через тело}{15--30 мин}{как я себя успокаиваю}{Тело и эмоции}

\questionI{1. Как ты можешь успокоить свой организм, когда эмоция становится слишком интенсивной?}

\questionI{2. Какое движение или прикосновение помогает тебе вернуться в баланс?}


\clusterWithProgram{Успех и трансформация}{15--30 мин}{кто я становлюсь}{Деньги и самоценность}

\questionI{1. Если бы ты знал, что добьёшься финансового успеха, что бы ты тогда делал?}


\clusterWithProgram{Новый смысл}{15--30 мин}{кем я становлюсь}{Кризис 30/40/50}

\questionI{1. Кем ты хочешь быть в следующие 10-15 лет?}

\questionI{2. Какой новый смысл жизни открывается перед тобой после кризиса?}


\clusterWithProgram{Действие и переход}{15--30 мин}{как я выйду из кризиса}{Кризис 30/40/50}

\questionI{1. Какой один конкретный шаг ты можешь сделать на этой неделе, чтобы начать переход?}

\questionI{2. Кому ты расскажешь о своём кризисе и поищешь поддержки?}

\questionI{3. УДАЛИТЬ программы}

\questionI{4. ГОТОВЫЕ К ПЕЧАТИ И AI}

\questionI{5. ИТОГО}


\clusterWithProgram{Переосмысление работы}{15--30 мин}{что я переосмысляю}{AI-тревожность и будущее работы}

\questionI{1. Если бы AI мог делать твою работу, что бы ты предложил взамен? Какую ценность ты внёс бы?}

\questionI{2. О чём молчит твой профессиональный образ (LinkedIn, резюме)? Что ты скрываешь?}


\clusterWithProgram{Действие и стратегия}{15--30 мин}{как я буду адаптироваться}{AI-тревожность и будущее работы}

\questionI{1. Какой первый конкретный шаг ты сделаешь на этой неделе, чтобы подготовиться к будущему работы?}

\questionI{2. Как ты будешь отслеживать прогресс в адаптации к профессиональным изменениям?}

\questionI{3. "Изучить себя"  полный дубликат "Подумать о жизни"}

\questionI{4. "Тело и эмоции"  дублирует "Здоровье" + "Эмоции" (или использовать специализированный вариант про соматику)}


\clusterWithProgram{Действие и баланс}{15--30 мин}{как я буду менять}{Инфо-ожирение}

\questionI{1. Какой один конкретный цифровой лимит ты установишь для себя на этой неделе?}

\questionI{2. Какую деятельность offline ты хочешь вернуть в свою жизнь?}

\questionI{3. "Изучить себя"  полный дубликат "Подумать о жизни"}

\questionI{4. "Тело и эмоции"  дублирует другие программы (или использовать специализированный вариант)}


\clusterWithProgram{Действие}{15--30 мин}{как я выхожу из беспомощности}{Выученная беспомощность 2.0}

\questionI{1. Какой один конкретный шаг ты можешь сделать сегодня, чтобы почувствовать себя более способным?}

\questionI{2. Какую привычку действия ты введешь, чтобы укрепить своё чувство компетентности?}

\questionI{3. "Изучить себя"  полный дубликат "Подумать о жизни"}

\questionI{4. "Тело и эмоции"  дублирует другие программы}


\clusterWithProgram{Границы и баланс}{15--30 мин}{здоровые виртуальные отношения}{Паразоциальная зависимость}

\questionI{1. Какие виртуальные отношения или контент можно оставить, потому что они тебе действительно помогают?}

\questionI{2. Какую границу между онлайн и офлайном ты хочешь установить?}

\questionI{3. "Изучить себя"  полный дубликат "Подумать о жизни"}

\questionI{4. "Тело и эмоции"  дублирует другие программы}


\clusterWithProgram{Действие и интеграция}{15--30 мин}{как я создам баланс}{Гибридная жизнь}

\questionI{1. Какую одну границу ты установишь на этой неделе между работой и жизнью?}

\questionI{2. Какой ритуал "отключения" от работы ты введешь в свой день?}

\questionI{3. "Изучить себя"  полный дубликат "Подумать о жизни"}

\questionI{4. "Тело и эмоции"  дублирует другие программы}


\clusterWithProgram{Действие и баланс}{15--30 мин}{как я создам аутентичный онлайн}{Аутентичность vs Алгоритмы}

\questionI{1. Какой одной лжи о себе ты откажешься прямо сейчас?}

\questionI{2. Как ты будешь измерять свою аутентичность? По лайкам или по собственному ощущению?}

\questionI{3. "Изучить себя"  полный дубликат "Подумать о жизнь"}

\questionI{4. "Тело и эмоции"  дублирует другие программы}


\clusterWithProgram{Активизм и согласованность}{15--30 мин}{как я участвую}{Эко-вина и климат-тревога}

\questionI{1. Есть ли экологическая причина, которая тебя вдохновляет? Почему?}

\questionI{2. Какую роль ты хочешь играть в экологическом будущем?}

\questionI{3. "Изучить себя"  полный дубликат "Подумать о жизни"}

\questionI{4. "Тело и эмоции"  дублирует другие программы}

\questionI{5. Какие у меня в жизни есть достижения, чем я горжусь?}

\questionI{6. Какими своими достижениями вы гордитесь больше всего?}

\questionI{7. Как я могу получать еще больше удовольствия в процессе достижения своих целей?}

\questionI{8. Ты определяешь себя через достижения или через то, кто ты есть без них?}


\clusterWithProgram{Выход из гонки}{15--30 мин}{новые настройки}{Синдром самозванца в эпоху LinkedIn}

\questionI{1. Что ты сделаешь уже сегодня, зная, что ты достаточно хорош для этой задачи?}

\questionI{2. В какой сфере ты работаешь или учишься?}

\questionI{3. Почему я думаю, что это сработает? Какие основания у меня есть полагать или верить, что это приведет меня к цели? Может ли быть более эффективная идея и как я могу о ней узнать? На каких конкретных данных я основываю свой выбор именно этой идеи из множества других?}

\questionI{4. Куда бы вы хотели пойти сейчас работать?}

\questionI{5. Если поставить на чашу весов отказы на собеседованиях с одной стороны, а с другой бесконечно работать там, где тебя не ценят. Что перевесит?}

\questionI{6. Вы критикуете себя за отдых? Но при этом, вы много работаете и у вас много работы?}

\questionI{7. Если завтра я решу не работать, сколько времени я смогу прожить на свои сбережения?}

\questionI{8. Как моя нынешняя работа помогает мне достичь этих целей?}

\questionI{9. Чем моя нынешняя работа мешает мне достичь этих целей?}

\questionI{10. Насколько по шкале от 1 до 10 меня удовлетворяет моя работа?}

\questionI{11. Работа моей мечты, какая она?}


\clusterWithProgram{Глобальная сверка}{15--30 мин}{если тревога хроническая}{Воскресная тревога}

\questionI{1. Если поставить на весы страх перемен и боль от работы там, где тебя не ценят  что сейчас тяжелее?}

\questionI{2. Сколько часов в день твой ребенок проводит перед экраном?}

\questionI{3. Какие приложения чаще всего выбирает твой ребенок?}

\questionI{4. Что ты чувствуешь, когда видишь ребенка с гаджетом?}

\questionI{5. Какие эмоции скрываются за твоим запретом на гаджеты?}

\questionI{6. Как твой страх технологий влияет на отношения с ребенком?}

\questionI{7. Чего на самом деле ты боишься, когда ограничиваешь экранное время?}

\questionI{8. Какую травму собственного детства ты проецируешь на ребенка?}

\questionI{9. Кем ты станешь как родитель, приняв цифровой мир ребенка?}


\clusterWithProgram{Скрытые смыслы}{15--30 мин}{о чем это на самом деле}{Родительская вина за экранное время}

\questionI{1. Когда ты запрещаешь гаджет или ругаешься из-за него  какую свою тревогу ты пытаешься заглушить?}

\questionI{2. Если вспомнить твое детство: чего тебе не хватало, что ты теперь пытаешься компенсировать (или запретить) ребенку?}


\clusterWithProgram{Цифровой мост}{15--30 мин}{решение}{Родительская вина за экранное время}

\questionI{1. Кем ты станешь для своего ребенка, если перестанешь быть "цензором" и станешь "проводником" в цифровой мир?}

\questionI{2. Сколько времени в день ты тратишь на мониторинг криптовалютных графиков?}

\questionI{3. Какие каналы и блогеры определяют твои инвестиционные решения?}

\questionI{4. Как страх упустить выгодную сделку влияет на твои финансовые решения?}

\questionI{5. Что на самом деле движет твоим желанием быстро заработать?}

\questionI{6. Какие внутренние пустоты ты пытаешься заполнить погоней за криптовалютными трендами?}

\questionI{7. От какого болезненного убеждения о себе ты убегаешь через азарт инвестиций?}

\questionI{8. Какую часть своей уязвимости ты прячешь за криптовалютными спекуляциями?}

\questionI{9. Кто ты без виртуальных активов и цифровых достижений?}


\clusterWithProgram{Крипто-детокс}{15--30 мин}{стратегия выхода}{Криптовалютное FOMO}

\questionI{1. Какие правила цифровой гигиены ты готов ввести, чтобы вернуть себе спокойный сон?}



\newpage
\thispagestyle{empty}
\vspace*{0.2\textheight}

{\centering\sffamily\bfseries\fontsize{24pt}{28pt}\selectfont В завершение\par}

\vspace{2em}

Вы прошли долгий путь — от поверхности до самой глубины.

Не важно, ответили вы на все вопросы или только на часть. Важно, что вы начали этот разговор с собой.

\vspace{1.5em}
{\sffamily\bfseries Что делать дальше}

\textbf{Перечитайте записи.} Через неделю, через месяц. Вы удивитесь, как изменится восприятие.

\textbf{Замечайте паттерны.} Какие темы повторяются? На каком уровне было сложнее всего?

\textbf{Возвращайтесь.} Эта книга не одноразовая. Вы меняетесь. Ваши ответы будут меняться вместе с вами.

\vspace{2em}
{\centering\itshape Самопознание — это не пункт назначения, а путь.\par}

\vspace{1em}
{\centering\itshape Вы уже на нём.\par}

\vspace{3em}
{\centering\sffamily Selfology — искусство понимать себя.\par}

\end{document}